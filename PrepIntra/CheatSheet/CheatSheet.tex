\documentclass{report}
\usepackage[utopia]{mathdesign} 
%\usepackage{amsmath,amsfonts,amsthm,amssymb,mathtools}

\usepackage[french]{babel}

% Permet d'ajuster la taille des marges et de la distance pour les footer
\usepackage[tmargin=2cm,rmargin=0.4in,lmargin=0.4in,bmargin=2cm,footskip=.2in]{geometry}

% Permet d'optimiser l'affichage de différents symboles et formules mathématiques
\usepackage{amsmath,amsfonts,amsthm,amssymb,mathtools}

\usepackage{svg}
% Modifie l'apparence des nombre en mathmode et textmode
%\usepackage[varbb]{newpxmath}

% Modifier l'apparence des fractions
\usepackage{xfrac}

            %%%%%%%%%%%%%%%%%  Sect.        14 Oct 2024     %%%%%%%%%%%%%%%%%%%%%%%%%%%%%%%%%%%%%%%%%%%%%%%%%%%%%%%%%%%
\usepackage{graphicx}
\usepackage{caption}
\usepackage{subcaption}
\usepackage{arydshln}
            %%%%%%%%%%%%%%%%%  Sect.        14 Oct 2024     %%%%%%%%%%%%%%%%%%%%%%%%%%%%%%%%%%%%%%%%%%%%%%%%%%%%%%%%%%%
\usepackage{balance}
\usepackage{dirtree}
\usepackage{titlesec}






% Permet de rayer (barrer) l'argument avec la touche
% \cancel{} \bcancel{} ou \xcancel{}
\usepackage[makeroom]{cancel}

% Extension du package amsmath; corrige certains bugs et déficiences de son prédecesseur
\usepackage{mathtools}

% This package provides most of the flexibility you may want to customize the three basic list
% environments (enumerate, itemize and description)
\usepackage{bookmark} 

% Réorganiser les théorèmes et Lemmes. Usage complexe. 
% Référence : https://ctan.math.illinois.edu/macros/latex/contrib/theoremref/theoremref-doc.pdf
\hypersetup{hidelinks}
\usepackage{hyperref,theoremref} 

% Fournit un environnement pour créer des boîtes colorées
\usepackage[most,many,breakable]{tcolorbox}


%\newcommand\mycommfont[1]{\footnotesize\ttfamily\textcolor{blue}{#1}}\SetCommentSty{mycommfont}

%\newcommand{\incfig}[1]{%\def\svgwidth{\columnwidth}\import{./figures/}{#1.pdf_tex}}
\newcommand{\arc}[1]{\wideparen{#1}}

%Pour colorer les lignes séparatrices de tableaux
\usepackage{colortbl}
\usepackage{tikzsymbols}

\usepackage{framed}
\usepackage{titletoc}
\usepackage{etoolbox}
\usepackage{lmodern}
\usepackage{tabularx}
\usepackage{enumitem}
\usepackage{amsthm}
            %%%%%%%%%%%%%%%%%  Sect.        14 Oct 2024     %%%%%%%%%%%%%%%%%%%%%%%%%%%%%%%%%%%%%%%%%%%%%%%%%%%%%%%%%%%

\usepackage{lipsum}
\usepackage{titling}
\renewcommand\maketitlehooka{\null\mbox{}\vfill}
\renewcommand\maketitlehookd{\vfill\null}

\newcommand{\varitem}[3][black]{%
  \item[%
   \colorbox{#2}{\textcolor{#1}{\makebox(5.5,7){#3}}}%
  ]
}
\usepackage{afterpage}
\newcommand\myemptypage{
    \null
    \thispagestyle{empty}
    \addtocounter{page}{-1}
    \newpage
    }




% from https://tex.stackexchange.com/a/167024/121799
\newcommand{\ClaudioList}{red,DarkOrange1,Goldenrod1,Green3,blue!50!cyan,DarkOrchid2}
\newcommand{\SebastianoItem}[1]{\foreach \X[count=\Y] in \ClaudioList
{\ifnum\Y=#1\relax
\xdef\SebastianoColor{\X}
\fi
}
\tikz[baseline=(SebastianoItem.base),remember
picture]{%
\node[fill=\SebastianoColor,inner sep=4pt,font=\sffamily,fill opacity=0.5] (SebastianoItem){#1)};}
}
\newcommand{\SebastianoHighlight}{\tikz[overlay,remember picture]{%
\fill[\SebastianoColor,fill opacity=0.5] ([yshift=4pt,xshift=-\pgflinewidth]SebastianoItem.east) -- ++(4pt,-4pt)
-- ++(-4pt,-4pt) -- cycle;
}}   
            %%%%%%%%%%%%%%%%%  Sect.        14 Oct 2024     %%%%%%%%%%%%%%%%%%%%%%%%%%%%%%%%%%%%%%%%%%%%%%%%%%%%%%%%%%%





%====================================================================

%====================================================================
\newcommand*{\authorimg}[1]%
    { \raisebox{-1\baselineskip}{\includegraphics[width=\imagesize]{#1}}}
\newlength\imagesize  

\usepackage{pgfplots}
\pgfplotsset{compat=1.17}

%==========================================================================================
\usepackage{libris} 
\usepackage{etoolbox}
\usepackage[export]{adjustbox}% for positioning figures

\makeatletter
% Force le chapitre suivant sur la ligne succedant la fin du 
% chapitre précédent
\patchcmd{\chapter}{\if@openright\cleardoublepage\else\clearpage\fi}{}{}{}
\makeatother
\usepackage[Sonny]{fncychap}


%boîte de couleur grise
\tcbset{
  graybox/.style={
    colback=gray!20,
    colframe=black,
    sharp corners=downhill,
    boxrule=1pt,
    left=5pt,
    right=5pt,
    top=5pt,
    bottom=5pt,
    boxsep=0pt,
	 % <-- add four values for each corner
  }
}
\newtcolorbox{graybox}{graybox}

%==========================================================================================



\usepackage{xcolor}
\usepackage{varwidth}
\usepackage{varwidth}
\usepackage{etoolbox}
%\usepackage{authblk}
\usepackage{nameref}
\usepackage{multicol,array}
\usepackage{tikz-cd}
\usepackage[ruled,linesnumbered,ruled]{algorithm2e}
\usepackage{comment} % enables the use of multi-line comments (\ifx \fi) 
\usepackage{import}
\usepackage{xifthen}
\usepackage{pdfpages}
\usepackage{transparent}


%\usepackage[french]{babel}
\usepackage{listings} % pour écrire du code dans un environnement
\lstset{
  basicstyle=\ttfamily,
  columns=fullflexible,
  keepspaces=true
}
\usepackage{caption}
\usepackage{float} % Pour forcer les images au bon endroit



\usepackage[T1]{fontenc}
\usepackage{csquotes}
%%%%%%%%%%%%%%%%%%%%%%%%%%%%%%%%%%%%%%%%%%%%%%%%%%%%%%%%%%%%%%%%%%%%%%%%%%%%%%%%%%%%%%%%%%%%%%%%%
%									ENSEMBLE DE COULEURS
%%%%%%%%%%%%%%%%%%%%%%%%%%%%%%%%%%%%%%%%%%%%%%%%%%%%%%%%%%%%%%%%%%%%%%%%%%%%%%%%%%%%%%%%%%%%%%%%%

\definecolor{myg}{RGB}{56, 140, 70}
\definecolor{myb}{RGB}{45, 111, 177}

\definecolor{mygbg}{RGB}{235, 253, 241}


\definecolor{myr}{RGB}{199, 68, 64}
\definecolor{mytheorembg}{HTML}{F2F2F9}
\definecolor{mytheoremfr}{HTML}{00007B}
\definecolor{mylenmabg}{HTML}{FFFAF8}
\definecolor{mylenmafr}{HTML}{983b0f}
\definecolor{mypropbg}{HTML}{f2fbfc}
\definecolor{mypropfr}{HTML}{191971}
\definecolor{myexamplebg}{HTML}{F2FBF8}
\definecolor{myexamplefr}{HTML}{88D6D1}
\definecolor{myexampleti}{HTML}{2A7F7F}
\definecolor{mydefinitbg}{HTML}{E5E5FF}
\definecolor{mydefinitfr}{HTML}{3F3FA3}
\definecolor{notesgreen}{RGB}{0,162,0}
\definecolor{myp}{RGB}{197, 92, 212}
\definecolor{mygr}{HTML}{2C3338}
\definecolor{myred}{RGB}{127,0,0}
\definecolor{myyellow}{RGB}{169,121,69}
\definecolor{myexercisebg}{HTML}{F2FBF8}
\definecolor{myexercisefg}{HTML}{88D6D1}
\definecolor{myred}{RGB}{127,0,0}
\definecolor{myyellow}{RGB}{169,121,69}
\definecolor{LightLavender}{HTML}{DFC5FE}

\definecolor{blue}{HTML}{008ED7}
\definecolor{mygray}{gray}{0.75}
\definecolor{lightBlue}{RGB}{235, 245, 255}
\definecolor{tcbcolred}{RGB}{255,0,0}
\definecolor{myGreen}{HTML}{009900}

% command to circle a text
\newtcbox{\entoure}[1][red]{on line,
	arc=3pt,colback=#1!10!white,colframe=#1!50!black,
	before upper={\rule[-3pt]{0pt}{10pt}},boxrule=1pt,
	boxsep=0pt,left=2pt,right=2pt,top=1pt,bottom=.5pt}
% command for the circle for the number of part entries
\newcommand\Circle[1]{\tikz[overlay,remember picture]
	\node[draw,circle, text width=18pt,line width=1pt] {#1};}

\newtcbox{\entouree}[1][red]{on line,
	arc=3pt,colback=#1!10!white,colframe=#1!50!white,
	before upper={\rule[-3pt]{0pt}{10pt}},boxrule=1pt,
	boxsep=0pt,left=2pt,right=2pt,top=1pt,bottom=.5pt}

\newcommand{\shellcmd}[1]{\\\indent\indent\texttt{\footnotesize\# #1}\\}

%=====================================================================

\patchcmd{\tableofcontents}{\contentsname}{\rmfamily\contentsname}{}{}
% patching of \@part to typeset the part number inside a framed box in its own line
% and to add color
\makeatletter
\patchcmd{\@part}
  {\addcontentsline{toc}{part}{\thepart\hspace{1em}#1}}
  {\addtocontents{toc}{\protect\addvspace{20pt}}
    \addcontentsline{toc}{part}{\huge{\protect\color{myyellow}%
      \setlength\fboxrule{2pt}\protect\Circle{%
        \hfil\thepart\hfil%
      }%
    }\\[2ex]\color{myred}\rmfamily#1}}{}{}

%\patchcmd{\@part}
%  {\addcontentsline{toc}{part}{\thepart\hspace{1em}#1}}
%  {\addtocontents{toc}{\protect\addvspace{20pt}}
%    \addcontentsline{toc}{part}{\huge{\protect\color{myyellow}%
%      \setlength\fboxrule{2pt}\protect\fbox{\protect\parbox[c][1em][c]{1.5em}{%
%        \hfil\thepart\hfil%
%      }}%
%    }\\[2ex]\color{myred}\sffamily#1}}{}{}
\makeatother
% this is the environment used to typeset the chapter entries in the ToC
% it is a modification of the leftbar environment of the framed package
\renewenvironment{leftbar}
  {\def\FrameCommand{\hspace{6em}%
    {\color{myyellow}\vrule width 2pt depth 6pt}\hspace{1em}}%
    \MakeFramed{\parshape 1 0cm \dimexpr\textwidth-6em\relax\FrameRestore}\vskip2pt%
  }
 {\endMakeFramed}

% using titletoc we redefine the ToC entries for parts, chapters, sections, and subsections
\titlecontents{part}
  [0em]{\centering}
  {\contentslabel}
  {}{}
\titlecontents{chapter}
  [0em]{\vspace*{2\baselineskip}}
  {\parbox{4.5em}{%
    \hfill\Huge\rmfamily\bfseries\color{myred}\thecontentspage}%
   \vspace*{-2.3\baselineskip}\leftbar\textsc{\small\chaptername~\thecontentslabel}\\\rmfamily}
  {}{\endleftbar}
\titlecontents{section}
  [8.4em]
  {\rmfamily\contentslabel{3em}}{}{}
  {\hspace{0.5em}\nobreak\color{myred}\normalfont\contentspage}
\titlecontents{subsection}
  [8.4em]
  {\rmfamily\contentslabel{3em}}{}{}  
  {\hspace{0.5em}\nobreak\color{myred}\contentspage}


\tcbset{
  tbcsetLavender/.style={
    on line, 
    boxsep=4pt, left=0pt,right=0pt,top=0pt,bottom=0pt,
    colframe=white, colback=LightLavender,  
    highlight math style={enhanced}
  }
}
\tcbset{
  grayb/.style={
    on line, 
    boxsep=4pt, left=0pt,right=0pt,top=0pt,bottom=0pt,
    colframe=white, colback=gray!30,  
    highlight math style={enhanced}
  }
}


%==========================================================================

%PYTHON LSTLISTING STYLE

% Define colors
\definecolor{Pgruvbox-bg}{HTML}{282828}
\definecolor{Pgruvbox-fg}{HTML}{ebdbb2}
\definecolor{Pgruvbox-red}{HTML}{fb4934}
\definecolor{Pgruvbox-green}{HTML}{b8bb26}
\definecolor{Pgruvbox-yellow}{HTML}{fabd2f}
\definecolor{Pgruvbox-blue}{HTML}{83a598}
\definecolor{Pgruvbox-purple}{HTML}{d3869b}
\definecolor{Pgruvbox-aqua}{HTML}{8ec07c}

% Define Python style
\lstdefinestyle{PythonGruvbox}{
	language=Python,
	identifierstyle=\color{lst-fg},
	basicstyle=\ttfamily\color{Pgruvbox-fg},
	keywordstyle=\color{Pgruvbox-yellow},
	keywordstyle=[2]\color{Pgruvbox-blue},
	stringstyle=\color{Pgruvbox-green},
	commentstyle=\color{Pgruvbox-aqua},
	backgroundcolor=\color{Pgruvbox-bg},
	%frame=tb,
	rulecolor=\color{Pgruvbox-fg},
	showstringspaces=false,
	keepspaces=true,
	captionpos=b,
	breaklines=true,
	tabsize=4,
	showspaces=false,
	numbers=left,
	numbersep=5pt,
	numberstyle=\tiny\color{gray},
	showtabs=false,
	columns=fullflexible,
	morekeywords={True,False,None},
	morekeywords=[2]{and,as,assert,break,class,continue,def,del,elif,else,except,exec,finally,for,from,global,if,import,in,is,lambda,nonlocal,not,or,pass,print,raise,return,try,while,with,yield},
	morecomment=[s]{"""}{"""},
	morecomment=[s]{'''}{'''},
	morecomment=[l]{\#},
	morestring=[b]",
	morestring=[b]',
	literate=
	{0}{{\textcolor{Pgruvbox-purple}{0}}}{1}
	{1}{{\textcolor{Pgruvbox-purple}{1}}}{1}
	{2}{{\textcolor{Pgruvbox-purple}{2}}}{1}
	{3}{{\textcolor{Pgruvbox-purple}{3}}}{1}
	{4}{{\textcolor{Pgruvbox-purple}{4}}}{1}
	{5}{{\textcolor{Pgruvbox-purple}{5}}}{1}
	{6}{{\textcolor{Pgruvbox-purple}{6}}}{1}
	{7}{{\textcolor{Pgruvbox-purple}{7}}}{1}
	{8}{{\textcolor{Pgruvbox-purple}{8}}}{1}
	{9}{{\textcolor{Pgruvbox-purple}{9}}}{1}
}
%====================================================================
% 
%====================================================================

% JAVA LSTLISTING STYLE IN Gruvbox Colorscheme
\definecolor{gruvbox-bg}{rgb}{0.282, 0.247, 0.204}
\definecolor{gruvbox-fg1}{rgb}{0.949, 0.898, 0.776}
\definecolor{gruvbox-fg2}{rgb}{0.871, 0.804, 0.671}
\definecolor{gruvbox-red}{rgb}{0.788, 0.255, 0.259}
\definecolor{gruvbox-green}{rgb}{0.518, 0.604, 0.239}
\definecolor{gruvbox-yellow}{rgb}{0.914, 0.808, 0.427}
\definecolor{gruvbox-blue}{rgb}{0.353, 0.510, 0.784}
\definecolor{gruvbox-purple}{rgb}{0.576, 0.412, 0.659}
\definecolor{gruvbox-aqua}{rgb}{0.459, 0.631, 0.737}
\definecolor{gruvbox-gray}{rgb}{0.518, 0.494, 0.471}

\definecolor{lst-bg}{RGB}{45, 45, 45}
\definecolor{lst-fg}{RGB}{220, 220, 204}
\definecolor{lst-keyword}{RGB}{215, 186, 125}
\definecolor{lst-comment}{RGB}{117, 113, 94}
\definecolor{lst-string}{RGB}{163, 190, 140}
\definecolor{lst-number}{RGB}{181, 206, 168}
\definecolor{lst-type}{RGB}{218, 142, 130}


\lstdefinestyle{JavaGruvbox}{
	language=Java,
	basicstyle=\ttfamily\color{lst-fg},
	keywordstyle=\color{lst-keyword},
	keywordstyle=[2]\color{lst-type},
	commentstyle=\itshape\color{lst-comment},
	stringstyle=\color{lst-string},
	numberstyle=\color{lst-number},
	backgroundcolor=\color{lst-bg},
	%frame=tb,
	rulecolor=\color{gruvbox-aqua},
	showstringspaces=false,
	keepspaces=true,
	captionpos=b,
	breaklines=true,
	tabsize=4,
	showspaces=false,
	showtabs=false,
	columns=fullflexible,
	morekeywords={var},
	morekeywords=[2]{boolean, byte, char, double, float, int, long, short, void},
	morecomment=[s]{/}{/},
	morecomment=[l]{//},
	morestring=[b]",
	morestring=[b]',
	numbers=left,
	numbersep=5pt,
	numberstyle=\tiny\color{gray},
}



%====================================================================
% 
%====================================================================


% Define Dracula color scheme for Java
\definecolor{draculawhite-background}{RGB}{237, 239, 252}
\definecolor{draculawhite-comment}{RGB}{98, 114, 164}
\definecolor{draculawhite-keyword}{RGB}{189, 147, 249}
\definecolor{draculawhite-string}{RGB}{152, 195, 121}
\definecolor{draculawhite-number}{RGB}{249, 189, 89}
\definecolor{draculawhite-operator}{RGB}{248, 248, 242}

% Define JavaDraculaWhite lstlisting environment
\lstdefinestyle{JavaDraculaWhite}{
    language=Java,
    backgroundcolor=\color{draculawhite-background},
    commentstyle=\itshape\color{draculawhite-comment},
    keywordstyle=\color{draculawhite-keyword},
    stringstyle=\color{draculawhite-string},
    basicstyle=\ttfamily\footnotesize\color{black},
    identifierstyle=\color{black},
    keywordstyle=\color{draculawhite-keyword}\bfseries,
    morecomment=[s][\color{draculawhite-comment}]{/**}{*/},
    showstringspaces=false,
    showspaces=false,
    breaklines=true,
    frame=single,
    rulecolor=\color{draculawhite-operator},
    tabsize=2,  
	numbers=left,
	numbersep=4pt,
	numberstyle=\ttfamily\tiny\color{gray}
}
%====================================================================
% 
%====================================================================
% Define PythonDraculaWhite lstlisting environment 
\definecolor{draculawhite-bg}{HTML}{FAFAFA}
\definecolor{draculawhite-fg}{HTML}{282A36}
\definecolor{pdraculawhite-keyword}{HTML}{BD93F9}

\definecolor{pdraculawhite-comment}{HTML}{6272A4}
\definecolor{draculawhite-number}{HTML}{FF79C6}


\lstdefinestyle{PythonDraculaWhite}{
    language=Python,
    basicstyle=\ttfamily\small\color{draculawhite-fg},
    backgroundcolor=\color{draculawhite-background},
    keywordstyle=\color{orange}\bfseries,
    stringstyle=\color{draculawhite-string},
    commentstyle=\color{pdraculawhite-comment}\itshape,
    numberstyle=\color{draculawhite-number},
    showstringspaces=false,
	showspaces=false,
    breaklines=true,
	frame=single,
	rulecolor=\color{draculawhite-operator}, 
    tabsize=4,
    morekeywords={as,with,1,2,3,4, 5,6,7,8,9,True,False},
    %escapeinside={(*@}{@*)},
    numbers=left,
    numbersep=5pt,
    %xleftmargin=15pt,
    %framexleftmargin=15pt,
    %framexrightmargin=0pt,
    %framexbottommargin=0pt,
    %framextopmargin=0pt,
    %rulecolor=\color{draculawhite-fg},
    %frame=tb,
    %aboveskip=0pt,
    %belowskip=0pt,
    %captionpos=b,
	numberstyle=\ttfamily\tiny\color{gray} 
}
%====================================================================
% 
%====================================================================

% Define colors for HTML langage
\definecolor{html-orange}{HTML}{FF5733}
\definecolor{html-yellow}{HTML}{F0E130}
\definecolor{html-green}{HTML}{50FA7B}
\definecolor{html-blue}{HTML}{5AFBFF}
\definecolor{html-purple}{HTML}{BD93F9}
\definecolor{html-pink}{HTML}{FF80BF}
\definecolor{html-gray}{HTML}{6272A4}
\definecolor{html-white}{HTML}{F8F8F2}

% Defines a new HTML5 langage that extend on the html langange
\lstdefinestyle{HTMLDraculaWhite}{
  language=HTML,
  backgroundcolor=\color{html-white},
  basicstyle=\ttfamily\color{html-gray},
  keywordstyle=\color{html-blue},
  stringstyle=\color{html-orange},
  commentstyle=\color{html-green},
  tagstyle=\color{html-yellow},
  moredelim=[s][\color{html-pink}]{<!--}{-->},
  moredelim=[s][\color{html-purple}]{\{}{\}},
  showstringspaces=false,
  tabsize=2,
  breaklines=true,
  columns=fullflexible,
  %frame=single,
  framexleftmargin=5mm,
  xleftmargin=10mm,
  numbers=left,
  numberstyle=\tiny\color{html-gray},
  escapeinside={<@}{@>}
}

%====================================================================
% 
%====================================================================
% Define the colors needed for the HTMLDraculaDark environment
\definecolor{htmltag}{HTML}{ff79c6}
\definecolor{htmlattr}{HTML}{f1fa8c}
\definecolor{htmlvalue}{HTML}{bd93f9}
\definecolor{htmlcomment}{HTML}{6272a4}
%\definecolor{htmltext}{HTML}{f8f8f2}
\definecolor{htmltext}{HTML}{401E31}
\definecolor{htmlbackground}{HTML}{282a36}
\definecolor{comphtmlbackground}{HTML}{8093FF}
%\definecolor{htmlbackground}{HTML}{4D5169}

% Define the HTMLDraculaDark environment
\lstdefinestyle{HTMLDraculaDark}{
    basicstyle=\bfseries\ttfamily\color{htmltext},
    commentstyle=\itshape\color{htmlcomment},
    keywordstyle=\bfseries\color{htmltag},
    stringstyle=\color{htmlvalue},
    emph={DOCTYPE,html,head,body,div,span,a,script},
    emphstyle={\color{htmltag}\bfseries},
    sensitive=true,
    showstringspaces=false,
    backgroundcolor=\color{white},
    %frame=tb,
    language=HTML,
    tabsize=4,
    breaklines=true,
    breakatwhitespace=true,
    numbers=left,
    numbersep=10pt,
    numberstyle=\small\bfseries\ttfamily\color{htmlcomment},
    escapeinside={<@}{@>},
	rulecolor=\color{htmlbackground},
	xleftmargin=20pt,
	% Add a vertical line for opening and closing tags
    %frame=lines,
    framesep=2pt,
    framexleftmargin=4pt,
    % Add a vertical line for closing tags that go through multiple lines
    breaklines=true,
    postbreak=\mbox{\textcolor{gray}{$\hookrightarrow$}\space},
    showlines=true,
	% Add a rule to apply commentstyle to keywords inside comments
    moredelim=[s][\itshape\color{htmlcomment}]{<!--}{-->},
    morekeywords={id,class,type,name,value,placeholder,checked,src,href,alt}
}




%====================================================================
% 
%====================================================================






% Crée un environnement "Theorem" numéroté en fonction du document
\tcbuselibrary{theorems,skins,hooks} 
\newtcbtheorem{Theorem}{Théorème}
{%
	enhanced,
	breakable,
	colback = mytheorembg,
	frame hidden,
	boxrule = 0sp,
	borderline west = {2pt}{0pt}{mytheoremfr},
	sharp corners,
	detach title,
	before upper = \tcbtitle\par\smallskip,
	coltitle = mytheoremfr,
	fonttitle = \bfseries\fontfamily{lmss}\selectfont,
	description font = \mdseries\fontfamily{lmss}\selectfont,
	separator sign none,
	segmentation style={solid, mytheoremfr},
}
{thm}

% Crée un environnement "Preuve" numéroté en fonction du document
\tcbuselibrary{theorems,skins,hooks}
\newtcbtheorem{Preuve}{Preuve}
{
	enhanced,
	breakable,
	colback=white,
	frame hidden,
	boxrule = 0sp,
	borderline west = {2pt}{0pt}{mytheoremfr},
	sharp corners,
	detach title,
	before upper = \tcbtitle\par\smallskip,
	coltitle = mytheoremfr,
	description font=\fontfamily{lmss}\selectfont,
	fonttitle=\fontfamily{lmss}\selectfont\bfseries,
	separator sign none,
	segmentation style={solid, mytheoremfr},
}
{th}


% Crée un environnement "Preuve" numéroté en fonction du document
\tcbuselibrary{theorems,skins,hooks}
\newtcbtheorem{Explication}{Explication}
{
	enhanced,
	breakable,
	colback=white,
	frame hidden,
	boxrule = 0sp,
	borderline west = {2pt}{0pt}{mytheoremfr},
	sharp corners,
	detach title,
	before upper = \tcbtitle\par\smallskip,
	coltitle = mytheoremfr,
	description font=\fontfamily{lmss}\selectfont,
	fonttitle=\fontfamily{lmss}\selectfont\bfseries,
	separator sign none,
	segmentation style={solid, mytheoremfr},
}
{th}




% Crée un environnement "Example" numéroté en fonction du document
\tcbuselibrary{theorems,skins,hooks}
\newtcbtheorem{Example}{Exemple.}
{
	enhanced,
	breakable,
	colback=lightBlue,
	frame hidden,
	boxrule = 0sp,
	borderline west = {2pt}{0pt}{myb},
	sharp corners,
	detach title,
	before upper = \tcbtitle\par\smallskip,
	coltitle = myb,
	description font=\fontfamily{lmss}\selectfont,
	fonttitle=\fontfamily{lmss}\selectfont\bfseries,
	separator sign none,
	segmentation style={solid, mytheoremfr},
}
{th}



% Crée un environnement "EExample" numéroté en fonction du document
\tcbuselibrary{theorems,skins,hooks}
\newtcbtheorem{EExample}{Exemple}
{
	enhanced,
	breakable,
	colback=white,
	frame hidden,
	boxrule = 0sp,
	borderline west = {2pt}{0pt}{myb},
	sharp corners,
	detach title,
	before upper = \tcbtitle\par\smallskip,
	coltitle = myb,
	description font=\mdseries\fontfamily{lmss}\selectfont,
	fonttitle=\fontfamily{lmss}\selectfont\bfseries,
	separator sign none,
	segmentation style={solid, mytheoremfr},
}
{th}



% Crée un environnement "Lemme" numéroté en fonction du document
\tcbuselibrary{theorems,skins,hooks}
\newtcbtheorem{Lemme}{Lemme}
{
	enhanced,
	breakable,
	colback=mylenmabg,
	frame hidden,
	boxrule = 0sp,
	borderline west = {2pt}{0pt}{mylenmafr},
	sharp corners,
	detach title,
	before upper = \tcbtitle\par\smallskip,
	coltitle = mylenmafr,
	description font=\mdseries\fontfamily{lmss}\selectfont,
	fonttitle=\fontfamily{lmss}\selectfont\bfseries,
	separator sign none,
	segmentation style={solid, mytheoremfr},
}
{th}


\tcbuselibrary{theorems,skins,hooks}
\newtcbtheorem{PreuveL}{Preuve.}
{
	enhanced,
	breakable,
	colback=white,
	frame hidden,
	boxrule = 0sp,
	borderline west = {2pt}{0pt}{mylenmafr},
	sharp corners,
	detach title,
	before upper = \tcbtitle\par\smallskip,
	coltitle = mylenmafr,
	description font=\fontfamily{lmss}\selectfont,
	fonttitle=\fontfamily{lmss}\selectfont\bfseries,
	separator sign none,
	segmentation style={solid, mytheoremfr},
}
{th}


\newtcbtheorem{Remarque}{Remarque}
{
	enhanced,
	breakable,
	colback=white,
	frame hidden,
	boxrule = 0sp,
	borderline west = {2pt}{0pt}{myb},
	sharp corners,
	detach title,
	before upper = \tcbtitle\par\smallskip,
	coltitle = myb,
	description font=\mdseries\fontfamily{lmss}\selectfont,
	fonttitle=\fontfamily{lmss}\selectfont\bfseries,
	separator sign none,
	segmentation style={solid, mytheoremfr},
}
{th}


\newtcbtheorem{DefG}{Définition}
{
	enhanced,
	breakable,
	colback=mygbg,
	frame hidden,
	boxrule = 0sp,
	borderline west = {2pt}{0pt}{myg},
	sharp corners,
	detach title,
	before upper = \tcbtitle\par\smallskip,
	coltitle = myg,
	description font=\mdseries\fontfamily{lmss}\selectfont,
	fonttitle=\fontfamily{lmss}\selectfont\bfseries,
	separator sign none,
	segmentation style={solid, mytheoremfr},
}
{th}



% Crée une boîte ayant la même couleur que l'environnement theorem.
\tcbuselibrary{theorems,skins,hooks}
\newtcolorbox{Theoremcon}
{%
	enhanced
	,breakable
	,colback = mytheorembg
	,frame hidden
	,boxrule = 0sp
	,borderline west = {2pt}{0pt}{mytheoremfr}
	,sharp corners
	,description font = \mdseries
	,separator sign none
}

% Crée un environnement "Definition" numéroté en fonction de la section
\newtcbtheorem[number within=chapter]{Definition}{Définition}{enhanced,
	before skip=2mm,after skip=2mm, colback=red!5,colframe=red!80!black,boxrule=0.5mm,
	attach boxed title to top left={xshift=1cm,yshift*=1mm-\tcboxedtitleheight}, varwidth boxed title*=-3cm,
	boxed title style={frame code={
			\path[fill=tcbcolback!10!red]
			([yshift=-1mm,xshift=-1mm]frame.north west)
			arc[start angle=0,end angle=180,radius=1mm]
			([yshift=-1mm,xshift=1mm]frame.north east)
			arc[start angle=180,end angle=0,radius=1mm];
			\path[left color=tcbcolback!10!myred,right color=tcbcolback!10!myred,
			middle color=tcbcolback!60!myred]
			([xshift=-2mm]frame.north west) -- ([xshift=2mm]frame.north east)
			[rounded corners=1mm]-- ([xshift=1mm,yshift=-1mm]frame.north east)
			-- (frame.south east) -- (frame.south west)
			-- ([xshift=-1mm,yshift=-1mm]frame.north west)
			[sharp corners]-- cycle;
		},interior engine=empty,
	},
	fonttitle=\bfseries,
	title={#2},#1}{def}

% Crée un environnement "definition" numéroté en fonction du Chapitre
\newtcbtheorem[number within=section]{definition}{Définition}{enhanced,
	before skip=2mm,after skip=2mm, colback=red!5,colframe=red!80!black,boxrule=0.5mm,
	attach boxed title to top left={xshift=1cm,yshift*=1mm-\tcboxedtitleheight}, varwidth boxed title*=-3cm,
	boxed title style={frame code={
			\path[fill=tcbcolback]
			([yshift=-1mm,xshift=-1mm]frame.north west)
			arc[start angle=0,end angle=180,radius=1mm]
			([yshift=-1mm,xshift=1mm]frame.north east)
			arc[start angle=180,end angle=0,radius=1mm];
			\path[left color=tcbcolback!60!black,right color=tcbcolback!60!black,
			middle color=tcbcolback!80!black]
			([xshift=-2mm]frame.north west) -- ([xshift=2mm]frame.north east)
			[rounded corners=1mm]-- ([xshift=1mm,yshift=-1mm]frame.north east)
			-- (frame.south east) -- (frame.south west)
			-- ([xshift=-1mm,yshift=-1mm]frame.north west)
			[sharp corners]-- cycle;
		},interior engine=empty,
	},
	fonttitle=\bfseries,
	title={#2},#1}{def}

\usetikzlibrary{arrows,calc,shadows.blur}
\tcbuselibrary{skins}
\newtcolorbox{note}[1][]{%
	enhanced jigsaw,
	colback=gray!20!white,%
	colframe=gray!80!black,
	size=small,
	boxrule=1pt,
	title=\textbf{Note : },
	halign title=flush center,
	coltitle=black,
	breakable,
	drop shadow=black!50!white,
	attach boxed title to top left={xshift=1cm,yshift=-\tcboxedtitleheight/2,yshifttext=-\tcboxedtitleheight/2},
	minipage boxed title=1.5cm,
	boxed title style={%
		colback=white,
		size=fbox,
		boxrule=1pt,
		boxsep=2pt,
		underlay={%
			\coordinate (dotA) at ($(interior.west) + (-0.5pt,0)$);
			\coordinate (dotB) at ($(interior.east) + (0.5pt,0)$);
			\begin{scope}
				\clip (interior.north west) rectangle ([xshift=3ex]interior.east);
				\filldraw [white, blur shadow={shadow opacity=60, shadow yshift=-.75ex}, rounded corners=2pt] (interior.north west) rectangle (interior.south east);
			\end{scope}
			\begin{scope}[gray!80!black]
				\fill (dotA) circle (2pt);
				\fill (dotB) circle (2pt);
			\end{scope}
		},
	},
	#1,
}


% Crée un environnement "qstion" 
\newtcbtheorem{qstion}{Question}{enhanced,
	breakable,
	colback=white,
	colframe=mygr,
	attach boxed title to top left={yshift*=-\tcboxedtitleheight},
	fonttitle=\bfseries,
	title={#2},
	boxed title size=title,
	boxed title style={%
		sharp corners,
		rounded corners=northwest,
		colback=tcbcolframe,
		boxrule=0pt,
	},
}{def}


% Pour créer un environnement "Liste" 

\tcbuselibrary{theorems,skins,hooks}
\newtcbtheorem[number within=section]{Liste}{Liste}
{%
	enhanced
	,breakable
	,colback = myp!10
	,frame hidden
	,boxrule = 0sp
	,borderline west = {2pt}{0pt}{myp!85!black}
	,sharp corners
	,detach title
	,before upper = \tcbtitle\par\smallskip
	,coltitle = myp!85!black
	,fonttitle = \bfseries\sffamily
	,description font = \mdseries
	,separator sign none
	,segmentation style={solid, myp!85!black}
}
{th}


\tcbuselibrary{theorems,skins,hooks}
\newtcbtheorem{Syntaxe}{Syntaxe.}
{%
	enhanced
	,breakable
	,colback = myp!10
	,frame hidden
	,boxrule = 0sp
	,borderline west = {2pt}{0pt}{myp!85!black}
	,sharp corners
	,detach title
	,before upper = \tcbtitle\par\smallskip
	,coltitle = myp!85!black
	,fonttitle = \bfseries\fontfamily{lmss}\selectfont 
	,description font = \mdseries\fontfamily{lmss}\selectfont 
	,separator sign none
	,segmentation style={solid, myp!85!black}
}
{th}



% Crée un environnement "Concept" numéroté en fonction du document
\tcbuselibrary{theorems,skins,hooks}
\newtcbtheorem{Concept}{Concept.}
{
	enhanced,
	breakable,
	colback=mylenmabg,
	frame hidden,
	boxrule = 0sp,
	borderline west = {2pt}{0pt}{mylenmafr},
	sharp corners,
	detach title,
	before upper = \tcbtitle\par\smallskip,
	coltitle = mylenmafr,
	description font=\mdseries\fontfamily{lmss}\selectfont,
	fonttitle=\fontfamily{lmss}\selectfont\bfseries,
	separator sign none,
	segmentation style={solid, mytheoremfr},
}
{th}


% Crée un environnement "codeEx" numéroté en fonction du document
\tcbuselibrary{theorems,skins,hooks}
\newtcbtheorem{codeEx}{Exemple}
{
	enhanced,
	breakable,
	colback=white,
	frame hidden,
	boxrule = 0sp,
	borderline west = {2pt}{0pt}{gruvbox-bg},
	sharp corners,
	detach title,
	before upper = \tcbtitle\par\smallskip,
	coltitle = gruvbox-bg,
	description font=\md:wqseries\fontfamily{lmss}\selectfont,
	fonttitle=\fontfamily{lmss}\selectfont\bfseries,
	separator sign none,
	segmentation style={solid, mytheoremfr},
}
{th}


% Crée un environnement "codeEx" numéroté en fonction du document
\tcbuselibrary{theorems,skins,hooks}
\newtcbtheorem{codeRem}{Remarque.}
{
	enhanced,
	breakable,
	colback=white,
	frame hidden,
	boxrule = 0sp,
	borderline west = {2pt}{0pt}{gruvbox-bg},
	sharp corners,
	detach title,
	before upper = \tcbtitle\par\smallskip,
	coltitle = gruvbox-bg,
	description font=\mdseries\fontfamily{lmss}\selectfont,
	fonttitle=\fontfamily{lmss}\selectfont\bfseries,
	separator sign none,
	segmentation style={solid, mytheoremfr},
}
{th}


\tcbuselibrary{theorems,skins,hooks}
\newtcbtheorem{Identite}{Identité.}
{
	enhanced,
	breakable,
	colback=white,
  before upper=\tcbtitle\par\Hugeskip,
	frame hidden,
	boxrule = 0sp,
	borderline west = {2pt}{0pt}{gruvbox-bg},
	sharp corners,
	detach title,
	before upper = \tcbtitle\par\smallskip,
	coltitle = gruvbox-bg,
	description font=\mdseries\fontfamily{lmss}\selectfont,
	fonttitle=\fontfamily{lmss}\selectfont\bfseries,
	fontlower=\fontfamily{cmr}\selectfont,
  separator sign none,
	segmentation style={solid, mytheoremfr},
}
{th}

\tcbuselibrary{theorems,skins,hooks}
\newtcbtheorem{Exercice}{Exercice}
{
	enhanced,
	breakable,
	colback=white,
  before upper=\tcbtitle\par\Hugeskip,
	frame hidden,
	boxrule = 0sp,
	borderline west = {2pt}{0pt}{gruvbox-green},
	sharp corners,
	detach title,
	before upper = \tcbtitle\par\smallskip,
	coltitle = gruvbox-green,
	description font=\mdseries\fontfamily{lmss}\selectfont,
	fonttitle=\fontfamily{lmss}\selectfont\bfseries,
	fontlower=\fontfamily{cmr}\selectfont,
  separator sign none,
	segmentation style={solid, mytheoremfr},
}
{th}

% Crée un environnement "Réponse" numéroté en fonction du document
\tcbuselibrary{theorems,skins,hooks}
\newtcbtheorem{Reponse}{Reponse}
{
	enhanced,
	breakable,
	colback=white,
	frame hidden,
	boxrule = 0sp,
	borderline west = {2pt}{0pt}{mytheoremfr},
	sharp corners,
	detach title,
	before upper = \tcbtitle\par\smallskip,
	coltitle = mytheoremfr,
	description font=\fontfamily{lmss}\selectfont,
	fonttitle=\fontfamily{lmss}\selectfont\bfseries,
	separator sign none,
	segmentation style={solid, mytheoremfr},
}
{th}

\newtcbtheorem{Definitionx}{Définition}
{
enhanced,
breakable,
colback=red!5,
  before upper=\tcbtitle\par\Hugeskip,
frame hidden,
boxrule = 0sp,
borderline west = {2pt}{0pt}{red!80!black},
sharp corners,
detach title,
before upper = \tcbtitle\par\smallskip,
coltitle = red!80!black,
description font=\mdseries\fontfamily{lmss}\selectfont,
fonttitle=\fontfamily{lmss}\selectfont\bfseries,
fontlower=\fontfamily{cmr}\selectfont,
  separator sign none,
segmentation style={solid, mytheoremfr},
}
{th}

\tcbuselibrary{theorems,skins,hooks}
\newtcbtheorem[number within=chapter]{prop}{Proposition}
{%
	enhanced,
	breakable,
	colback = mypropbg,
	frame hidden,
	boxrule = 0sp,
	borderline west = {2pt}{0pt}{mypropfr},
	sharp corners,
	detach title,
	before upper = \tcbtitle\par\smallskip,
	coltitle = mypropfr,
	fonttitle = \bfseries\sffamily,
	description font = \mdseries,
	separator sign none,
	segmentation style={solid, mypropfr},
}
{th}


\tcbuselibrary{theorems,skins,hooks}
\newtcbtheorem[number within=section]{Prop}{Proposition}
{%
	enhanced,
	breakable,
	colback = mypropbg,
	frame hidden,
	boxrule = 0sp,
	borderline west = {2pt}{0pt}{mypropfr},
	sharp corners,
	detach title,
	before upper = \tcbtitle\par\smallskip,
	coltitle = mypropfr,
	fonttitle = \bfseries\sffamily,
	description font = \mdseries,
	separator sign none,
	segmentation style={solid, mypropfr},
}
{th}


%================================
% Corollery
%================================
\tcbuselibrary{theorems,skins,hooks}
\newtcbtheorem[number within=section]{Corollary}{Corollary}
{%
	enhanced
	,breakable
	,colback = myp!10
	,frame hidden
	,boxrule = 0sp
	,borderline west = {2pt}{0pt}{myp!85!black}
	,sharp corners
	,detach title
	,before upper = \tcbtitle\par\smallskip
	,coltitle = myp!85!black
	,fonttitle = \bfseries\sffamily
	,description font = \mdseries
	,separator sign none
	,segmentation style={solid, myp!85!black}
}
{th}
\tcbuselibrary{theorems,skins,hooks}
\newtcbtheorem[number within=chapter]{corollary}{Corollaire}
{%
	enhanced
	,breakable
	,colback = myp!10
	,frame hidden
	,boxrule = 0sp
	,borderline west = {2pt}{0pt}{myp!85!black}
	,sharp corners
	,detach title
	,before upper = \tcbtitle\par\smallskip
	,coltitle = myp!85!black
	,fonttitle = \bfseries\sffamily
	,description font = \mdseries
	,separator sign none
	,segmentation style={solid, myp!85!black}
}
{th}




\usepackage{amsmath,amsthm,mathtools}



\usepackage[scr]{rsfso}

%\usepackage{libertine}
%\usepackage{mathpazo}
%\usepackage{palatino}
%usepackage{crimson}


\title{\Huge{Calcul 1}\\{MATH1400}\\{\textbf{Introduction}}}
\author{\huge{Franz Girardin}}
\date{\today}
\lstset{inputencoding=utf8/latin1}

            %%%%%%%%%%%%%%%%%  Sect.                          %%%%%%%%%%%%%%%%%%%%%%%%%%%%%%%%%%%%%%%%%%%%%%%%%%%%%%%%%
\usepackage{helvet}
\titleformat{\chapter}
  {\fontfamily{phv}\bfseries\huge} % format
  {}                % label
  {0pt}             % sep
  {\color{myb}\huge}           % before-code



\titleformat{\section}
  {\normalfont\scshape}{\thesection}{1em}{}


% Customizing the spacing for the chapter titles
\titlespacing*{\chapter}{0pt}{0pt}{20pt}

% Allow hfill in math environment
\newcommand{\specialcell}[1]{\ifmeasuring@#1\else\omit$\displaystyle#1$\ignorespaces\fi}

% Allow you to do the non implication (implication barred)
\newcommand{\notimplies}{%
  \mathrel{{\ooalign{\hidewidth$\not\phantom{=}$\hidewidth\cr$\implies$}}}}



\DeclareRobustCommand{\looongrightarrow}{%
  \DOTSB\relbar\joinrel\relbar\joinrel\relbar\joinrel\rightarrow
}

\begin{document}
\maketitle

\pagebreak

\pagebreak
\begin{multicols*}{3}


    \footnotesize

    \paragraph{Définition d'une suite}
        \textbf{Fonction} $\mathbb{N}^* 
        \rightarrow \mathbb{R}$ qui accepte 
        $n \in \mathbb{N}^*$ et engendre une \textbf{séquence ordonnées} de $a_n \in \mathbb{R}$. 


    \paragraph{Définition d'une suite arithmétique}    
        \begin{align*}
                &a_n \Coloneqq 
                \begin{cases}
                    a_1 = r  &\omit\quad\quad\quad Raison\\  
                    a_{n} = a_{n-1} + r &\omit\hfill \quad\quad\quad Récurrence
                \end{cases}
                \\\\
                &r = a_n - a_{n-1} \;\;| n \geq 2  
                \quad\quad\quad\quad\;\;\text{Trouver $r$} \\
                &a_n = a_1 \text{+} \left(n - 1\right)\cdot{n}  \;\;|n \geq 1  
                \quad\quad\text{Trouver $n$\up{e} terme}
        \end{align*}
    \paragraph{Définition d'une suite géométrique}
        \begin{align*}
                &a_n \Coloneqq 
                \begin{cases}
                    a_1 = r  &\omit\quad\quad Raison\\  
                    a_{n} = a_{n-1} \cdot r &\omit\hfill 
                    \quad\quad Récurrence
                \end{cases}
                \\\\ 
                &r = \frac{a_n}{a_{n-1}} \; | \; n \geq 2
                \quad\quad\quad\quad\;\;\text{Trouver $r$} \\ 
                &a_n = a_1r^{n - 1} \; | \; n \geq 1
                \text{\quad\quad\quad\; Trouver $n$\up{e} terme}
        \end{align*}






        \paragraph{Convergence d'une suite géométrique}
        $\forall r \in \mathbb{R}$, la suite $\{r^n\}$ converge        
        \textit{ssi} $-1 < r \leq 1$ : 
        \begin{align*}
            \lim\limits_{n\to+\infty }\{r^n\} = 
                    \begin{cases}
                        0 \quad \text{si $-1 < r < 1$} \\
                        1 \quad \text{si $r = 1$}
                    \end{cases}
        \end{align*}


        

    \paragraph{Définition formelle de convergence d'une suite}
      \begin{align*}
        \lim\limits_{n \to\infty  }{a_n} = L \\
      \end{align*}
      \textbf{si et seulement si}, 
      \begin{align*}
        \forall\varepsilon > 0, \exists N\left( \varepsilon \right) > 0 : 
        n > N\left( \varepsilon \right) \implies 
        |a_n -L| < \varepsilon
      \end{align*}



    \paragraph{Définition formelle de divergence d'une suite}
      \begin{align*}
        \lim\limits_{n  \to\infty  }{a_n} = \infty
      \end{align*}
      \textbf{si et seulement si},  
        \begin{align*}
            \forall M \in \mathbb{R}, \exists N \in \mathbb{N}^{*} :
            n > N \implies |a_n| > M 
        \end{align*}

    \paragraph{Corollaire}
    \textbf{Si} $\lim\limits_{n\to+\infty }a_n  = \infty$,  
    \textbf{alors}, 
    $\lim\limits_{n\to+\infty }{\dfrac{1}{a_n}}  = 0$




    \paragraph{Attention}
        \begin{align*}
            \lim\limits_{n\to\infty  }\frac{1}{a_n} = 0 
            \textcolor{red}{\notimplies}
            \lim\limits_{n\to+\infty }a_n  = \infty
        \end{align*}    

    \paragraph{Lemme de convergence des suite éventuellement signées}



        \begin{enumerate}
            \item 
                Si $\{ a_n \}$ est une suite 
                \textbf{éventuellement positive}, 
                alors,  
                    $\lim\limits_{n\to+\infty }\frac{1}{a_n}  = 0 
                    \implies 
                    \lim\limits_{n\to+\infty }a_n  = \infty$
            \item                                 
                Si $\{ a_n \}$ est une suite 
                \textbf{éventuellement négative}, 
                alors,   
                    $\lim\limits_{n\to+\infty }\frac{1}{a_n}  = 0 
                    \implies 
                    \lim\limits_{n\to+\infty }a_n  = -\infty$
        \end{enumerate} 


 

    \paragraph{Propriétés des limites}
        Si $\{a_n\}$ et $\{b_n\}$ sont des suites convergentes et 
        si $c$ est une constante, \textbf{alors} \\\\ 
        $\lim\limits_{n\to\infty  }\left(a_n \text{+} b_n \right) = 
        \lim\limits_{n\to\infty  }a_n \text{+} 
        \lim\limits_{n\to\infty  }b_n$
        \\\\
        $\lim\limits_{n\to\infty  }\left(a_n - b_n \right) = 
        \lim\limits_{n\to\infty  }a_n - \lim\limits_{n\to\infty  }b_n$ 
        \\\\
        $\lim\limits_{n\to\infty  }ca_n = c \lim\limits_{n \to \infty  }a_n$ 
        \\\\
        $\lim\limits_{n\to\infty  }\left(a_nb_n \right) = 
        \lim\limits_{n\to\infty  }a_n \cdot \lim\limits_{n\to\infty  }b_n$
        \\\\
        $\lim\limits_{n\to\infty  }\left(\frac{a_n}{b_n} \right) = 
        \frac{\lim\limits_{n\to\infty  }a_n}{\lim\limits_{n\to\infty  }b_n}
        \;
        \text{si} \lim\limits_{x\to\infty  }b_n \neq 0$
        \\\\
        $\lim\limits_{n\to\infty  }a_n^{p} = 
        \left[\lim\limits_{n\to\infty  }a_n \right]^p \text{si } 
        p > 0 \; \text{et} \; a_n > 0$



    \paragraph{Limite d'une suite polynomiale} 
        Soit deux polynomes,
        $\lim\limits_{n\to \infty } \dfrac{p(n)}{q(n)}$, 
        et 
        $k = \min\bigl(deg(p), deg(q)\bigr)$
        \textbf{Alors},   
        \[ \lim\limits_{n\to \infty } \dfrac{p(n)}{q(n)} =
        \lim\limits_{n\to+\infty}\dfrac{p(n)/{n^k}}{q(n)/n^{k}} \]



    \paragraph{Règle de l'Hôpital}
        Soit une \textbf{constante} $c \in \mathbb{R} \cup \{+\infty\}$ et supposon que : 
        \begin{itemize}
        \item $\lim\limits_{x\to c}\dfrac{|f(x)|}{|g(x)|}$ 
            est de la forme $\dfrac{0}{0}$ ou 
            $\dfrac{\infty }{\infty }$
        \item $\lim\limits_{x\to c}\dfrac{f^{\prime}(x)}{|g^{\prime}(x)|}$
            \textbf{existe} et 
            \textcolor{red}{$g^{\prime}(x) \neq 0 \;\; \forall x \approx c$ }
        \end{itemize}
        \textbf{Alors}, 
        \begin{align*}
            \lim\limits_{x\to c}\dfrac{f(x)}{g(x)} = 
            \lim\limits_{x\to c}\dfrac{f^{\prime}(x)}{g^{\prime}(x)}
        \end{align*}
        


    \paragraph{Monotonicité} 
        Soit une suite $\{a_n \}$, on dit que la suite est :
        \begin{itemize}
            \item \textbf{Strictement croissant} si $\forall n \geq 1, 
                a_{n+1} > a_n$
            \item \textbf{Croissante} si $\forall n \geq 1, 
                a_{n+1} \geq a_n$ 
            \item \textbf{Strictement décroissante} si $\forall n \geq 1, 
                a_{n+1} < a_n$ 
            \item \textbf{Décroissante} si $\forall n \geq 1, 
                a_{n+1} < a_n$ 
            \item \textbf{Stationnaire} ou \textbf{constante} si 
                $\forall n \geq 1, 
                a_{n+1} < a_n$ 
            \item \textbf{Monotone}  
        \end{itemize} 



    \paragraph{Définitions de bornes d'une suite}
       \begin{align*}
           \textbf{Minonant } m \Coloneqq 
           \exists m \in \mathbb{R} : \forall n \in \{ a_n \}, 
           a_n \geq m
       \end{align*}
       \begin{align*}
           \textbf{Majorant} M \Coloneqq 
           \exists M \in \mathbb{R} : \forall n \in \{ a_n \}, 
           a_n \leq M
       \end{align*}
       \begin{align*}
           \textbf{Bornée} \Coloneqq 
            \textit{Majorée} \land \textit{minorée}
       \end{align*}
       


    \paragraph{Théorème des suites monotones}
        Toute suite monotone et bornée est \textbf{convergente}  


    \paragraph{Lemmes des suites monotones}
        \begin{itemize}
            \item Toute suite éventuellement croissante et majorée 
        est également \textbf{convergente}  
            \item Tout suite éventuellement décroissante et 
            minorée est également \textbf{convergente}  
        \end{itemize}
    


    \paragraph{Association d'une fonction à une suite}
        Soit $f\left(x\right)$ une fonction admettant une limite $L$ à 
        $\text{+}\infty$, Alors, la suite 
        $\{a_n\} = f\left(n\right)$ admet la même limite : 
        \begin{align*}
          \lim\limits_{x\to\infty  }f(x) = L \implies \lim\limits_{n\to\infty  }a_n = L
        \end{align*}
        De la même façon :
        \begin{align*}
          \lim\limits_{x\to\infty  }f(x) = \infty \implies \lim\limits_{n\to\infty  }a_n = \infty
        \end{align*}
        Par ailleurs, si $f\left(x\right)$ est une fonction continue en 
        $L$ et si la suite $\{a_n\}$
        converge vers $L$, alors la limite suivante converge vers $f\left(L\right)$ :
        \begin{align*}
                    \lim\limits_{n\to+\infty  }f(a_n)  = f(L) \\ 
                    \lim\limits_{n\to+\infty  }f(\lim\limits_{n \to \infty} a_n)  
                    = f(L)  
        \end{align*}
        $\rhd$ $ \textbf{Exemple}  \lim\limits_{n \to+\infty }\sin(\pi/2)  = 
        \sin(\lim\limits_{n \to+\infty })  \pi/2  = 0$




    \paragraph{Comparaison des suites}
        Si $a > 1$ et $k > 0$, on a 
        \begin{align*}
            \ln(n) \ll n^K \ll a^n \ll n! \ll n^n
        \end{align*}
        $c_n \ll d_n \implies 
        \lim\limits_{n\to+\infty }\dfrac{c_n}{d_n} = 0$ 


    \paragraph{Théorème des gendarmes}{}
        Soient $\{a_n\}$, $\{b_n\}$ et $\{c_n\}$ des suites et $n_0 \in \mathbb{N}$ tels 
        que
        \begin{itemize}
            \item $\lim\limits_{n\to+\infty }a_n  = 
                \lim\limits_{n\to+\infty }c_n  = L 
                \in \mathbb{R} \cup \{\infty \}$; 
            \item $\forall n \geq n_0, \; a_n \leq b_n \leq c_n$ 
        \end{itemize}
        \textbf{Alors},
        \begin{align*}
            \lim\limits_{n\to+\infty}b_n  = L                   
        \end{align*}



    \paragraph{Corollaire}
         \textbf{Si} $\lim\limits_{n\to+\infty }|a_n|  = 0$, 
         \textbf{alors}      
         $\lim\limits_{n\to+\infty }a_n  = 0$

    \paragraph{Définition d'une série numérique}
    Somme infinie des termes d'une suite numérique 
    correspondante $a_n$ : 
    $\sum_{n=1}^{\infty}a_n$. 
    \begin{itemize}
        \item \textbf{Premiers termes} $s_n = \sum_{i=1}^{n } a_i$   
        \item \textbf{Convergence}   
        $\lim\limits_{n\to+\infty }s_n  = s \implies 
        \sum_{n}^{\infty}a_n \textbf{ conv.}$
    \end{itemize}



    \paragraph{Critère de divergence}
        Si la série converge, la suite correspondante 
        \textbf{converge vers 0},
        et si la suite ne converge pas vers zéro, la série 
            est divergente
        \begin{itemize} 
            \item $\sum_ {n}^{ \infty }a_n = s \; (\textbf{conv.}) 
                \implies 
            \lim\limits_{n\to+\infty }a_n  = 0$
            \item
            $\lim\limits_{n\to+\infty }a_n  \neq 0  
            \implies 
            \sum_{ n}^{\infty }a_n$ \textbf{ div.}  
        \end{itemize}

    \paragraph{Attention} 
    $\lim\limits_{n\to+\infty }a_n = 0 
    \textcolor{red}{\notimplies} 
    \sum_{n=1}^{\infty }a_n$ \textbf{conv}.   



    \paragraph{Convergence d'une série géométrique}
    \begin{itemize}
        \item[$\rhd$] 
            $|r| < 1 \implies  
            \sum_{n=0}^{\infty }ar^{n} = 
            \dfrac{a}{1 - r} $ \;\;(\textbf{conv.})  
        \item [$\rhd$] 
            $(|r| \geq 1) \land (a \neq 0) \implies 
            \sum_{n=0}^{\infty }ar^{n} = \infty$ 
            (\textbf{div.})
        
    \end{itemize}



    \paragraph{Propriétés des séries}
    \begin{itemize}
        \item[$\blacktriangleright$]  \textbf{+} et \textbf{-} de deux séries convergentes 
            ainsi que $c \cdot \sum_{n}^{\infty }a_n$ engendrent une série \textbf{conv}.
        \item[$\blacktriangleright$] 
            $\sum_{n=1}^{\infty}a_nb_n \textcolor{red}{\neq} 
            (\sum_{n=1}^{\infty}a_n)(\sum_{n=1}^{\infty}b_n)$
        \item[$\blacktriangleright$]
            $\sum_{n=1}^{\infty }\dfrac{a_n}{b_n} 
            \textcolor{red}{\neq} 
            \dfrac{\sum_{n=1}^{\infty}a_n}{\sum_{n=1}^{\infty}b_n}$
    \end{itemize}
    
    
    \paragraph{Test de l'intégrale}
    Soit $f : [1, \infty [ \rightarrow \mathbb{R}$ 
    \textbf{continue}, \textbf{positive} et \textbf{décroissante} 
    et $a_n : f(n) = a_n$,
    \begin{align*}
        \sum_{n=1}^{\infty }a_n \;\; \textbf{conv.} \;\; \leftrightarrow
        \lim\limits_{a \to+\infty }\int_{x=1}^{x = a}f(x)dx = s \;\; (\textbf{conv.})    
    \end{align*}


    \paragraph{Série de Riemann et série puissance} 
    \begin{align*}
        \sum_{n=1}^{\infty }\dfrac{1}{n^p} 
        \text{ converge si } p > 1 \text{, diverge si } p \leq 1
        \\
        \sum_{n=1}^{\infty}n^p \text{ converge  si }
         p > 1 
    \end{align*}
    La première est un \textbf{série de Riemann}; la seconde 
    est une \textbf{série puissance}.  


    
    \paragraph{Estimation du reste par TI}
    \textbf{Si} $f: \lambda, + , \downarrow [m, \infty [$ et soit $m \in \mathbb{N}^*, 
     a_n = f(n), \sum_{n=1}^{\infty}a_n = s \in \mathbb{R}, 
     R_m = s - s_m$, \textbf{alors} le reste $R_m$ est borné et peut être 
     estimé :
     \begin{align*}
        \int_{m+1}^{\infty }f(x)dx \leq \left(R_m = \sum_{k=1}^{m} a_k\right) 
        \leq \int_{m}^{\infty }f(x)dx
     \end{align*}


    
     \paragraph{Test de comparaison}
     Soient $\sum a_n, \sum b_n$ des séries à \textbf{termes positifs} 
     et $n_0 \in \mathbb{R}$ :
     \begin{itemize}
         \item[$\rhd$ ] $\left(\sum   b_n  \;\; \textbf{conv.}\right) \land 
             (a_n \leq b_n \forall n \geq n_0) 
             \implies a_n \; \textbf{conv.}$ 
         \item[$\rhd$ ] $\left(\sum   b_n  \;\; \textbf{div.}\right) \land 
             (a_n \geq b_n \forall n \geq n_0) 
             \implies a_n \;\; \textbf{div.}$ 
         \item[$\rhd$ ] $\lim\limits_{n\to\infty}\dfrac{a_n}{b_n}
             \in \mathbb{R} \textbf{ existe } \textbf{ et } L > 0 
             \implies 
        \sum a_n \;\; \textbf{conv.} \;\; \leftrightarrow \;\; 
        \sum b_n \;\; \textbf{conv.}$ 
    \item[$\blacktriangleright$] Principalement pour Riemann \& géométriques
     \end{itemize}

     

     \paragraph{Test sur séries alternées}
     Soit un \textbf{rang} $n_0 \in \mathbb{N}$ et 
     soit une \textbf{série alternée} 
     $\sum_{n=1}^{\infty } (-1)^nb_n$ telle que 
     \begin{itemize}
       \item [$\rhd$ ]  $0 \leq b_{n+1} \leq b_n \;\;(\downarrow \textbf{ et } +)$ 
       \item [$\rhd$ ] $\lim\limits_{n\to\infty }b_n = 0$ 
     \end{itemize}
     \textbf{Alors}, 
     \begin{itemize}
         \item[$\blacktriangleright$]
     $\sum  (-1)^nb_n \textbf{ conv. }$ vers $s \in
     \mathbb{R} \;\; \forall 
     m \geq n_0 $ \textbf{et}  
        \item[$\blacktriangleright$] 
    $|s - s_m| \leq b_{m+1}$
     \end{itemize}

  \paragraph{Définition de convergence absolue}
     \mbox{}\\
     \[ \sum_{n=1}^{\infty }|a_n| \textbf{conv.} \implies \sum_{n=1}^{\infty }a_n  
     \textbf{ conv. } \textit{absoluement} \]
      \\
      $\rhd$  \textbf{Semi-conv.} : $\sum_{n=1}^{\infty }a_n$ conv. 
         \texttt{\&\&} $\neg(\sum_{n=1}^{\infty }|a_n|)$ 
      $\blacktriangleright$  \textbf{Exemple} $\sum_{n=1}^{\infty }\frac{(-1)^n}{n}$ conv 
          \textbf{mais} $\neg(\sum_{n=1}^{\infty }\frac{1}{n})$ conv.  

  \paragraph{Test du rapport (d'Alembert)}
  \mbox{}\vspace{0.2em}
  Soit $\lim\limits_{n \to+\infty } \Big|\frac{a_{n+1}}{a_n}  \Big| = L$ 
  \textbf{alors} si :   
  \begin{itemize}
    \item [$\rhd$ ] $L = 1 \implies$ \textit{ inconclusif}  
    \item [$\blacktriangleright$ ] $L > 1 \implies \sum_{n=1}^{\infty } a_n$ 
      \textbf{div}.   
    \item [$\blacktriangleright$ ] $L < 1 \implies \sum_{n=1}^{\infty } a_n$ \textbf{conv}.   
  \end{itemize}



  \paragraph{Test de Cauchy}
  \mbox{}\vspace{0.2em}
  Soit $\lim\limits_{n \to+\infty } \sqrt[n]{\big| a_n \big|} = L$ 
  \textbf{alors} si :   
  \begin{itemize}
    \item [$\rhd$ ] $L = 1 \implies$ \textit{inconclusif}  
    \item [$\blacktriangleright$ ] $L > 1 \implies \sum_{n=1}^{\infty }a_n$ 
      \textbf{div}.   
    \item [$\blacktriangleright$ ] $L < 1 \implies \sum_{n=1}^{\infty }a_n$ 
      \textbf{conv}.
  \end{itemize}

  \paragraph{Défininition d'une série entière} Soit une variable 
  $x$, les contantes $c_0, c_1, \cdots , c_n$ 
  et $n \in \mathbb{N}$, une série 
  est dite centrée en $a \in \mathbb{R}$ si on a :  
  \[ \sum_{n=0}^{\infty } c_n(x-a)^n = c_0 + c_1(x-a) + c_2(x-a)^2 + \cdots \]

  \paragraph{Famille de séries paramétrées par $x$} 
  Soit l'ensemble $D$ des $x$ pour lesquels la série converge, 
  la fonction $f : D \rightarrow  \mathbb{R}$ engendre une 
  somme $f(x) \in \mathbb{D}$ :
  \[ f(x) = \sum_{n=0}^{\infty } c_n(x - a)^n \]
  \vspace{-1.5em} %vspace

  \begin{table}[H]
    \begin{center}
      \renewcommand{\arraystretch}{2}
      \fontfamily{flr}\selectfont
      \footnotesize
      \begin{tabular}{l|p{3.75cm}}
          \arrayrulecolor{blue}\hline
          \rowcolor{lightBlue}
          \textcolor{myb}{Fonction} & \textcolor{myb}{Somme}
          \\
          \hline
          \arrayrulecolor{black}
          $\sin x$ & $x - \frac{x^3}{3!} +  \frac{x^5}{5!} 
          -  \frac{x^7}{7!} + \dots $ 
          \\
          \hline
          $\cos x$ & $1 - \frac{x^2}{2!} +  \frac{x^4}{4!} 
          -  \frac{x^6}{6!} + \dots $ 
          \\
          \hline
          $\dfrac{1}{1-x}$ & $\forall \; |x| <1 \textcolor{red}{\colon}  
          1 + x + x^2 + x^3 + \dots $  
          \\
          \hline 
          $e^x$ & $1 + x + \frac{x^2}{2!} + \frac{x^3}{3!} \cdots$ 
          \\
          \hline
          $\ln(1 + x)$ & $\forall \; |x| < 1 \textcolor{red}{\colon} 
          x - \frac{x^2}{2} +  \frac{x^3}{3} 
          -  \frac{x^4}{4} + \dots $ 
          \\
          \hline
          $\arctan (x)$ & $\forall \; |x| < 1 \textcolor{red}{\colon}  
          x - \frac{x^3}{3} +  \frac{x^5}{5!} 
          -  \frac{x^7}{7!} + \dots $ 
          \\
          \hline
          $(1 + x)^k$ &           
              $\forall \; |x| < 1 \textcolor{red}{\colon}  
              1+ kx + \frac{k(k-1)}{2!}x^2 
                    + \frac{k(k-1)(k-2)}{3!}x^3 
              \cdots$
          \\
          \hline

          \end{tabular}
  \end{center}
  \end{table}

  \paragraph{Rayon de convergence}
  Soit la série $S = \sum_{n=0}^{\infty }c_n(x - a)^n$, il a 
  \textit{trois possibilités} :   
  \begin{itemize}
    \item [$\rhd$ ] $S$ \textbf{conv.} \textcolor{red}{à}   
           $x = a \implies  R = 0$  
    \item [$\rhd$ ] $\forall \; x, \; S \textbf{ conv.} \implies  
                 R = \infty$       
    \item [$\rhd$ ] $\exists R > 0 \colon $ 
        \begin{itemize}
          \item [$\blacktriangleright$ ] $|x - a| < R \implies 
            S \textbf{ conv.}  $ 
          \item [$\blacktriangleright$ ] $|x - a| > R \implies 
            S \textbf{ div.}$ 
        \end{itemize}
  \end{itemize}












     











 \end{multicols*}
\end{document}
