\documentclass{report}
%\usepackage[utopia]{mathdesign} 

\usepackage{amsmath,amsfonts,amsthm,amssymb,mathtools}
%\usepackage{amsmath,amsthm,mathtools}

\usepackage[french]{babel}

% Permet d'ajuster la taille des marges et de la distance pour les footer
\usepackage[tmargin=2cm,rmargin=0.4in,lmargin=0.4in,bmargin=2cm,footskip=.2in]{geometry}

% Permet d'optimiser l'affichage de différents symboles et formules mathématiques
\usepackage{amsmath,amsfonts,amsthm,amssymb,mathtools}

\usepackage{svg}
% Modifie l'apparence des nombre en mathmode et textmode
%\usepackage[varbb]{newpxmath}

% Modifier l'apparence des fractions
\usepackage{xfrac}

            %%%%%%%%%%%%%%%%%  Sect.        14 Oct 2024     %%%%%%%%%%%%%%%%%%%%%%%%%%%%%%%%%%%%%%%%%%%%%%%%%%%%%%%%%%%
\usepackage{graphicx}
\usepackage{caption}
\usepackage{subcaption}
\usepackage{arydshln}
            %%%%%%%%%%%%%%%%%  Sect.        14 Oct 2024     %%%%%%%%%%%%%%%%%%%%%%%%%%%%%%%%%%%%%%%%%%%%%%%%%%%%%%%%%%%
\usepackage{balance}
\usepackage{dirtree}
\usepackage{titlesec}






% Permet de rayer (barrer) l'argument avec la touche
% \cancel{} \bcancel{} ou \xcancel{}
\usepackage[makeroom]{cancel}

% Extension du package amsmath; corrige certains bugs et déficiences de son prédecesseur
\usepackage{mathtools}

% This package provides most of the flexibility you may want to customize the three basic list
% environments (enumerate, itemize and description)
\usepackage{bookmark} 

% Réorganiser les théorèmes et Lemmes. Usage complexe. 
% Référence : https://ctan.math.illinois.edu/macros/latex/contrib/theoremref/theoremref-doc.pdf
\hypersetup{hidelinks}
\usepackage{hyperref,theoremref} 

% Fournit un environnement pour créer des boîtes colorées
\usepackage[most,many,breakable]{tcolorbox}


%\newcommand\mycommfont[1]{\footnotesize\ttfamily\textcolor{blue}{#1}}\SetCommentSty{mycommfont}

%\newcommand{\incfig}[1]{%\def\svgwidth{\columnwidth}\import{./figures/}{#1.pdf_tex}}
\newcommand{\arc}[1]{\wideparen{#1}}

%Pour colorer les lignes séparatrices de tableaux
\usepackage{colortbl}
\usepackage{tikzsymbols}

\usepackage{framed}
\usepackage{titletoc}
\usepackage{etoolbox}
\usepackage{lmodern}
\usepackage{tabularx}
\usepackage{enumitem}
\usepackage{amsthm}
            %%%%%%%%%%%%%%%%%  Sect.        14 Oct 2024     %%%%%%%%%%%%%%%%%%%%%%%%%%%%%%%%%%%%%%%%%%%%%%%%%%%%%%%%%%%

\usepackage{lipsum}
\usepackage{titling}
\renewcommand\maketitlehooka{\null\mbox{}\vfill}
\renewcommand\maketitlehookd{\vfill\null}

\newcommand{\varitem}[3][black]{%
  \item[%
   \colorbox{#2}{\textcolor{#1}{\makebox(5.5,7){#3}}}%
  ]
}
\usepackage{afterpage}
\newcommand\myemptypage{
    \null
    \thispagestyle{empty}
    \addtocounter{page}{-1}
    \newpage
    }




% from https://tex.stackexchange.com/a/167024/121799
\newcommand{\ClaudioList}{red,DarkOrange1,Goldenrod1,Green3,blue!50!cyan,DarkOrchid2}
\newcommand{\SebastianoItem}[1]{\foreach \X[count=\Y] in \ClaudioList
{\ifnum\Y=#1\relax
\xdef\SebastianoColor{\X}
\fi
}
\tikz[baseline=(SebastianoItem.base),remember
picture]{%
\node[fill=\SebastianoColor,inner sep=4pt,font=\sffamily,fill opacity=0.5] (SebastianoItem){#1)};}
}
\newcommand{\SebastianoHighlight}{\tikz[overlay,remember picture]{%
\fill[\SebastianoColor,fill opacity=0.5] ([yshift=4pt,xshift=-\pgflinewidth]SebastianoItem.east) -- ++(4pt,-4pt)
-- ++(-4pt,-4pt) -- cycle;
}}   
            %%%%%%%%%%%%%%%%%  Sect.        14 Oct 2024     %%%%%%%%%%%%%%%%%%%%%%%%%%%%%%%%%%%%%%%%%%%%%%%%%%%%%%%%%%%





%====================================================================

%====================================================================
\newcommand*{\authorimg}[1]%
    { \raisebox{-1\baselineskip}{\includegraphics[width=\imagesize]{#1}}}
\newlength\imagesize  

\usepackage{pgfplots}
\pgfplotsset{compat=1.17}

%==========================================================================================
\usepackage{libris} 
\usepackage{etoolbox}
\usepackage[export]{adjustbox}% for positioning figures

\makeatletter
% Force le chapitre suivant sur la ligne succedant la fin du 
% chapitre précédent
\patchcmd{\chapter}{\if@openright\cleardoublepage\else\clearpage\fi}{}{}{}
\makeatother
\usepackage[Sonny]{fncychap}


%boîte de couleur grise
\tcbset{
  graybox/.style={
    colback=gray!20,
    colframe=black,
    sharp corners=downhill,
    boxrule=1pt,
    left=5pt,
    right=5pt,
    top=5pt,
    bottom=5pt,
    boxsep=0pt,
	 % <-- add four values for each corner
  }
}
\newtcolorbox{graybox}{graybox}

%==========================================================================================



\usepackage{xcolor}
\usepackage{varwidth}
\usepackage{varwidth}
\usepackage{etoolbox}
%\usepackage{authblk}
\usepackage{nameref}
\usepackage{multicol,array}
\usepackage{tikz-cd}
\usepackage[ruled,linesnumbered,ruled]{algorithm2e}
\usepackage{comment} % enables the use of multi-line comments (\ifx \fi) 
\usepackage{import}
\usepackage{xifthen}
\usepackage{pdfpages}
\usepackage{transparent}


%\usepackage[french]{babel}
\usepackage{listings} % pour écrire du code dans un environnement
\lstset{
  basicstyle=\ttfamily,
  columns=fullflexible,
  keepspaces=true
}
\usepackage{caption}
\usepackage{float} % Pour forcer les images au bon endroit



\usepackage[T1]{fontenc}
\usepackage{csquotes}
%%%%%%%%%%%%%%%%%%%%%%%%%%%%%%%%%%%%%%%%%%%%%%%%%%%%%%%%%%%%%%%%%%%%%%%%%%%%%%%%%%%%%%%%%%%%%%%%%
%									ENSEMBLE DE COULEURS
%%%%%%%%%%%%%%%%%%%%%%%%%%%%%%%%%%%%%%%%%%%%%%%%%%%%%%%%%%%%%%%%%%%%%%%%%%%%%%%%%%%%%%%%%%%%%%%%%

\definecolor{myg}{RGB}{56, 140, 70}
\definecolor{myb}{RGB}{45, 111, 177}

\definecolor{mygbg}{RGB}{235, 253, 241}


\definecolor{myr}{RGB}{199, 68, 64}
\definecolor{mytheorembg}{HTML}{F2F2F9}
\definecolor{mytheoremfr}{HTML}{00007B}
\definecolor{mylenmabg}{HTML}{FFFAF8}
\definecolor{mylenmafr}{HTML}{983b0f}
\definecolor{mypropbg}{HTML}{f2fbfc}
\definecolor{mypropfr}{HTML}{191971}
\definecolor{myexamplebg}{HTML}{F2FBF8}
\definecolor{myexamplefr}{HTML}{88D6D1}
\definecolor{myexampleti}{HTML}{2A7F7F}
\definecolor{mydefinitbg}{HTML}{E5E5FF}
\definecolor{mydefinitfr}{HTML}{3F3FA3}
\definecolor{notesgreen}{RGB}{0,162,0}
\definecolor{myp}{RGB}{197, 92, 212}
\definecolor{mygr}{HTML}{2C3338}
\definecolor{myred}{RGB}{127,0,0}
\definecolor{myyellow}{RGB}{169,121,69}
\definecolor{myexercisebg}{HTML}{F2FBF8}
\definecolor{myexercisefg}{HTML}{88D6D1}
\definecolor{myred}{RGB}{127,0,0}
\definecolor{myyellow}{RGB}{169,121,69}
\definecolor{LightLavender}{HTML}{DFC5FE}

\definecolor{blue}{HTML}{008ED7}
\definecolor{mygray}{gray}{0.75}
\definecolor{lightBlue}{RGB}{235, 245, 255}
\definecolor{tcbcolred}{RGB}{255,0,0}
\definecolor{myGreen}{HTML}{009900}

% command to circle a text
\newtcbox{\entoure}[1][red]{on line,
	arc=3pt,colback=#1!10!white,colframe=#1!50!black,
	before upper={\rule[-3pt]{0pt}{10pt}},boxrule=1pt,
	boxsep=0pt,left=2pt,right=2pt,top=1pt,bottom=.5pt}
% command for the circle for the number of part entries
\newcommand\Circle[1]{\tikz[overlay,remember picture]
	\node[draw,circle, text width=18pt,line width=1pt] {#1};}

\newtcbox{\entouree}[1][red]{on line,
	arc=3pt,colback=#1!10!white,colframe=#1!50!white,
	before upper={\rule[-3pt]{0pt}{10pt}},boxrule=1pt,
	boxsep=0pt,left=2pt,right=2pt,top=1pt,bottom=.5pt}

\newcommand{\shellcmd}[1]{\\\indent\indent\texttt{\footnotesize\# #1}\\}

%=====================================================================

\patchcmd{\tableofcontents}{\contentsname}{\rmfamily\contentsname}{}{}
% patching of \@part to typeset the part number inside a framed box in its own line
% and to add color
\makeatletter
\patchcmd{\@part}
  {\addcontentsline{toc}{part}{\thepart\hspace{1em}#1}}
  {\addtocontents{toc}{\protect\addvspace{20pt}}
    \addcontentsline{toc}{part}{\huge{\protect\color{myyellow}%
      \setlength\fboxrule{2pt}\protect\Circle{%
        \hfil\thepart\hfil%
      }%
    }\\[2ex]\color{myred}\rmfamily#1}}{}{}

%\patchcmd{\@part}
%  {\addcontentsline{toc}{part}{\thepart\hspace{1em}#1}}
%  {\addtocontents{toc}{\protect\addvspace{20pt}}
%    \addcontentsline{toc}{part}{\huge{\protect\color{myyellow}%
%      \setlength\fboxrule{2pt}\protect\fbox{\protect\parbox[c][1em][c]{1.5em}{%
%        \hfil\thepart\hfil%
%      }}%
%    }\\[2ex]\color{myred}\sffamily#1}}{}{}
\makeatother
% this is the environment used to typeset the chapter entries in the ToC
% it is a modification of the leftbar environment of the framed package
\renewenvironment{leftbar}
  {\def\FrameCommand{\hspace{6em}%
    {\color{myyellow}\vrule width 2pt depth 6pt}\hspace{1em}}%
    \MakeFramed{\parshape 1 0cm \dimexpr\textwidth-6em\relax\FrameRestore}\vskip2pt%
  }
 {\endMakeFramed}

% using titletoc we redefine the ToC entries for parts, chapters, sections, and subsections
\titlecontents{part}
  [0em]{\centering}
  {\contentslabel}
  {}{}
\titlecontents{chapter}
  [0em]{\vspace*{2\baselineskip}}
  {\parbox{4.5em}{%
    \hfill\Huge\rmfamily\bfseries\color{myred}\thecontentspage}%
   \vspace*{-2.3\baselineskip}\leftbar\textsc{\small\chaptername~\thecontentslabel}\\\rmfamily}
  {}{\endleftbar}
\titlecontents{section}
  [8.4em]
  {\rmfamily\contentslabel{3em}}{}{}
  {\hspace{0.5em}\nobreak\color{myred}\normalfont\contentspage}
\titlecontents{subsection}
  [8.4em]
  {\rmfamily\contentslabel{3em}}{}{}  
  {\hspace{0.5em}\nobreak\color{myred}\contentspage}


\tcbset{
  tbcsetLavender/.style={
    on line, 
    boxsep=4pt, left=0pt,right=0pt,top=0pt,bottom=0pt,
    colframe=white, colback=LightLavender,  
    highlight math style={enhanced}
  }
}
\tcbset{
  grayb/.style={
    on line, 
    boxsep=4pt, left=0pt,right=0pt,top=0pt,bottom=0pt,
    colframe=white, colback=gray!30,  
    highlight math style={enhanced}
  }
}


%==========================================================================

%PYTHON LSTLISTING STYLE

% Define colors
\definecolor{Pgruvbox-bg}{HTML}{282828}
\definecolor{Pgruvbox-fg}{HTML}{ebdbb2}
\definecolor{Pgruvbox-red}{HTML}{fb4934}
\definecolor{Pgruvbox-green}{HTML}{b8bb26}
\definecolor{Pgruvbox-yellow}{HTML}{fabd2f}
\definecolor{Pgruvbox-blue}{HTML}{83a598}
\definecolor{Pgruvbox-purple}{HTML}{d3869b}
\definecolor{Pgruvbox-aqua}{HTML}{8ec07c}

% Define Python style
\lstdefinestyle{PythonGruvbox}{
	language=Python,
	identifierstyle=\color{lst-fg},
	basicstyle=\ttfamily\color{Pgruvbox-fg},
	keywordstyle=\color{Pgruvbox-yellow},
	keywordstyle=[2]\color{Pgruvbox-blue},
	stringstyle=\color{Pgruvbox-green},
	commentstyle=\color{Pgruvbox-aqua},
	backgroundcolor=\color{Pgruvbox-bg},
	%frame=tb,
	rulecolor=\color{Pgruvbox-fg},
	showstringspaces=false,
	keepspaces=true,
	captionpos=b,
	breaklines=true,
	tabsize=4,
	showspaces=false,
	numbers=left,
	numbersep=5pt,
	numberstyle=\tiny\color{gray},
	showtabs=false,
	columns=fullflexible,
	morekeywords={True,False,None},
	morekeywords=[2]{and,as,assert,break,class,continue,def,del,elif,else,except,exec,finally,for,from,global,if,import,in,is,lambda,nonlocal,not,or,pass,print,raise,return,try,while,with,yield},
	morecomment=[s]{"""}{"""},
	morecomment=[s]{'''}{'''},
	morecomment=[l]{\#},
	morestring=[b]",
	morestring=[b]',
	literate=
	{0}{{\textcolor{Pgruvbox-purple}{0}}}{1}
	{1}{{\textcolor{Pgruvbox-purple}{1}}}{1}
	{2}{{\textcolor{Pgruvbox-purple}{2}}}{1}
	{3}{{\textcolor{Pgruvbox-purple}{3}}}{1}
	{4}{{\textcolor{Pgruvbox-purple}{4}}}{1}
	{5}{{\textcolor{Pgruvbox-purple}{5}}}{1}
	{6}{{\textcolor{Pgruvbox-purple}{6}}}{1}
	{7}{{\textcolor{Pgruvbox-purple}{7}}}{1}
	{8}{{\textcolor{Pgruvbox-purple}{8}}}{1}
	{9}{{\textcolor{Pgruvbox-purple}{9}}}{1}
}
%====================================================================
% 
%====================================================================

% JAVA LSTLISTING STYLE IN Gruvbox Colorscheme
\definecolor{gruvbox-bg}{rgb}{0.282, 0.247, 0.204}
\definecolor{gruvbox-fg1}{rgb}{0.949, 0.898, 0.776}
\definecolor{gruvbox-fg2}{rgb}{0.871, 0.804, 0.671}
\definecolor{gruvbox-red}{rgb}{0.788, 0.255, 0.259}
\definecolor{gruvbox-green}{rgb}{0.518, 0.604, 0.239}
\definecolor{gruvbox-yellow}{rgb}{0.914, 0.808, 0.427}
\definecolor{gruvbox-blue}{rgb}{0.353, 0.510, 0.784}
\definecolor{gruvbox-purple}{rgb}{0.576, 0.412, 0.659}
\definecolor{gruvbox-aqua}{rgb}{0.459, 0.631, 0.737}
\definecolor{gruvbox-gray}{rgb}{0.518, 0.494, 0.471}

\definecolor{lst-bg}{RGB}{45, 45, 45}
\definecolor{lst-fg}{RGB}{220, 220, 204}
\definecolor{lst-keyword}{RGB}{215, 186, 125}
\definecolor{lst-comment}{RGB}{117, 113, 94}
\definecolor{lst-string}{RGB}{163, 190, 140}
\definecolor{lst-number}{RGB}{181, 206, 168}
\definecolor{lst-type}{RGB}{218, 142, 130}


\lstdefinestyle{JavaGruvbox}{
	language=Java,
	basicstyle=\ttfamily\color{lst-fg},
	keywordstyle=\color{lst-keyword},
	keywordstyle=[2]\color{lst-type},
	commentstyle=\itshape\color{lst-comment},
	stringstyle=\color{lst-string},
	numberstyle=\color{lst-number},
	backgroundcolor=\color{lst-bg},
	%frame=tb,
	rulecolor=\color{gruvbox-aqua},
	showstringspaces=false,
	keepspaces=true,
	captionpos=b,
	breaklines=true,
	tabsize=4,
	showspaces=false,
	showtabs=false,
	columns=fullflexible,
	morekeywords={var},
	morekeywords=[2]{boolean, byte, char, double, float, int, long, short, void},
	morecomment=[s]{/}{/},
	morecomment=[l]{//},
	morestring=[b]",
	morestring=[b]',
	numbers=left,
	numbersep=5pt,
	numberstyle=\tiny\color{gray},
}



%====================================================================
% 
%====================================================================


% Define Dracula color scheme for Java
\definecolor{draculawhite-background}{RGB}{237, 239, 252}
\definecolor{draculawhite-comment}{RGB}{98, 114, 164}
\definecolor{draculawhite-keyword}{RGB}{189, 147, 249}
\definecolor{draculawhite-string}{RGB}{152, 195, 121}
\definecolor{draculawhite-number}{RGB}{249, 189, 89}
\definecolor{draculawhite-operator}{RGB}{248, 248, 242}

% Define JavaDraculaWhite lstlisting environment
\lstdefinestyle{JavaDraculaWhite}{
    language=Java,
    backgroundcolor=\color{draculawhite-background},
    commentstyle=\itshape\color{draculawhite-comment},
    keywordstyle=\color{draculawhite-keyword},
    stringstyle=\color{draculawhite-string},
    basicstyle=\ttfamily\footnotesize\color{black},
    identifierstyle=\color{black},
    keywordstyle=\color{draculawhite-keyword}\bfseries,
    morecomment=[s][\color{draculawhite-comment}]{/**}{*/},
    showstringspaces=false,
    showspaces=false,
    breaklines=true,
    frame=single,
    rulecolor=\color{draculawhite-operator},
    tabsize=2,  
	numbers=left,
	numbersep=4pt,
	numberstyle=\ttfamily\tiny\color{gray}
}
%====================================================================
% 
%====================================================================
% Define PythonDraculaWhite lstlisting environment 
\definecolor{draculawhite-bg}{HTML}{FAFAFA}
\definecolor{draculawhite-fg}{HTML}{282A36}
\definecolor{pdraculawhite-keyword}{HTML}{BD93F9}

\definecolor{pdraculawhite-comment}{HTML}{6272A4}
\definecolor{draculawhite-number}{HTML}{FF79C6}


\lstdefinestyle{PythonDraculaWhite}{
    language=Python,
    basicstyle=\ttfamily\small\color{draculawhite-fg},
    backgroundcolor=\color{draculawhite-background},
    keywordstyle=\color{orange}\bfseries,
    stringstyle=\color{draculawhite-string},
    commentstyle=\color{pdraculawhite-comment}\itshape,
    numberstyle=\color{draculawhite-number},
    showstringspaces=false,
	showspaces=false,
    breaklines=true,
	frame=single,
	rulecolor=\color{draculawhite-operator}, 
    tabsize=4,
    morekeywords={as,with,1,2,3,4, 5,6,7,8,9,True,False},
    %escapeinside={(*@}{@*)},
    numbers=left,
    numbersep=5pt,
    %xleftmargin=15pt,
    %framexleftmargin=15pt,
    %framexrightmargin=0pt,
    %framexbottommargin=0pt,
    %framextopmargin=0pt,
    %rulecolor=\color{draculawhite-fg},
    %frame=tb,
    %aboveskip=0pt,
    %belowskip=0pt,
    %captionpos=b,
	numberstyle=\ttfamily\tiny\color{gray} 
}
%====================================================================
% 
%====================================================================

% Define colors for HTML langage
\definecolor{html-orange}{HTML}{FF5733}
\definecolor{html-yellow}{HTML}{F0E130}
\definecolor{html-green}{HTML}{50FA7B}
\definecolor{html-blue}{HTML}{5AFBFF}
\definecolor{html-purple}{HTML}{BD93F9}
\definecolor{html-pink}{HTML}{FF80BF}
\definecolor{html-gray}{HTML}{6272A4}
\definecolor{html-white}{HTML}{F8F8F2}

% Defines a new HTML5 langage that extend on the html langange
\lstdefinestyle{HTMLDraculaWhite}{
  language=HTML,
  backgroundcolor=\color{html-white},
  basicstyle=\ttfamily\color{html-gray},
  keywordstyle=\color{html-blue},
  stringstyle=\color{html-orange},
  commentstyle=\color{html-green},
  tagstyle=\color{html-yellow},
  moredelim=[s][\color{html-pink}]{<!--}{-->},
  moredelim=[s][\color{html-purple}]{\{}{\}},
  showstringspaces=false,
  tabsize=2,
  breaklines=true,
  columns=fullflexible,
  %frame=single,
  framexleftmargin=5mm,
  xleftmargin=10mm,
  numbers=left,
  numberstyle=\tiny\color{html-gray},
  escapeinside={<@}{@>}
}

%====================================================================
% 
%====================================================================
% Define the colors needed for the HTMLDraculaDark environment
\definecolor{htmltag}{HTML}{ff79c6}
\definecolor{htmlattr}{HTML}{f1fa8c}
\definecolor{htmlvalue}{HTML}{bd93f9}
\definecolor{htmlcomment}{HTML}{6272a4}
%\definecolor{htmltext}{HTML}{f8f8f2}
\definecolor{htmltext}{HTML}{401E31}
\definecolor{htmlbackground}{HTML}{282a36}
\definecolor{comphtmlbackground}{HTML}{8093FF}
%\definecolor{htmlbackground}{HTML}{4D5169}

% Define the HTMLDraculaDark environment
\lstdefinestyle{HTMLDraculaDark}{
    basicstyle=\bfseries\ttfamily\color{htmltext},
    commentstyle=\itshape\color{htmlcomment},
    keywordstyle=\bfseries\color{htmltag},
    stringstyle=\color{htmlvalue},
    emph={DOCTYPE,html,head,body,div,span,a,script},
    emphstyle={\color{htmltag}\bfseries},
    sensitive=true,
    showstringspaces=false,
    backgroundcolor=\color{white},
    %frame=tb,
    language=HTML,
    tabsize=4,
    breaklines=true,
    breakatwhitespace=true,
    numbers=left,
    numbersep=10pt,
    numberstyle=\small\bfseries\ttfamily\color{htmlcomment},
    escapeinside={<@}{@>},
	rulecolor=\color{htmlbackground},
	xleftmargin=20pt,
	% Add a vertical line for opening and closing tags
    %frame=lines,
    framesep=2pt,
    framexleftmargin=4pt,
    % Add a vertical line for closing tags that go through multiple lines
    breaklines=true,
    postbreak=\mbox{\textcolor{gray}{$\hookrightarrow$}\space},
    showlines=true,
	% Add a rule to apply commentstyle to keywords inside comments
    moredelim=[s][\itshape\color{htmlcomment}]{<!--}{-->},
    morekeywords={id,class,type,name,value,placeholder,checked,src,href,alt}
}




%====================================================================
% 
%====================================================================






% Crée un environnement "Theorem" numéroté en fonction du document
\tcbuselibrary{theorems,skins,hooks} 
\newtcbtheorem{Theorem}{Théorème}
{%
	enhanced,
	breakable,
	colback = mytheorembg,
	frame hidden,
	boxrule = 0sp,
	borderline west = {2pt}{0pt}{mytheoremfr},
	sharp corners,
	detach title,
	before upper = \tcbtitle\par\smallskip,
	coltitle = mytheoremfr,
	fonttitle = \bfseries\fontfamily{lmss}\selectfont,
	description font = \mdseries\fontfamily{lmss}\selectfont,
	separator sign none,
	segmentation style={solid, mytheoremfr},
}
{thm}

% Crée un environnement "Preuve" numéroté en fonction du document
\tcbuselibrary{theorems,skins,hooks}
\newtcbtheorem{Preuve}{Preuve}
{
	enhanced,
	breakable,
	colback=white,
	frame hidden,
	boxrule = 0sp,
	borderline west = {2pt}{0pt}{mytheoremfr},
	sharp corners,
	detach title,
	before upper = \tcbtitle\par\smallskip,
	coltitle = mytheoremfr,
	description font=\fontfamily{lmss}\selectfont,
	fonttitle=\fontfamily{lmss}\selectfont\bfseries,
	separator sign none,
	segmentation style={solid, mytheoremfr},
}
{th}


% Crée un environnement "Preuve" numéroté en fonction du document
\tcbuselibrary{theorems,skins,hooks}
\newtcbtheorem{Explication}{Explication}
{
	enhanced,
	breakable,
	colback=white,
	frame hidden,
	boxrule = 0sp,
	borderline west = {2pt}{0pt}{mytheoremfr},
	sharp corners,
	detach title,
	before upper = \tcbtitle\par\smallskip,
	coltitle = mytheoremfr,
	description font=\fontfamily{lmss}\selectfont,
	fonttitle=\fontfamily{lmss}\selectfont\bfseries,
	separator sign none,
	segmentation style={solid, mytheoremfr},
}
{th}




% Crée un environnement "Example" numéroté en fonction du document
\tcbuselibrary{theorems,skins,hooks}
\newtcbtheorem{Example}{Exemple.}
{
	enhanced,
	breakable,
	colback=lightBlue,
	frame hidden,
	boxrule = 0sp,
	borderline west = {2pt}{0pt}{myb},
	sharp corners,
	detach title,
	before upper = \tcbtitle\par\smallskip,
	coltitle = myb,
	description font=\fontfamily{lmss}\selectfont,
	fonttitle=\fontfamily{lmss}\selectfont\bfseries,
	separator sign none,
	segmentation style={solid, mytheoremfr},
}
{th}



% Crée un environnement "EExample" numéroté en fonction du document
\tcbuselibrary{theorems,skins,hooks}
\newtcbtheorem{EExample}{Exemple}
{
	enhanced,
	breakable,
	colback=white,
	frame hidden,
	boxrule = 0sp,
	borderline west = {2pt}{0pt}{myb},
	sharp corners,
	detach title,
	before upper = \tcbtitle\par\smallskip,
	coltitle = myb,
	description font=\mdseries\fontfamily{lmss}\selectfont,
	fonttitle=\fontfamily{lmss}\selectfont\bfseries,
	separator sign none,
	segmentation style={solid, mytheoremfr},
}
{th}



% Crée un environnement "Lemme" numéroté en fonction du document
\tcbuselibrary{theorems,skins,hooks}
\newtcbtheorem{Lemme}{Lemme}
{
	enhanced,
	breakable,
	colback=mylenmabg,
	frame hidden,
	boxrule = 0sp,
	borderline west = {2pt}{0pt}{mylenmafr},
	sharp corners,
	detach title,
	before upper = \tcbtitle\par\smallskip,
	coltitle = mylenmafr,
	description font=\mdseries\fontfamily{lmss}\selectfont,
	fonttitle=\fontfamily{lmss}\selectfont\bfseries,
	separator sign none,
	segmentation style={solid, mytheoremfr},
}
{th}


\tcbuselibrary{theorems,skins,hooks}
\newtcbtheorem{PreuveL}{Preuve.}
{
	enhanced,
	breakable,
	colback=white,
	frame hidden,
	boxrule = 0sp,
	borderline west = {2pt}{0pt}{mylenmafr},
	sharp corners,
	detach title,
	before upper = \tcbtitle\par\smallskip,
	coltitle = mylenmafr,
	description font=\fontfamily{lmss}\selectfont,
	fonttitle=\fontfamily{lmss}\selectfont\bfseries,
	separator sign none,
	segmentation style={solid, mytheoremfr},
}
{th}


\newtcbtheorem{Remarque}{Remarque}
{
	enhanced,
	breakable,
	colback=white,
	frame hidden,
	boxrule = 0sp,
	borderline west = {2pt}{0pt}{myb},
	sharp corners,
	detach title,
	before upper = \tcbtitle\par\smallskip,
	coltitle = myb,
	description font=\mdseries\fontfamily{lmss}\selectfont,
	fonttitle=\fontfamily{lmss}\selectfont\bfseries,
	separator sign none,
	segmentation style={solid, mytheoremfr},
}
{th}


\newtcbtheorem{DefG}{Définition}
{
	enhanced,
	breakable,
	colback=mygbg,
	frame hidden,
	boxrule = 0sp,
	borderline west = {2pt}{0pt}{myg},
	sharp corners,
	detach title,
	before upper = \tcbtitle\par\smallskip,
	coltitle = myg,
	description font=\mdseries\fontfamily{lmss}\selectfont,
	fonttitle=\fontfamily{lmss}\selectfont\bfseries,
	separator sign none,
	segmentation style={solid, mytheoremfr},
}
{th}



% Crée une boîte ayant la même couleur que l'environnement theorem.
\tcbuselibrary{theorems,skins,hooks}
\newtcolorbox{Theoremcon}
{%
	enhanced
	,breakable
	,colback = mytheorembg
	,frame hidden
	,boxrule = 0sp
	,borderline west = {2pt}{0pt}{mytheoremfr}
	,sharp corners
	,description font = \mdseries
	,separator sign none
}

% Crée un environnement "Definition" numéroté en fonction de la section
\newtcbtheorem[number within=chapter]{Definition}{Définition}{enhanced,
	before skip=2mm,after skip=2mm, colback=red!5,colframe=red!80!black,boxrule=0.5mm,
	attach boxed title to top left={xshift=1cm,yshift*=1mm-\tcboxedtitleheight}, varwidth boxed title*=-3cm,
	boxed title style={frame code={
			\path[fill=tcbcolback!10!red]
			([yshift=-1mm,xshift=-1mm]frame.north west)
			arc[start angle=0,end angle=180,radius=1mm]
			([yshift=-1mm,xshift=1mm]frame.north east)
			arc[start angle=180,end angle=0,radius=1mm];
			\path[left color=tcbcolback!10!myred,right color=tcbcolback!10!myred,
			middle color=tcbcolback!60!myred]
			([xshift=-2mm]frame.north west) -- ([xshift=2mm]frame.north east)
			[rounded corners=1mm]-- ([xshift=1mm,yshift=-1mm]frame.north east)
			-- (frame.south east) -- (frame.south west)
			-- ([xshift=-1mm,yshift=-1mm]frame.north west)
			[sharp corners]-- cycle;
		},interior engine=empty,
	},
	fonttitle=\bfseries,
	title={#2},#1}{def}

% Crée un environnement "definition" numéroté en fonction du Chapitre
\newtcbtheorem[number within=section]{definition}{Définition}{enhanced,
	before skip=2mm,after skip=2mm, colback=red!5,colframe=red!80!black,boxrule=0.5mm,
	attach boxed title to top left={xshift=1cm,yshift*=1mm-\tcboxedtitleheight}, varwidth boxed title*=-3cm,
	boxed title style={frame code={
			\path[fill=tcbcolback]
			([yshift=-1mm,xshift=-1mm]frame.north west)
			arc[start angle=0,end angle=180,radius=1mm]
			([yshift=-1mm,xshift=1mm]frame.north east)
			arc[start angle=180,end angle=0,radius=1mm];
			\path[left color=tcbcolback!60!black,right color=tcbcolback!60!black,
			middle color=tcbcolback!80!black]
			([xshift=-2mm]frame.north west) -- ([xshift=2mm]frame.north east)
			[rounded corners=1mm]-- ([xshift=1mm,yshift=-1mm]frame.north east)
			-- (frame.south east) -- (frame.south west)
			-- ([xshift=-1mm,yshift=-1mm]frame.north west)
			[sharp corners]-- cycle;
		},interior engine=empty,
	},
	fonttitle=\bfseries,
	title={#2},#1}{def}

\usetikzlibrary{arrows,calc,shadows.blur}
\tcbuselibrary{skins}
\newtcolorbox{note}[1][]{%
	enhanced jigsaw,
	colback=gray!20!white,%
	colframe=gray!80!black,
	size=small,
	boxrule=1pt,
	title=\textbf{Note : },
	halign title=flush center,
	coltitle=black,
	breakable,
	drop shadow=black!50!white,
	attach boxed title to top left={xshift=1cm,yshift=-\tcboxedtitleheight/2,yshifttext=-\tcboxedtitleheight/2},
	minipage boxed title=1.5cm,
	boxed title style={%
		colback=white,
		size=fbox,
		boxrule=1pt,
		boxsep=2pt,
		underlay={%
			\coordinate (dotA) at ($(interior.west) + (-0.5pt,0)$);
			\coordinate (dotB) at ($(interior.east) + (0.5pt,0)$);
			\begin{scope}
				\clip (interior.north west) rectangle ([xshift=3ex]interior.east);
				\filldraw [white, blur shadow={shadow opacity=60, shadow yshift=-.75ex}, rounded corners=2pt] (interior.north west) rectangle (interior.south east);
			\end{scope}
			\begin{scope}[gray!80!black]
				\fill (dotA) circle (2pt);
				\fill (dotB) circle (2pt);
			\end{scope}
		},
	},
	#1,
}


% Crée un environnement "qstion" 
\newtcbtheorem{qstion}{Question}{enhanced,
	breakable,
	colback=white,
	colframe=mygr,
	attach boxed title to top left={yshift*=-\tcboxedtitleheight},
	fonttitle=\bfseries,
	title={#2},
	boxed title size=title,
	boxed title style={%
		sharp corners,
		rounded corners=northwest,
		colback=tcbcolframe,
		boxrule=0pt,
	},
}{def}


% Pour créer un environnement "Liste" 

\tcbuselibrary{theorems,skins,hooks}
\newtcbtheorem[number within=section]{Liste}{Liste}
{%
	enhanced
	,breakable
	,colback = myp!10
	,frame hidden
	,boxrule = 0sp
	,borderline west = {2pt}{0pt}{myp!85!black}
	,sharp corners
	,detach title
	,before upper = \tcbtitle\par\smallskip
	,coltitle = myp!85!black
	,fonttitle = \bfseries\sffamily
	,description font = \mdseries
	,separator sign none
	,segmentation style={solid, myp!85!black}
}
{th}


\tcbuselibrary{theorems,skins,hooks}
\newtcbtheorem{Syntaxe}{Syntaxe.}
{%
	enhanced
	,breakable
	,colback = myp!10
	,frame hidden
	,boxrule = 0sp
	,borderline west = {2pt}{0pt}{myp!85!black}
	,sharp corners
	,detach title
	,before upper = \tcbtitle\par\smallskip
	,coltitle = myp!85!black
	,fonttitle = \bfseries\fontfamily{lmss}\selectfont 
	,description font = \mdseries\fontfamily{lmss}\selectfont 
	,separator sign none
	,segmentation style={solid, myp!85!black}
}
{th}



% Crée un environnement "Concept" numéroté en fonction du document
\tcbuselibrary{theorems,skins,hooks}
\newtcbtheorem{Concept}{Concept.}
{
	enhanced,
	breakable,
	colback=mylenmabg,
	frame hidden,
	boxrule = 0sp,
	borderline west = {2pt}{0pt}{mylenmafr},
	sharp corners,
	detach title,
	before upper = \tcbtitle\par\smallskip,
	coltitle = mylenmafr,
	description font=\mdseries\fontfamily{lmss}\selectfont,
	fonttitle=\fontfamily{lmss}\selectfont\bfseries,
	separator sign none,
	segmentation style={solid, mytheoremfr},
}
{th}


% Crée un environnement "codeEx" numéroté en fonction du document
\tcbuselibrary{theorems,skins,hooks}
\newtcbtheorem{codeEx}{Exemple}
{
	enhanced,
	breakable,
	colback=white,
	frame hidden,
	boxrule = 0sp,
	borderline west = {2pt}{0pt}{gruvbox-bg},
	sharp corners,
	detach title,
	before upper = \tcbtitle\par\smallskip,
	coltitle = gruvbox-bg,
	description font=\md:wqseries\fontfamily{lmss}\selectfont,
	fonttitle=\fontfamily{lmss}\selectfont\bfseries,
	separator sign none,
	segmentation style={solid, mytheoremfr},
}
{th}


% Crée un environnement "codeEx" numéroté en fonction du document
\tcbuselibrary{theorems,skins,hooks}
\newtcbtheorem{codeRem}{Remarque.}
{
	enhanced,
	breakable,
	colback=white,
	frame hidden,
	boxrule = 0sp,
	borderline west = {2pt}{0pt}{gruvbox-bg},
	sharp corners,
	detach title,
	before upper = \tcbtitle\par\smallskip,
	coltitle = gruvbox-bg,
	description font=\mdseries\fontfamily{lmss}\selectfont,
	fonttitle=\fontfamily{lmss}\selectfont\bfseries,
	separator sign none,
	segmentation style={solid, mytheoremfr},
}
{th}


\tcbuselibrary{theorems,skins,hooks}
\newtcbtheorem{Identite}{Identité.}
{
	enhanced,
	breakable,
	colback=white,
  before upper=\tcbtitle\par\Hugeskip,
	frame hidden,
	boxrule = 0sp,
	borderline west = {2pt}{0pt}{gruvbox-bg},
	sharp corners,
	detach title,
	before upper = \tcbtitle\par\smallskip,
	coltitle = gruvbox-bg,
	description font=\mdseries\fontfamily{lmss}\selectfont,
	fonttitle=\fontfamily{lmss}\selectfont\bfseries,
	fontlower=\fontfamily{cmr}\selectfont,
  separator sign none,
	segmentation style={solid, mytheoremfr},
}
{th}

\tcbuselibrary{theorems,skins,hooks}
\newtcbtheorem{Exercice}{Exercice}
{
	enhanced,
	breakable,
	colback=white,
  before upper=\tcbtitle\par\Hugeskip,
	frame hidden,
	boxrule = 0sp,
	borderline west = {2pt}{0pt}{gruvbox-green},
	sharp corners,
	detach title,
	before upper = \tcbtitle\par\smallskip,
	coltitle = gruvbox-green,
	description font=\mdseries\fontfamily{lmss}\selectfont,
	fonttitle=\fontfamily{lmss}\selectfont\bfseries,
	fontlower=\fontfamily{cmr}\selectfont,
  separator sign none,
	segmentation style={solid, mytheoremfr},
}
{th}

% Crée un environnement "Réponse" numéroté en fonction du document
\tcbuselibrary{theorems,skins,hooks}
\newtcbtheorem{Reponse}{Reponse}
{
	enhanced,
	breakable,
	colback=white,
	frame hidden,
	boxrule = 0sp,
	borderline west = {2pt}{0pt}{mytheoremfr},
	sharp corners,
	detach title,
	before upper = \tcbtitle\par\smallskip,
	coltitle = mytheoremfr,
	description font=\fontfamily{lmss}\selectfont,
	fonttitle=\fontfamily{lmss}\selectfont\bfseries,
	separator sign none,
	segmentation style={solid, mytheoremfr},
}
{th}

\newtcbtheorem{Definitionx}{Définition}
{
enhanced,
breakable,
colback=red!5,
  before upper=\tcbtitle\par\Hugeskip,
frame hidden,
boxrule = 0sp,
borderline west = {2pt}{0pt}{red!80!black},
sharp corners,
detach title,
before upper = \tcbtitle\par\smallskip,
coltitle = red!80!black,
description font=\mdseries\fontfamily{lmss}\selectfont,
fonttitle=\fontfamily{lmss}\selectfont\bfseries,
fontlower=\fontfamily{cmr}\selectfont,
  separator sign none,
segmentation style={solid, mytheoremfr},
}
{th}

\tcbuselibrary{theorems,skins,hooks}
\newtcbtheorem[number within=chapter]{prop}{Proposition}
{%
	enhanced,
	breakable,
	colback = mypropbg,
	frame hidden,
	boxrule = 0sp,
	borderline west = {2pt}{0pt}{mypropfr},
	sharp corners,
	detach title,
	before upper = \tcbtitle\par\smallskip,
	coltitle = mypropfr,
	fonttitle = \bfseries\sffamily,
	description font = \mdseries,
	separator sign none,
	segmentation style={solid, mypropfr},
}
{th}


\tcbuselibrary{theorems,skins,hooks}
\newtcbtheorem[number within=section]{Prop}{Proposition}
{%
	enhanced,
	breakable,
	colback = mypropbg,
	frame hidden,
	boxrule = 0sp,
	borderline west = {2pt}{0pt}{mypropfr},
	sharp corners,
	detach title,
	before upper = \tcbtitle\par\smallskip,
	coltitle = mypropfr,
	fonttitle = \bfseries\sffamily,
	description font = \mdseries,
	separator sign none,
	segmentation style={solid, mypropfr},
}
{th}


%================================
% Corollery
%================================
\tcbuselibrary{theorems,skins,hooks}
\newtcbtheorem[number within=section]{Corollary}{Corollary}
{%
	enhanced
	,breakable
	,colback = myp!10
	,frame hidden
	,boxrule = 0sp
	,borderline west = {2pt}{0pt}{myp!85!black}
	,sharp corners
	,detach title
	,before upper = \tcbtitle\par\smallskip
	,coltitle = myp!85!black
	,fonttitle = \bfseries\sffamily
	,description font = \mdseries
	,separator sign none
	,segmentation style={solid, myp!85!black}
}
{th}
\tcbuselibrary{theorems,skins,hooks}
\newtcbtheorem[number within=chapter]{corollary}{Corollaire}
{%
	enhanced
	,breakable
	,colback = myp!10
	,frame hidden
	,boxrule = 0sp
	,borderline west = {2pt}{0pt}{myp!85!black}
	,sharp corners
	,detach title
	,before upper = \tcbtitle\par\smallskip
	,coltitle = myp!85!black
	,fonttitle = \bfseries\sffamily
	,description font = \mdseries
	,separator sign none
	,segmentation style={solid, myp!85!black}
}
{th}



\usepackage[scr]{rsfso}

%\usepackage{libertine}
\usepackage[euler-digits]{eulervm}
%\usepackage{mathpazo}
%\usepackage{palatino}
%usepackage{crimson}






\title{\Huge{Calcul 1}\\{MATH1400}\\{\textbf{Introduction}}}
\author{\huge{Franz Girardin}}
\date{\today}
\lstset{inputencoding=utf8/latin1}

            %%%%%%%%%%%%%%%%%  Sect.                          %%%%%%%%%%%%%%%%%%%%%%%%%%%%%%%%%%%%%%%%%%%%%%%%%%%%%%%%%
\usepackage{helvet}
\titleformat{\chapter}
  {\fontfamily{phv}\bfseries\huge} % format
  {}                % label
  {0pt}             % sep
  {\color{myb}\huge}           % before-code



\titleformat{\section}
  {\normalfont\scshape}{\thesection}{1em}{}


% Customizing the spacing for the chapter titles
\titlespacing*{\chapter}{0pt}{0pt}{20pt}

% Allow hfill in math environment
\newcommand{\specialcell}[1]{\ifmeasuring@#1\else\omit$\displaystyle#1$\ignorespaces\fi}

% Allow you to do the non implication (implication barred)
\newcommand{\notimplies}{%
  \mathrel{{\ooalign{\hidewidth$\not\phantom{=}$\hidewidth\cr$\implies$}}}}



\DeclareRobustCommand{\looongrightarrow}{%
  \DOTSB\relbar\joinrel\relbar\joinrel\relbar\joinrel\rightarrow
}


\DeclareMathOperator{\di}{d\!}
\newcommand*\Eval[3]{\left.#1\right\rvert_{#2}^{#3}}

\begin{document}
\maketitle

\pagebreak

\pagebreak
\begin{multicols*}{2}


  \paragraph{Théorème de correspondace}
  Montrer que $\lim\limits_{n \to+\infty }\sqrt[n]{n}  = 1$ 

  \mbox{}\vspace{1em}\\
  Considérons $a_n = \sqrt[n]{n} = n^{1/n}$ et évaluation sa convergence. 
  Nous avons une forme indéterminé $ \frac{\infty}{\infty} $. Nous pouvons 
  alors appliquer la règle de l'\textbf{H}. 

  \begin{align*}
    \lim\limits_{n \to+\infty } \ln(a_n) 
        &= 
    \ln\left(\lim\limits_{n \to+\infty }\sqrt[n]{n}\right) 
         = 
        \lim\limits_{n\to+\infty} \ln(n^{1/n}) \\ 
        &=        
        \lim\limits_{n \to+\infty } \ln \frac{n}{n} \\ 
        &= 
        \lim\limits_{n \to+\infty } 0 = 0  \\ 
  \end{align*}
  Puisque $\ln(a_n)$  converge vers $f(L) = 0$, on a alors 
  \begin{align*}
   \lim\limits_{n\to+\infty }a_n 
   = 
   \lim\limits_{n \to+\infty}e^{\ln(a_n)}               
   = 
   e^{\lim\limits_{n \to+\infty }\ln(a_n)}  =  e^0 = 1
  \end{align*}

  \paragraph{Règle de l'Hôpital}
  Calculer $\lim\limits_{n \to+\infty }\dfrac{\ln(1+x)}{x} $    
  \mbox{}\vspace{1em}\\


  \begin{align*}
    \lim\limits_{n \to+\infty }\dfrac{\ln(1+x)}{x} 
    &= \lim\limits_{n \to+\infty }\cancel{\frac{\ln(1)}{x}} + \frac{\ln(x)}{x}    \\ 
    &\overset{\mathcal{h}}{=} 
  \lim\limits_{n\to+\infty } \frac{1/x}{1} = 0
  \end{align*}


  \paragraph{Règle de l'Hôpital}
  Montrer que $\lim\limits_{n \to+\infty } (1 + \frac{1}{n})^n = e$ 

  \mbox{}\vspace{1em}\\

  \begin{align*}
    \lim\limits_{n \to+\infty } 
    \ln\left(1 + \frac{1}{n}\right)^n              
    &= 
    \lim\limits_{n \to+\infty }\cancel{n\ln(1)} + n\ln\frac{1}{n} \\ 
    &\implies (\infty \cdot 0) \\ 
    &= 
    \lim\limits_{n \to+\infty }\frac{\ln \left(1 + \frac{1}{n}\right)}{1/n} \\ 
    &\implies  \frac{0}{\infty } \\ 
    &\overset{\mathcal{H} }{=} 
    \lim\limits_{n \to+\infty } \frac{-n^{-2}}{-n^{-2}}  = 1 
  \end{align*}
  Puisque $\lim\limits_{n \to+\infty }\ln(a_n) = 1$, nous savons que
  la séquence $a_n$ converge vers $1$, nous avons alors :
  \[e^{\lim\limits_{n \to+\infty }\ln(a_n)} = 
  e^{\ln \left(\lim\limits_{n \to+\infty }a_n\right)} = e^1 = e \]
  \columnbreak

  \paragraph{Convergence série géométrique}
  \mbox{}\vspace{0.2em} \\
  Est-ce que la série $\sum_{n =1}^{\infty }2^n3^{1-n}$ 
  converge ? Si oui, calculer
  \begin{align*}
    \sum_{n =1}^{\infty }2^n3^{1-n} &=  
    \sum_{n=1}^{\infty }2^n3^{-n}3^1 = 3 \sum_{n=1}^{\infty } 
    \left(\frac{2}{3}\right)^n   \\ 
    \frac{3}{1 - 2/3} = 9 
  \end{align*}

  \paragraph{Propriétés additives}
  Calculer $\sum_{n=1}^{\infty }\left( \frac{3}{n(n+1)} 
  \frac{1}{2^n} \right)$

  \begin{align*}
   \sum_{n=1}^{\infty }\left( \frac{3}{n(n+1)} 
   \frac{1}{2^n} \right) 
   &= 
  \sum_{n=1}^{\infty }\frac{3}{n(n+1)} + 
  \sum_{n=1}^{\infty }\frac{1}{2^n} \\ 
   &= \lim\limits_{n \to+\infty } \frac{3/n}{n+1} + 
   \lim\limits_{n \to+\infty } \frac{1}{2^n}  \\
   &= 0 + 0 = 0  
  \end{align*}

  \paragraph{} 
  \mbox{}\vspace{0.2em} \\
  Trouver les valeurs positives de $b$  pour lesquelles la 
  série $\sum_{n=1}^{\infty }b^{\ln n}$ converge.

  \mbox{} \vspace{0.2em} \\ 
  Tentons de convertir la série donnée en série de 
  Riemann. Nous pouvons considérer la somme :
  \begin{align*}
    \sum_{n=1}^{\infty }b^{\ln n} = 
    \sum_{n=1}^{\infty }  a_n &\implies a_n = (e^{\ln b})^{\ln n} \\             
                  &\implies 
                  a_n = (e^{\ln n})^{\ln b} \\ 
                  &\implies 
                  a_n = n^{\ln b}
  \end{align*}


  \paragraph{Test de comparaison}
  Montrer que $\sum_{n=1}^{\infty }\frac{n^2 + 1}{n^3 +2}$ converge 

  \mbox{} \\
  Considérons $\sum_{n=1}^{\infty }b_n = \sum_{n=1}^{\infty }\frac{1}{n}$
  Nous pouvons alors évaluer la limite suivante


  \begin{align*}
    \lim\limits_{n \to \infty }\frac{a_n}{b_n} 
    &= 
    \lim\limits_{n \to+\infty }\sum_{n=1}^{\infty }\frac{n^2 + 1}{n^3 +2}
    \div \frac{1}{n}  
    = 
    \lim\limits_{n \to+\infty } \frac{n(n^2 + 1)}{n^3 + 2}   \\ 
    &=
    \lim\limits_{n \to+\infty } \frac{n^3 + n}{n^3 + 2}  
    \lim\limits_{n \to+\infty } 
    =
    \dfrac{n^3\left(1 + \dfrac{1}{n^2} \right)}{n^3\left(1 + \dfrac{1}{n^3} \right)}   
    = 1 > 0 
  \end{align*}
  Puisque $\lim\limits_{n \to \infty }\frac{a_n}{b_n} = 1 > 0$ et que 
  $\sum_{n=1}^{\infty }b_n$ diverge, il s'ensuit que $\sum_{n=1}^{\infty }a_n$ 
  diverge également. 
  $\lambda a_1 \vec{w_1}$


  \paragraph{Estimation du reste}
  Trouver un approximation de $\sum_{n=1}^{\infty }\frac{1}{n^3}$ telle que 
  $|erreur| \leq 0.005$. 
  \mbox{}\vspace{1em}\\
  Considérons la fonction $f : \mathbb{R} \rightarrow  \mathbb{R}$  telle que 
  $f(n) = a_n \forall \; n \in \mathbb{N}$. 
  Nous pouvons alors exprimer le reste $R_m$ en fonction de $f$ :

  \begin{align*}
    R_m \leq \int_{m}^{\infty }x^{-3}dx 
    & =  \Eval{-\frac{1}{2x^2}}{m}{\infty}  
      = \lim\limits_{m \to+\infty }\frac{-1}{2m^2} -\left[ \frac{-1}{2m^2} \right] 
    & = \frac{1}{2m^2} 
  \end{align*}

  Pour que le reste soit tel que $R_m \leq 0.005$, 
  c'est-à-dire l'erreur
  permise, nous avons :
  \begin{align*}
    R_m \leq 0.005 &\implies \frac{1}{2m^2} \leq 0.005 \\   
                   &\implies m \geq \sqrt{\frac{2}{0.005}} = 10 
  \end{align*}

  Donc, après $m = 10$ la somme $s_n$ est telle 
  que la différence 
  $s - s_n$ a une erreur $R_m$ de 0.005 ou moins. 
  On peut alors conclure que 
  
  \begin{align*}
    \sum_{n=1}^{\infty }\frac{1}{n^3} \approx \sum_{k=1}^{k = 10} a_k = 
    \frac{1}{2k^2} = \frac{1}{2(10)^2} \approx 1.1975
  \end{align*}


  \paragraph{Test de comparaison} Évaluer la convergence de 
  $\sum_{n=1}^{\infty }\sin\left( \frac{1}{n}  \right) $
  Nous savons que $\sum_{n=1}^{\infty } \sin \left(\frac{1}{n} \right)$ est une 
  \textbf{série à termes positifs}.   
  Considérons la série harmonique et faison le comparaison suivante : 

  \begin{align*}
    \lim\limits_{n \to \infty } \dfrac{\sin \frac{1}{n} }{\frac{1}{n} }  
    &= 
    \lim\limits_{a \to 0 + } \frac{\sin a}{a}   
    \overset{\mathcal{H} }{=} \lim\limits_{a \to 0+} \frac{\cos a}{1} 
    = 1 > 0
  \end{align*}

  Puisque la série harmonique diverge, on peut conclure, 
  par la forme limite 
  du teste de comparaison, que la série $\sum_{n=1}^{\infty }\sin(n)$ diverge 
  également. 

  \paragraph{Convergence absolue}
  \mbox{}\\
  Évaluer la convergence de $\sum_{n=1}^{\infty }
  \dfrac{\sin \left(\frac{n\pi}{4}\right)}{n^2}$ 

  \mbox{}\\ 
  Puisque la série oscille, nous alons évaluer les 
  termes positifs 
  de la fonction 
  et établir s'il y a convergence absolue par le test de 
  comparaison. Considérons 
  la série à termes positifs suivante : 
  \begin{align*}
      \sum_{n=1}^{\infty }|a_n| =  
      \sum_{n=1}^{\infty }\dfrac{|\sin \left(\frac{n\pi}{4}\right)|}{n^2} 
      \\ 
      \forall \; n \in \mathbb{N}, \big|\sin\left(\frac{n\pi}{4} \right)\big| \leq 1 
      \implies 
      |a_n| \leq \frac{1}{n^2}   
  \end{align*}    
  Puisque la série $\sum_{n=1}^{\infty }a_n$ bornée par 
  la série à terme positif $\sum_{n=1}^{\infty }|a_n|$ et que cette dernière 
  est elle même bornée par $\sum_{n=1}^{\infty }\frac{1}{n^2}$ qui est une 
  série de Riemann avec $p = 2$ et donc convergente, nous pouvons 
  conclure que $\sum_{n=1}^{\infty }a_n$ converge absoluement. 

  \paragraph{Identification d'une série entière}
      Déterminer si la série suivant est une série entière
      \begin{align*}
          \sum_{n=1}^{\infty }x^{n!}
      \end{align*}        
  \mbox{}\\ 
  Dans cet exemple, le coefficient constant est 
  $a_n = 1$ et $x$ est la variable de la série. Or, les termes de la série 
  augmentent par un facteur de $n!$ et non $n$. Ainsi, puisque la série 
  n'a pas la forme générale : 
  \begin{align*}
      \sum_{n=1}^{\infty }a_n(x-a)^n 
  \end{align*}
  on peut conclure que cette série \textbf{n'est pas} un 
  \textit{une série entière}.


  \paragraph{Identification d'une série entière}
      Déterminer si la série suivant est une série entière
      \begin{align*}
          \sum_{n=-1}^{\infty }x^{n} 
      \end{align*}        
  \mbox{}\\ 
  Cette série comment à $n = -1$, ce qui implique que le premier terme est 
  $x^{-n}$. Par conséquent, la série ne respecte pas la condition 
  $n \in \mathbb{N}$ et on peut conclure qu'elle n'est pas entière. 


  \paragraph{Rayon de convergence et Test du Rapport}
      Déterminer le rayon de convergence de la série
      \begin{align*}
          \sum_{n=1}^{\infty }2^nn^2x^n
      \end{align*}
  
  \mbox{}\\ 
  Soit $a_n = 2^nn^2x^n$, on a 
  \begin{align*}
      \lim\limits_{n \to+\infty }\sqrt[n]{|a_n|}  &= 
      \lim\limits_{n \to+\infty }\sqrt[n]{|2^nn^2x^n|}  = 2|x|
      \lim\limits_{n \to+\infty }\sqrt[n]{n^2}  \\ 
                                      &= 
      2|x|n^{\lim\limits_{n \to+\infty } \frac{1}{x}} = 2|x|n^0 = 2|x|
  \end{align*}
  Par le critère de Cauchy la série converge lorsque 
  $\sum a_n = L = 2|x| < 1$. Donc, il faut que $|x| < 1/2$. 
  ou $ -1/2 < x < 1/2 $  
  Cela implique que $R = 1/2$. 


  \paragraph{Rayon de convergence et Test du Rapport}
      Déterminer le rayon de convergence de la série
      \begin{align*}
          \sum_{n=1}^{\infty }\frac{x^n}{n!} 
      \end{align*} 
  \mbox{}\\ 

  Soit $a_n = \frac{x^n}{n!} $, on a 
  \begin{align*}
    \lim\limits_{n \to+\infty }\frac{a_{n+1}}{a_n}   &= 
    \lim\limits_{n\to+\infty } \frac{n! x^{(n+1)}}{(n+1)!x^n}= 
    \lim\limits_{n\to+\infty } \frac{x^{1}}{(n+1)!} \\ 
                                          &= x \cdot 0 = 0
  \end{align*}

  L'équation $x \cdot 0$ est vraie $\forall x \in \mathbb{R}$. Ainsi, 
  par le test du rapport, le rayon de convergence est $R = \infty$


  \paragraph{Trouver l'intervale de convergence}
  Soit la série $\sum_{n=2}^{\infty }\frac{x^{3n}}{n(\ln n)^2}$, 
  trouver le centre, $a$. 

  \mbox{}\\
  Soit $a_n = \frac{x^{3n}}{n(\ln n)^2}$ nous pouvons utiliser le 
  test du rapport pour déterminer la convergence de la série. 


  \begin{align*}
    \lim\limits_{n \to+\infty }  \Big|\frac{a_{n+1}}{a_n}\Big| &=  
    \lim\limits_{n \to+\infty } \frac{x^{3(n+1)}}{(n+1)(\ln n + 1 )^2}  
    \cdot                \frac{n(\ln n)^2}{x^{3n}} \\ 
                                          &\overset{\mathcal{H}}{=}
      |x|^3 
   \end{align*}

   Par le test du  rapport, la série converge lorsque 
   $\sum_{n=2}^{\infty }a_n = L = |x|^3 < 1$. Donc, il faut que 
   $x < 1$ ou $x < -1$. Cela implique que $R = 1$. 
   Puisque $R = 1 > 0$, il y a donc quatre possibilités. 
   Pour déterminer l'intervale de convergence, il faut faut étudier 
   la convergence aux points limites, c'est-à-dire $x = -1$ et 
   $x = 1$. 
   \vspace{1em}\\
   Pour $x = 1$, $\sum_{n=2}^{\infty }\frac{x^{3n}}{n(\ln n)^2} = 
     \sum_{n=2}^{\infty }\frac{1}{n(\ln n)^2}$. On peut utiliser le 
     test de l'intégrale : 

     \begin{align*}
        \int_{x=2}^{\infty }f(x)dx = \int_{x=2}^{\infty } 
        \frac{\ln x dx}{(\ln x)^2} = 
        \Eval{\Big| \frac{-1}{\ln x}  \Big|}{2}{\infty}
        = \frac{1}{\ln 2} \in \mathbb{R} 
     \end{align*}

     Selon le test de l'intégrale, la série converge à $\frac{1}{\ln 2}
     \in \mathbb{R}$, pour le cas limite $x = 1$.  
     \vspace{1em} \\ 
      Pour $x = -1$, $\sum_{n=2}^{\infty }\frac{x^{3n}}{n(\ln n)^2} = 
      \sum_{n=2}^{\infty }\frac{(-1)^{3n}}{n(\ln n)^2}$. On peut utiliser 
      le test des séries alternées avec

      \begin{align*}
       \sum_{n=2}^{\infty }\frac{(-1)^{3n}}{n(\ln n)^2} = 
       \sum_{n=2}^{\infty }\frac{(-1)^{n}}{n(\ln n)^2} = 
       \sum_{n=2}^{\infty }(-1)^n b_n, \; b_n = \frac{1}{n(\ln n)^2}
      \end{align*}
      Puisque le séries est positive est décroissante 
      $\forall \; n > m = 2$, par le test des séries alternées, 
      la séries converge. Après avoir vérifé les points limites, 
      nous pouvons conclure que $1 \in I$ et $-1 \in I$. 
      L'intervale de convergence est donc 
      $I = [-1, 1]$. 


      \paragraph{Estimation du reste}
      Calculer $s = \sum_{n=2}^{\infty } \frac{1}{n (\ln n)^2}$ à 
      $0.1$ près. 
      \vspace{1em}\\
      Soit $s_n=\sum_{k=2}^n \frac{1}{n(\ln n)^2}$. 
      La fonction $f(x)=\frac{1}{x(\ln x)^2}$ est continue, positive et décroissante 
      sur $[2, \infty)$. Ainsi, l'estimation du reste pour le test de l'intégrale 
      nous dit que :
      $$
      \int_{x=n+1}^{\infty } f(x) d x \leq R_n=s-s_n \leq \int_{x=n}^{\infty} f(x) d x .
      $$
      Par conséquent,
      $$
      \left[\frac{-1}{\ln x}\right]_{n+1}^{\infty}=\frac{1}{\ln (n+1)} 
      \leq R_n=s-s_n=\left[\frac{-1}{\ln x}\right]_n^{\infty}=\frac{1}{\ln n} .
      $$
      Sachant que le reste est borné par ces frontières, on cherche un 
      $n$ tel que $0.1 < \frac{1}{\ln n}$. On a donc : 
      \[ n = e^{\frac{1}{0.1}} =   e^{10} \]

      \paragraph{Convergence d'une série géométrique}
      Soit la série $1 + t + t^2 + t^3 + \cdot + t^n$ 
          $ = \sum_{n=0}^{\infty } t^n $ 
      quelle est l'intervalle de convergence de cette série ?
      \mbox{}\\ 
      Nous savons 
      \begin{align*}
        \lim\limits_{n \to+\infty } \left|\frac{t^{n+1}}{t^n}\right|   = |t|
      \end{align*}  
      Par le test du rapport, la série converge lorsque $|t| < 1$.  
      Nous avons donc l'intervale de convergence $I = ]-1, 1[$ ; 
      et nous avons $R = 1$  puisque  $\; \exists R > 0 : |t| < R$. 

      \paragraph{Développement en série entière}
      Trouver le développement en série entière de  
      $\frac{1}{t+5}$ et trouver son rayon de convergence.  
      Soit la série géométrique :
      \begin{align*}
        \sum_{n=}^{\infty }ar^n = \frac{1}{1 - r} 
      \end{align*}
      nous pouvons exprimer l'expression donnée de la façon suivante : 


      \begin{align*}
        \frac{1}{t+5} &= \frac{1}{\frac{1}{5}(1 + \frac{x}{5} )}  
                       = \frac{1}{\frac{1}{5}(1 - t) }  : t = -\frac{1}{5}x \\
                      &= \frac{1}{5} \sum_{n=0}^{\infty }(-\frac{1}{5})^n,
                      \;\; \forall \left|-\frac{1}{5}x \right| < 1 \\ 
                      & \implies  \frac{1}{5} \sum_{n=0}^{\infty }(-\frac{1}{5})^n
                      , \;\; \forall \;\; |x| < 5
      \end{align*}      


      \paragraph{Intervalle de convergence d'une série alternée}
      La série $\sum_{n=1}^{\infty } \frac{(-1)^n}{5^{n+1}}x^n$ est 
      convergente sur quelle intervalle ? 


      \begin{align*}
        \sum_{n=1}^{\infty } \frac{(-1)^n}{5^{n+1}}x^n 
        &= 
        \sum_{n=1}^{\infty } (-1)^nb_n, \;\; b_n = \frac{x^n}{5^{n+1}}
      \end{align*}

      Il faut donc trouver des $x$ tels que 
      $b_{n+1} \leq b_n, \;\; \forall n > m$ Nous devons donc considérer 
      les valeurs $x < 5 \in \mathbb{N}$. En évaluant la limite, nous avons :

      \begin{align*}
          \lim\limits_{n \to-\infty}b_n  &= 
          \lim\limits_{n \to+\infty }\frac{x^n}{5^{n+1}} = 
          \frac{1}{5}\lim\limits_{n \to+\infty }\left(\frac{x}{5}\right)^n  
      \end{align*}

      Qui est alors une série géométrique de raison 
      $ r = \frac{x}{5}$ Donc, 
      $\forall \;\; \left| \frac{x}{5}  \right| < 1$ ou 
      $|x| < 5$, la série converge. L'intervalle de convergence 
      est donc $] -5, 5[$. 


      \paragraph{Développement en série entière}
      Trouver une série entière pour $\sum_{n=1}^{\infty }\frac{x^{2023}}{x+5}$ 
      et calculez son rayon de convergence. 

      \begin{align*}
        \sum_{n=0}^{\infty }\frac{x^{2023}}{x+5} 
        &= 
        x^{2023} \times \sum_{n=0}^{\infty }\frac{1}{x+5} 
        = 
        x^{2023} \sum_{n=1}^{\infty } \frac{1}{5(\frac{x}{5} + 1)} \\
        &= 
        \frac{x^{2023}}{5}\sum_{n=0}^{\infty } \frac{1}{1 - t} \;\; : 
        t = - \frac{x}{5}  
        = 
       \frac{x^{2023}}{5} \sum_{n=0}^{\infty } \left( - \frac{x}{5} \right)^n \\ 
        &= \sum_{n=0}^{\infty }(-1)^n  \frac{x^{2023 + n}}{5^{n+1}} 
      \end{align*}

      Par la propriété des séries géométriques, 
      l'intervale de convergence 
      est donné par  :
      \[ I = (-5, 5) \]


      \paragraph{Série entière et rayon de convergence}
      Trouver un développement en série entière de  
      $\frac{1}{(1-x)^2}$ et  trouver son rayon de convergence.

      \mbox{}\\
      \begin{align*}
        \sum_{n=0}^{\infty }r^n =  \frac{1}{1 - x}  
          &\leftrightarrow 
              \frac{d}{dx}\left( \sum_{n=0}^{\infty }x^n\right) 
              = 
              \frac{d}{dx}\left( \frac{1}{1 - x}\right) \\
          &\implies
          \frac{1}{(1 - x)^2} = \sum_{n=1}^{\infty }nx^{n-1} \\ 
          &= 
          \sum_{u=0}^{\infty }(u+1)x^{u}
              \;\ : u = n - 1
      \end{align*}    

      La série converge pour $|u + 1| < 1$ ou $|(n -1) + 1| < 1$, 
      c'est-à-dire $|n| < 1$. Le rayon de convergence est donc 
      1. 


      \paragraph{Série dew Taylor et MacLaurin}
      Trouver la série de McLaudrin et son rayon de convergence 
      \begin{align*}
          \frac{1}{1 - x} 
      \end{align*}

      Pour la fonction $f(x) = (1 - x)^{-1}$, 
      nous avons les dérivées suivantes et les 
      évaluations suivantes : 
      \begin{align*}
        f^{\prime}(x) &= \frac{1}{(1 - x)^2}   
              \;\; f^{\prime}(a) = 1 \\ 
        f^{\prime\prime}(x) &=  \frac{2}{(1 - x)^3}  
          \;\; f^{\prime\prime}(a) = 2 \\
        f^{\prime\prime\prime}(x) &=  \frac{3 \times 2}{(1 - x)^4} 
          \;\; f^{\prime\prime\prime}(a) = 6 \\
        f^{\prime\prime\prime\prime}(x) &=  
              \frac{4 \times 3 \times 2}{(1 - x)^5} 
                  \;\; f^{\prime\prime\prime}(a) = 24 \\
        f^{n}(a) &= f^{n}(0) = n! \;\; \forall \; n \in \mathbb{N}
      \end{align*}

      Nous avons alors :

      \begin{align*}
        \frac{1}{1 - x} &= \sum_{n=0}^{\infty }c_n(x)^n 
                         = \sum_{n=0}^{\infty }\frac{f^{n}(0)}{n!}x^n 
                         = \sum_{n=0}^{\infty }\frac{n!}{n!}x^n 
                         = \sum_{n=0}^{\infty }x^n 
      \end{align*}  

      La série de MacLaurin que nous venons de dérivé est une 
      série géométrique de raison $x$ qui converge 
      tant que les valeurs de $x$ sont telles que 
      $|x| < 1$. Le rayon de convergence est donc $R = 1$ et 
      la série converge dans l'intervalle $(-1, 1)$.

      \paragraph{Série de Taylor et MacLaurin}
      Trouver la série de McLaudrin et son rayon de convergence 
      de la série $e^x$. 

      \mbox{}\\ 
      Pour la fonction $f(x) = e^x$, nous avons les dérivés et 
      évaluations suivantes : 

      \begin{align*}
        f^{\prime}(x) &=  f^{\prime\prime}(x) = f^{n}(x), \;\; 
            \forall \;\; n \in \mathbb{N} \\
        f^{\prime}(a) &= e^{a} = e^{0} = 1, 
            \;\;\forall \;\; n \in \mathbb{N}
      \end{align*}

      Ainsi, selon la formule de Taylor, nous avons  : 

      \begin{align*}
        e^{x} &= f(0) + \frac{f^{\prime}(a)}{1!}(x - a) 
                   + \frac{f^{\prime\prime}(a)}{2!}(x - a)^2  + \cdots \\  
        e^x &= 1    + \frac{x}{1!} + \frac{x^2}{2!} 
                   + \frac{x^3}{3!} +\cdots  \\  
        e^x &= \sum_{n=0}^{\infty }\frac{x^n}{n!} 
      \end{align*}  

      EN appliquant le test de d'Alembert, nous avons : 

      \begin{align*}
        \lim\limits_{n\to+\infty }\left| \frac{a_{n+1}}{a_n}  \right| &= 
              \lim\limits_{n \to+\infty } 
              \frac{x^{n+1}}{(n+1)!} \frac{n!}{x^n}  
         = 
        \lim\limits_{n \to+\infty } \frac{x}{(n+1)} \\
        &= 0 = L < 1 
      \end{align*} 
      Par le critère d'Alembert $\forall \;\; x \in \mathbb{R}$, la série 
      converge. Le rayon de convergence est donc $R = \infty$


      \paragraph{Série de Taylor et MacLaurin}
      Trouver la série de McLaudrin et son rayon de convergence 
      de la série $\sin x$
      
      \mbox{}\\
      Pour la fonction $f(x) = \sin x $, nous avons les dérivées et 
      évaluations suivantes : 

      \begin{align*}
        &f^{\prime}(x) = \cos x \quad f^{\prime}(0) = 1 \\ 
        &f^{\prime\prime}(x) = -\sin x \quad f^{\prime\prime}(0) = 0 \\ 
        &f^{\prime\prime\prime}(x) -\cos x \quad 
            f^{\prime\prime\prime}(0) = -1  \\ 
        &f^{4}(x) = sin(x) \quad f^{4}(0) = 0 
      \end{align*}

      Nous avons alors 

      \begin{align*}
          \sin x &= \sum_{n=0}^{\infty } 0 + \frac{x}{1!} + \frac{0}{2!}  
          + \frac{-x^3}{3!} + \frac{0}{4!} + \frac{x^5}{5!} + \cdots   
      \end{align*}

      De façon générale, tous les termes pairs sont tels que  
      $f^{2n}(x) = 0 $ et les termes impairs sont tels que  
      $f^{2n+1}(x) = (-1)^n$. Nous pouvons alors généraliser : 

      \[ f^{2n+1}(x) = (-1)^n \]

      \begin{align*}
        \sin x = \sum_{n=0}^{\infty }\frac{f^{n}(x)}{n!}x^n  
        = \sum_{k=0}^{\infty } \frac{f^{2k+1}}{(2k+1)!}x^{kn+1}  
        = \sum_{k=0}^{\infty } \frac{(-1)^k}{(2k+1)!}x^{2k+1} 
      \end{align*}  

      Il s'agit d'une série alternée que nous pouvons évaluer avec 
      le test du rapport : 

      \begin{align*}
        \lim\limits_{k \to+\infty } \left| 
        \frac{(-1)^{k+1}x^{2k+2}}{(2k +2)(2k+ 1)!} \right|
        \left| \frac{(2k + 1)!}{(-1)^k x^{2k + 1}} \right| &=  
        \lim\limits_{k \to+\infty }  = \frac{x}{2k + 2} \\ 
        &= 
        0 = L < 1, \;\; \forall \;\; x \in \mathbb{R}
      \end{align*} 
      \noindent
      Ainsi, par le test d'Alembert, la série converge 
      pour tout $x$ et le rayon de convergence est donc 
      $R = \infty$ . 


      \paragraph{Série de Taylor et MacLaurin}
      Trouver la série de McLaudrin et son rayon de convergence 
      de la série $\cos x$
      
      \mbox{}\\
      Pour la fonction $f(x) = \cos x $, nous avons les dérivées et 
      évaluations suivantes : 

      \begin{align*}
        &f^{\prime}(x) = -\sin x \quad f^{\prime}(0) = 0 \\ 
        &f^{\prime\prime}(x) = -\cos x \quad f^{\prime\prime}(0) = -1 \\ 
        &f^{\prime\prime\prime}(x) = \sin x \quad 
            f^{\prime\prime\prime}(0) = 0  \\ 
        &f^{4}(x) = \cos x \quad f^{4}(0) = 1 
      \end{align*} 

      Nous observons alors que de façon générale, 
      
      $f^{2n+1}(0) = 0$ et $f^{2n}(0) = (-1)^{n}$

      \begin{align*}
          \cos x = \sum_{n=0}^{\infty }\frac{f^{n}(0)}{n!}x^n  
          = 
          \sum_{k=0}^{\infty }\frac{f^{2k}(0)}{k!}x^{k} 
          = 
          \sum_{k=0}^{\infty }\frac{(-1)^k}{2k!}x^{2k} 
      \end{align*}

      Le test du rapport nous permet d'évaluer la convergence : 


      \begin{align*}
          \lim\limits_{k \to+\infty }  
              \left|
                  \frac{x^{2k}x}{(2k + 1)2k!} 
              \right|
              \left|
                  \frac{2k!}{x^{2k}} 
              \right|
          &= 
          \lim\limits_{k \to+\infty }\frac{x}{(2k + 1)}   
          \\ 
          &= 0 = L < 1 \;\; \forall \;\; x \in \mathbb{R}
      \end{align*}      

      Ainsi, par le test d'Alembert, la série converge 
      pour tout $x$ et le rayon de convergence est donc 
      $R = \infty$ . 


      \paragraph{Série de Taylor et MacLaurin}
      Trouver la série de McLaudrin et son rayon de convergence 
      de la série $(1 + x)^k$
      
      \mbox{}\\
      Pour la fonction $f(x) = (1 + x)^k $, nous avons les dérivées et 
      évaluations suivantes : 


      \begin{align*}
        f^{\prime}(x) = k(1 + x)^{k - 1}  
            \quad %\quad 
        f^{\prime}(0) = k 
            \\ 
        f^{\prime\prime}(x) = (k - 1)k(1 + x)^{k - 2}
            \quad %quad 
        f^{\prime\prime}(x) = k(k - 1)
        \\ 
        f^{\prime\prime\prime}(x) = (k - 2)(k - 1)k(1 + x)^{k - 3}
          \quad %quad 
        f^{\prime\prime\prime}(0) = k(k - 1)(k - 2)
        \\
        \vdots 
        \\
        f^{n}(x) = k(k - 1)(k - 2)\cdots(k - n + 1)
        \\ 
        = \frac{k!}{(k - n)!}  = {k \choose n}
      \end{align*}


      Nous avons alors la représentation suivante : 

      \begin{align*}
        (1 + x)^k &= \sum_{n=0}^{\infty }\frac{f^{(n)}(0)}{n!}x^n 
                  = \sum_{d=0}^{\infty } \frac{k!}{d!(k - d)!}x^d 
      \end{align*}

      En utilisant le test d'Alembert, nous obtenons : 

      \begin{align*}
          \lim\limits_{d \to+\infty }
              \left| 
                  \frac{a_{d+1}}{a_d} 
              \right|
          &= 
          \lim\limits_{d \to+\infty }  
          \left| 
          \frac{k!xx^d}{(d+1)d!(k - d - 1)!} 
          \right|
          \left| 
          \frac{d!(k-d)(k-d -1)!}{x^d k!} 
          \right|
          \\
          &=  
          \lim\limits_{d \to+\infty }\frac{(k - d)x}{(d + 1)} 
          = 
          1|x|
      \end{align*}    
      Ainsi, la série converge pour toutes valeur de $x$ telle que 
      $x < 1$ et le rayon de convergence est $R = 1$ . 



      \paragraph{Série de Taylor et MacLaurin}
      Trouver la série de McLaudrin et son rayon de convergence 
      de la série $\ln(1 + x)$
      
      \mbox{}\\
      Pour la fonction $f(x) = \ln(1 + x)$, nous avons les dérivées et 
      évaluations suivantes : 

      \begin{align*}
        f^{\prime}(x) = \frac{1}{1 + x}   
            \quad f^{\prime}(0) = 1 \quad 
        f^{\prime\prime}(x) =  \frac{-2}{(1 + x)^2} 
            \quad f^{\prime\prime}(0) =  -1 
        \\ 
        f^{\prime\prime\prime}(x) = \frac{2}{(1 + x)^3} 
            \quad f^{\prime\prime\prime}(0) = 2 \quad
        f^{4}(x) = \frac{-2 \cdot 3}{(1 + x)^4} 
            \quad f^{4}(0) = -6 
            \\
            \cdots 
            \\
            f^{n}(x) = \frac{(-1)^{n+1}n!}{(1 + x)^n} 
            \\ 
            f^{n}(0) = (-1)^{n+1}n!
      \end{align*}

      Nous avons alors la série de McLaurin suivante : 
      

      \begin{align*}
        \ln(x + 1) &= \sum_{n=0}^{\infty }\frac{f^{n}(0)}{n!}x^n 
          = 
          \sum_{n=0}^{\infty }\frac{(-1)^{n+1}(n-1)!}{n(n-1)!}x^{n+1}          
          \\ 
          &= 
          \sum_{n=0}^{\infty }\frac{(-1)^{n+1}x^{n+1}}{n} 
      \end{align*}

      En appliquant le test du rapport, on obtient :

      \begin{align*}
          \lim\limits_{n \to+\infty } 
              \left| 
              \frac{a_{n+1}}{a_n} 
              \right|
          &= 
          \lim\limits_{n \to+\infty }       
          \left| 
          \frac{x^{n+1}x}{n+1} 
          \right| 
          \left|
          \frac{n}{x^{n+1}} 
          \right|
          = 
          \lim\limits_{n \to+\infty }  
          \left| 
          \frac{nx}{n+1} 
          \right|
          \\ 
          &= 
          |x| 
      \end{align*}  

      Donc, la série converge pour tout $|x| < 1$ et et rayon de 
      convergence de la série est $R = 1$ . 

      \paragraph{Inégalité de Taylor}
      Montrer que $\ln(1 + x)$ est égale à sa série de 
      McLaurin pour $|x| < 1$. 

      \begin{align*}
          \ln(1 + x) = \sum_{n=0}^{\infty }\frac{(-1)^n(n-1)}{n(n-1)!} 
                      x^{n + 1}
      \end{align*}      

      Soit la $n$-ième dérivée de $f$, 
      $f^{n}(x) = \frac{(-1)^{n+1}n!}{(1 + x)^n}$





















  








 









\end{multicols*}
\end{document}
