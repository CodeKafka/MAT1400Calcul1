\documentclass[2pt]{report}
\usepackage[french]{babel}

% Permet d'ajuster la taille des marges et de la distance pour les footer
\usepackage[tmargin=2cm,rmargin=0.4in,lmargin=0.4in,bmargin=2cm,footskip=.2in]{geometry}

% Permet d'optimiser l'affichage de différents symboles et formules mathématiques
\usepackage{amsmath,amsfonts,amsthm,amssymb,mathtools}

\usepackage{svg}
% Modifie l'apparence des nombre en mathmode et textmode
%\usepackage[varbb]{newpxmath}

% Modifier l'apparence des fractions
\usepackage{xfrac}

            %%%%%%%%%%%%%%%%%  Sect.        14 Oct 2024     %%%%%%%%%%%%%%%%%%%%%%%%%%%%%%%%%%%%%%%%%%%%%%%%%%%%%%%%%%%
\usepackage{graphicx}
\usepackage{caption}
\usepackage{subcaption}
\usepackage{arydshln}
            %%%%%%%%%%%%%%%%%  Sect.        14 Oct 2024     %%%%%%%%%%%%%%%%%%%%%%%%%%%%%%%%%%%%%%%%%%%%%%%%%%%%%%%%%%%
\usepackage{balance}
\usepackage{dirtree}
\usepackage{titlesec}






% Permet de rayer (barrer) l'argument avec la touche
% \cancel{} \bcancel{} ou \xcancel{}
\usepackage[makeroom]{cancel}

% Extension du package amsmath; corrige certains bugs et déficiences de son prédecesseur
\usepackage{mathtools}

% This package provides most of the flexibility you may want to customize the three basic list
% environments (enumerate, itemize and description)
\usepackage{bookmark} 

% Réorganiser les théorèmes et Lemmes. Usage complexe. 
% Référence : https://ctan.math.illinois.edu/macros/latex/contrib/theoremref/theoremref-doc.pdf
\hypersetup{hidelinks}
\usepackage{hyperref,theoremref} 

% Fournit un environnement pour créer des boîtes colorées
\usepackage[most,many,breakable]{tcolorbox}


%\newcommand\mycommfont[1]{\footnotesize\ttfamily\textcolor{blue}{#1}}\SetCommentSty{mycommfont}

%\newcommand{\incfig}[1]{%\def\svgwidth{\columnwidth}\import{./figures/}{#1.pdf_tex}}
\newcommand{\arc}[1]{\wideparen{#1}}

%Pour colorer les lignes séparatrices de tableaux
\usepackage{colortbl}
\usepackage{tikzsymbols}

\usepackage{framed}
\usepackage{titletoc}
\usepackage{etoolbox}
\usepackage{lmodern}
\usepackage{tabularx}
\usepackage{enumitem}
\usepackage{amsthm}
            %%%%%%%%%%%%%%%%%  Sect.        14 Oct 2024     %%%%%%%%%%%%%%%%%%%%%%%%%%%%%%%%%%%%%%%%%%%%%%%%%%%%%%%%%%%

\usepackage{lipsum}
\usepackage{titling}
\renewcommand\maketitlehooka{\null\mbox{}\vfill}
\renewcommand\maketitlehookd{\vfill\null}

\newcommand{\varitem}[3][black]{%
  \item[%
   \colorbox{#2}{\textcolor{#1}{\makebox(5.5,7){#3}}}%
  ]
}
\usepackage{afterpage}
\newcommand\myemptypage{
    \null
    \thispagestyle{empty}
    \addtocounter{page}{-1}
    \newpage
    }




% from https://tex.stackexchange.com/a/167024/121799
\newcommand{\ClaudioList}{red,DarkOrange1,Goldenrod1,Green3,blue!50!cyan,DarkOrchid2}
\newcommand{\SebastianoItem}[1]{\foreach \X[count=\Y] in \ClaudioList
{\ifnum\Y=#1\relax
\xdef\SebastianoColor{\X}
\fi
}
\tikz[baseline=(SebastianoItem.base),remember
picture]{%
\node[fill=\SebastianoColor,inner sep=4pt,font=\sffamily,fill opacity=0.5] (SebastianoItem){#1)};}
}
\newcommand{\SebastianoHighlight}{\tikz[overlay,remember picture]{%
\fill[\SebastianoColor,fill opacity=0.5] ([yshift=4pt,xshift=-\pgflinewidth]SebastianoItem.east) -- ++(4pt,-4pt)
-- ++(-4pt,-4pt) -- cycle;
}}   
            %%%%%%%%%%%%%%%%%  Sect.        14 Oct 2024     %%%%%%%%%%%%%%%%%%%%%%%%%%%%%%%%%%%%%%%%%%%%%%%%%%%%%%%%%%%





%====================================================================

%====================================================================
\newcommand*{\authorimg}[1]%
    { \raisebox{-1\baselineskip}{\includegraphics[width=\imagesize]{#1}}}
\newlength\imagesize  

\usepackage{pgfplots}
\pgfplotsset{compat=1.17}

%==========================================================================================
\usepackage{libris} 
\usepackage{etoolbox}
\usepackage[export]{adjustbox}% for positioning figures

\makeatletter
% Force le chapitre suivant sur la ligne succedant la fin du 
% chapitre précédent
\patchcmd{\chapter}{\if@openright\cleardoublepage\else\clearpage\fi}{}{}{}
\makeatother
\usepackage[Sonny]{fncychap}


%boîte de couleur grise
\tcbset{
  graybox/.style={
    colback=gray!20,
    colframe=black,
    sharp corners=downhill,
    boxrule=1pt,
    left=5pt,
    right=5pt,
    top=5pt,
    bottom=5pt,
    boxsep=0pt,
	 % <-- add four values for each corner
  }
}
\newtcolorbox{graybox}{graybox}

%==========================================================================================



\usepackage{xcolor}
\usepackage{varwidth}
\usepackage{varwidth}
\usepackage{etoolbox}
%\usepackage{authblk}
\usepackage{nameref}
\usepackage{multicol,array}
\usepackage{tikz-cd}
\usepackage[ruled,linesnumbered,ruled]{algorithm2e}
\usepackage{comment} % enables the use of multi-line comments (\ifx \fi) 
\usepackage{import}
\usepackage{xifthen}
\usepackage{pdfpages}
\usepackage{transparent}


%\usepackage[french]{babel}
\usepackage{listings} % pour écrire du code dans un environnement
\lstset{
  basicstyle=\ttfamily,
  columns=fullflexible,
  keepspaces=true
}
\usepackage{caption}
\usepackage{float} % Pour forcer les images au bon endroit



\usepackage[T1]{fontenc}
\usepackage{csquotes}
%%%%%%%%%%%%%%%%%%%%%%%%%%%%%%%%%%%%%%%%%%%%%%%%%%%%%%%%%%%%%%%%%%%%%%%%%%%%%%%%%%%%%%%%%%%%%%%%%
%									ENSEMBLE DE COULEURS
%%%%%%%%%%%%%%%%%%%%%%%%%%%%%%%%%%%%%%%%%%%%%%%%%%%%%%%%%%%%%%%%%%%%%%%%%%%%%%%%%%%%%%%%%%%%%%%%%

\definecolor{myg}{RGB}{56, 140, 70}
\definecolor{myb}{RGB}{45, 111, 177}

\definecolor{mygbg}{RGB}{235, 253, 241}


\definecolor{myr}{RGB}{199, 68, 64}
\definecolor{mytheorembg}{HTML}{F2F2F9}
\definecolor{mytheoremfr}{HTML}{00007B}
\definecolor{mylenmabg}{HTML}{FFFAF8}
\definecolor{mylenmafr}{HTML}{983b0f}
\definecolor{mypropbg}{HTML}{f2fbfc}
\definecolor{mypropfr}{HTML}{191971}
\definecolor{myexamplebg}{HTML}{F2FBF8}
\definecolor{myexamplefr}{HTML}{88D6D1}
\definecolor{myexampleti}{HTML}{2A7F7F}
\definecolor{mydefinitbg}{HTML}{E5E5FF}
\definecolor{mydefinitfr}{HTML}{3F3FA3}
\definecolor{notesgreen}{RGB}{0,162,0}
\definecolor{myp}{RGB}{197, 92, 212}
\definecolor{mygr}{HTML}{2C3338}
\definecolor{myred}{RGB}{127,0,0}
\definecolor{myyellow}{RGB}{169,121,69}
\definecolor{myexercisebg}{HTML}{F2FBF8}
\definecolor{myexercisefg}{HTML}{88D6D1}
\definecolor{myred}{RGB}{127,0,0}
\definecolor{myyellow}{RGB}{169,121,69}
\definecolor{LightLavender}{HTML}{DFC5FE}

\definecolor{blue}{HTML}{008ED7}
\definecolor{mygray}{gray}{0.75}
\definecolor{lightBlue}{RGB}{235, 245, 255}
\definecolor{tcbcolred}{RGB}{255,0,0}
\definecolor{myGreen}{HTML}{009900}

% command to circle a text
\newtcbox{\entoure}[1][red]{on line,
	arc=3pt,colback=#1!10!white,colframe=#1!50!black,
	before upper={\rule[-3pt]{0pt}{10pt}},boxrule=1pt,
	boxsep=0pt,left=2pt,right=2pt,top=1pt,bottom=.5pt}
% command for the circle for the number of part entries
\newcommand\Circle[1]{\tikz[overlay,remember picture]
	\node[draw,circle, text width=18pt,line width=1pt] {#1};}

\newtcbox{\entouree}[1][red]{on line,
	arc=3pt,colback=#1!10!white,colframe=#1!50!white,
	before upper={\rule[-3pt]{0pt}{10pt}},boxrule=1pt,
	boxsep=0pt,left=2pt,right=2pt,top=1pt,bottom=.5pt}

\newcommand{\shellcmd}[1]{\\\indent\indent\texttt{\footnotesize\# #1}\\}

%=====================================================================

\patchcmd{\tableofcontents}{\contentsname}{\rmfamily\contentsname}{}{}
% patching of \@part to typeset the part number inside a framed box in its own line
% and to add color
\makeatletter
\patchcmd{\@part}
  {\addcontentsline{toc}{part}{\thepart\hspace{1em}#1}}
  {\addtocontents{toc}{\protect\addvspace{20pt}}
    \addcontentsline{toc}{part}{\huge{\protect\color{myyellow}%
      \setlength\fboxrule{2pt}\protect\Circle{%
        \hfil\thepart\hfil%
      }%
    }\\[2ex]\color{myred}\rmfamily#1}}{}{}

%\patchcmd{\@part}
%  {\addcontentsline{toc}{part}{\thepart\hspace{1em}#1}}
%  {\addtocontents{toc}{\protect\addvspace{20pt}}
%    \addcontentsline{toc}{part}{\huge{\protect\color{myyellow}%
%      \setlength\fboxrule{2pt}\protect\fbox{\protect\parbox[c][1em][c]{1.5em}{%
%        \hfil\thepart\hfil%
%      }}%
%    }\\[2ex]\color{myred}\sffamily#1}}{}{}
\makeatother
% this is the environment used to typeset the chapter entries in the ToC
% it is a modification of the leftbar environment of the framed package
\renewenvironment{leftbar}
  {\def\FrameCommand{\hspace{6em}%
    {\color{myyellow}\vrule width 2pt depth 6pt}\hspace{1em}}%
    \MakeFramed{\parshape 1 0cm \dimexpr\textwidth-6em\relax\FrameRestore}\vskip2pt%
  }
 {\endMakeFramed}

% using titletoc we redefine the ToC entries for parts, chapters, sections, and subsections
\titlecontents{part}
  [0em]{\centering}
  {\contentslabel}
  {}{}
\titlecontents{chapter}
  [0em]{\vspace*{2\baselineskip}}
  {\parbox{4.5em}{%
    \hfill\Huge\rmfamily\bfseries\color{myred}\thecontentspage}%
   \vspace*{-2.3\baselineskip}\leftbar\textsc{\small\chaptername~\thecontentslabel}\\\rmfamily}
  {}{\endleftbar}
\titlecontents{section}
  [8.4em]
  {\rmfamily\contentslabel{3em}}{}{}
  {\hspace{0.5em}\nobreak\color{myred}\normalfont\contentspage}
\titlecontents{subsection}
  [8.4em]
  {\rmfamily\contentslabel{3em}}{}{}  
  {\hspace{0.5em}\nobreak\color{myred}\contentspage}


\tcbset{
  tbcsetLavender/.style={
    on line, 
    boxsep=4pt, left=0pt,right=0pt,top=0pt,bottom=0pt,
    colframe=white, colback=LightLavender,  
    highlight math style={enhanced}
  }
}
\tcbset{
  grayb/.style={
    on line, 
    boxsep=4pt, left=0pt,right=0pt,top=0pt,bottom=0pt,
    colframe=white, colback=gray!30,  
    highlight math style={enhanced}
  }
}


%==========================================================================

%PYTHON LSTLISTING STYLE

% Define colors
\definecolor{Pgruvbox-bg}{HTML}{282828}
\definecolor{Pgruvbox-fg}{HTML}{ebdbb2}
\definecolor{Pgruvbox-red}{HTML}{fb4934}
\definecolor{Pgruvbox-green}{HTML}{b8bb26}
\definecolor{Pgruvbox-yellow}{HTML}{fabd2f}
\definecolor{Pgruvbox-blue}{HTML}{83a598}
\definecolor{Pgruvbox-purple}{HTML}{d3869b}
\definecolor{Pgruvbox-aqua}{HTML}{8ec07c}

% Define Python style
\lstdefinestyle{PythonGruvbox}{
	language=Python,
	identifierstyle=\color{lst-fg},
	basicstyle=\ttfamily\color{Pgruvbox-fg},
	keywordstyle=\color{Pgruvbox-yellow},
	keywordstyle=[2]\color{Pgruvbox-blue},
	stringstyle=\color{Pgruvbox-green},
	commentstyle=\color{Pgruvbox-aqua},
	backgroundcolor=\color{Pgruvbox-bg},
	%frame=tb,
	rulecolor=\color{Pgruvbox-fg},
	showstringspaces=false,
	keepspaces=true,
	captionpos=b,
	breaklines=true,
	tabsize=4,
	showspaces=false,
	numbers=left,
	numbersep=5pt,
	numberstyle=\tiny\color{gray},
	showtabs=false,
	columns=fullflexible,
	morekeywords={True,False,None},
	morekeywords=[2]{and,as,assert,break,class,continue,def,del,elif,else,except,exec,finally,for,from,global,if,import,in,is,lambda,nonlocal,not,or,pass,print,raise,return,try,while,with,yield},
	morecomment=[s]{"""}{"""},
	morecomment=[s]{'''}{'''},
	morecomment=[l]{\#},
	morestring=[b]",
	morestring=[b]',
	literate=
	{0}{{\textcolor{Pgruvbox-purple}{0}}}{1}
	{1}{{\textcolor{Pgruvbox-purple}{1}}}{1}
	{2}{{\textcolor{Pgruvbox-purple}{2}}}{1}
	{3}{{\textcolor{Pgruvbox-purple}{3}}}{1}
	{4}{{\textcolor{Pgruvbox-purple}{4}}}{1}
	{5}{{\textcolor{Pgruvbox-purple}{5}}}{1}
	{6}{{\textcolor{Pgruvbox-purple}{6}}}{1}
	{7}{{\textcolor{Pgruvbox-purple}{7}}}{1}
	{8}{{\textcolor{Pgruvbox-purple}{8}}}{1}
	{9}{{\textcolor{Pgruvbox-purple}{9}}}{1}
}
%====================================================================
% 
%====================================================================

% JAVA LSTLISTING STYLE IN Gruvbox Colorscheme
\definecolor{gruvbox-bg}{rgb}{0.282, 0.247, 0.204}
\definecolor{gruvbox-fg1}{rgb}{0.949, 0.898, 0.776}
\definecolor{gruvbox-fg2}{rgb}{0.871, 0.804, 0.671}
\definecolor{gruvbox-red}{rgb}{0.788, 0.255, 0.259}
\definecolor{gruvbox-green}{rgb}{0.518, 0.604, 0.239}
\definecolor{gruvbox-yellow}{rgb}{0.914, 0.808, 0.427}
\definecolor{gruvbox-blue}{rgb}{0.353, 0.510, 0.784}
\definecolor{gruvbox-purple}{rgb}{0.576, 0.412, 0.659}
\definecolor{gruvbox-aqua}{rgb}{0.459, 0.631, 0.737}
\definecolor{gruvbox-gray}{rgb}{0.518, 0.494, 0.471}

\definecolor{lst-bg}{RGB}{45, 45, 45}
\definecolor{lst-fg}{RGB}{220, 220, 204}
\definecolor{lst-keyword}{RGB}{215, 186, 125}
\definecolor{lst-comment}{RGB}{117, 113, 94}
\definecolor{lst-string}{RGB}{163, 190, 140}
\definecolor{lst-number}{RGB}{181, 206, 168}
\definecolor{lst-type}{RGB}{218, 142, 130}


\lstdefinestyle{JavaGruvbox}{
	language=Java,
	basicstyle=\ttfamily\color{lst-fg},
	keywordstyle=\color{lst-keyword},
	keywordstyle=[2]\color{lst-type},
	commentstyle=\itshape\color{lst-comment},
	stringstyle=\color{lst-string},
	numberstyle=\color{lst-number},
	backgroundcolor=\color{lst-bg},
	%frame=tb,
	rulecolor=\color{gruvbox-aqua},
	showstringspaces=false,
	keepspaces=true,
	captionpos=b,
	breaklines=true,
	tabsize=4,
	showspaces=false,
	showtabs=false,
	columns=fullflexible,
	morekeywords={var},
	morekeywords=[2]{boolean, byte, char, double, float, int, long, short, void},
	morecomment=[s]{/}{/},
	morecomment=[l]{//},
	morestring=[b]",
	morestring=[b]',
	numbers=left,
	numbersep=5pt,
	numberstyle=\tiny\color{gray},
}



%====================================================================
% 
%====================================================================


% Define Dracula color scheme for Java
\definecolor{draculawhite-background}{RGB}{237, 239, 252}
\definecolor{draculawhite-comment}{RGB}{98, 114, 164}
\definecolor{draculawhite-keyword}{RGB}{189, 147, 249}
\definecolor{draculawhite-string}{RGB}{152, 195, 121}
\definecolor{draculawhite-number}{RGB}{249, 189, 89}
\definecolor{draculawhite-operator}{RGB}{248, 248, 242}

% Define JavaDraculaWhite lstlisting environment
\lstdefinestyle{JavaDraculaWhite}{
    language=Java,
    backgroundcolor=\color{draculawhite-background},
    commentstyle=\itshape\color{draculawhite-comment},
    keywordstyle=\color{draculawhite-keyword},
    stringstyle=\color{draculawhite-string},
    basicstyle=\ttfamily\footnotesize\color{black},
    identifierstyle=\color{black},
    keywordstyle=\color{draculawhite-keyword}\bfseries,
    morecomment=[s][\color{draculawhite-comment}]{/**}{*/},
    showstringspaces=false,
    showspaces=false,
    breaklines=true,
    frame=single,
    rulecolor=\color{draculawhite-operator},
    tabsize=2,  
	numbers=left,
	numbersep=4pt,
	numberstyle=\ttfamily\tiny\color{gray}
}
%====================================================================
% 
%====================================================================
% Define PythonDraculaWhite lstlisting environment 
\definecolor{draculawhite-bg}{HTML}{FAFAFA}
\definecolor{draculawhite-fg}{HTML}{282A36}
\definecolor{pdraculawhite-keyword}{HTML}{BD93F9}

\definecolor{pdraculawhite-comment}{HTML}{6272A4}
\definecolor{draculawhite-number}{HTML}{FF79C6}


\lstdefinestyle{PythonDraculaWhite}{
    language=Python,
    basicstyle=\ttfamily\small\color{draculawhite-fg},
    backgroundcolor=\color{draculawhite-background},
    keywordstyle=\color{orange}\bfseries,
    stringstyle=\color{draculawhite-string},
    commentstyle=\color{pdraculawhite-comment}\itshape,
    numberstyle=\color{draculawhite-number},
    showstringspaces=false,
	showspaces=false,
    breaklines=true,
	frame=single,
	rulecolor=\color{draculawhite-operator}, 
    tabsize=4,
    morekeywords={as,with,1,2,3,4, 5,6,7,8,9,True,False},
    %escapeinside={(*@}{@*)},
    numbers=left,
    numbersep=5pt,
    %xleftmargin=15pt,
    %framexleftmargin=15pt,
    %framexrightmargin=0pt,
    %framexbottommargin=0pt,
    %framextopmargin=0pt,
    %rulecolor=\color{draculawhite-fg},
    %frame=tb,
    %aboveskip=0pt,
    %belowskip=0pt,
    %captionpos=b,
	numberstyle=\ttfamily\tiny\color{gray} 
}
%====================================================================
% 
%====================================================================

% Define colors for HTML langage
\definecolor{html-orange}{HTML}{FF5733}
\definecolor{html-yellow}{HTML}{F0E130}
\definecolor{html-green}{HTML}{50FA7B}
\definecolor{html-blue}{HTML}{5AFBFF}
\definecolor{html-purple}{HTML}{BD93F9}
\definecolor{html-pink}{HTML}{FF80BF}
\definecolor{html-gray}{HTML}{6272A4}
\definecolor{html-white}{HTML}{F8F8F2}

% Defines a new HTML5 langage that extend on the html langange
\lstdefinestyle{HTMLDraculaWhite}{
  language=HTML,
  backgroundcolor=\color{html-white},
  basicstyle=\ttfamily\color{html-gray},
  keywordstyle=\color{html-blue},
  stringstyle=\color{html-orange},
  commentstyle=\color{html-green},
  tagstyle=\color{html-yellow},
  moredelim=[s][\color{html-pink}]{<!--}{-->},
  moredelim=[s][\color{html-purple}]{\{}{\}},
  showstringspaces=false,
  tabsize=2,
  breaklines=true,
  columns=fullflexible,
  %frame=single,
  framexleftmargin=5mm,
  xleftmargin=10mm,
  numbers=left,
  numberstyle=\tiny\color{html-gray},
  escapeinside={<@}{@>}
}

%====================================================================
% 
%====================================================================
% Define the colors needed for the HTMLDraculaDark environment
\definecolor{htmltag}{HTML}{ff79c6}
\definecolor{htmlattr}{HTML}{f1fa8c}
\definecolor{htmlvalue}{HTML}{bd93f9}
\definecolor{htmlcomment}{HTML}{6272a4}
%\definecolor{htmltext}{HTML}{f8f8f2}
\definecolor{htmltext}{HTML}{401E31}
\definecolor{htmlbackground}{HTML}{282a36}
\definecolor{comphtmlbackground}{HTML}{8093FF}
%\definecolor{htmlbackground}{HTML}{4D5169}

% Define the HTMLDraculaDark environment
\lstdefinestyle{HTMLDraculaDark}{
    basicstyle=\bfseries\ttfamily\color{htmltext},
    commentstyle=\itshape\color{htmlcomment},
    keywordstyle=\bfseries\color{htmltag},
    stringstyle=\color{htmlvalue},
    emph={DOCTYPE,html,head,body,div,span,a,script},
    emphstyle={\color{htmltag}\bfseries},
    sensitive=true,
    showstringspaces=false,
    backgroundcolor=\color{white},
    %frame=tb,
    language=HTML,
    tabsize=4,
    breaklines=true,
    breakatwhitespace=true,
    numbers=left,
    numbersep=10pt,
    numberstyle=\small\bfseries\ttfamily\color{htmlcomment},
    escapeinside={<@}{@>},
	rulecolor=\color{htmlbackground},
	xleftmargin=20pt,
	% Add a vertical line for opening and closing tags
    %frame=lines,
    framesep=2pt,
    framexleftmargin=4pt,
    % Add a vertical line for closing tags that go through multiple lines
    breaklines=true,
    postbreak=\mbox{\textcolor{gray}{$\hookrightarrow$}\space},
    showlines=true,
	% Add a rule to apply commentstyle to keywords inside comments
    moredelim=[s][\itshape\color{htmlcomment}]{<!--}{-->},
    morekeywords={id,class,type,name,value,placeholder,checked,src,href,alt}
}




%====================================================================
% 
%====================================================================






% Crée un environnement "Theorem" numéroté en fonction du document
\tcbuselibrary{theorems,skins,hooks} 
\newtcbtheorem{Theorem}{Théorème}
{%
	enhanced,
	breakable,
	colback = mytheorembg,
	frame hidden,
	boxrule = 0sp,
	borderline west = {2pt}{0pt}{mytheoremfr},
	sharp corners,
	detach title,
	before upper = \tcbtitle\par\smallskip,
	coltitle = mytheoremfr,
	fonttitle = \bfseries\fontfamily{lmss}\selectfont,
	description font = \mdseries\fontfamily{lmss}\selectfont,
	separator sign none,
	segmentation style={solid, mytheoremfr},
}
{thm}

% Crée un environnement "Preuve" numéroté en fonction du document
\tcbuselibrary{theorems,skins,hooks}
\newtcbtheorem{Preuve}{Preuve}
{
	enhanced,
	breakable,
	colback=white,
	frame hidden,
	boxrule = 0sp,
	borderline west = {2pt}{0pt}{mytheoremfr},
	sharp corners,
	detach title,
	before upper = \tcbtitle\par\smallskip,
	coltitle = mytheoremfr,
	description font=\fontfamily{lmss}\selectfont,
	fonttitle=\fontfamily{lmss}\selectfont\bfseries,
	separator sign none,
	segmentation style={solid, mytheoremfr},
}
{th}


% Crée un environnement "Preuve" numéroté en fonction du document
\tcbuselibrary{theorems,skins,hooks}
\newtcbtheorem{Explication}{Explication}
{
	enhanced,
	breakable,
	colback=white,
	frame hidden,
	boxrule = 0sp,
	borderline west = {2pt}{0pt}{mytheoremfr},
	sharp corners,
	detach title,
	before upper = \tcbtitle\par\smallskip,
	coltitle = mytheoremfr,
	description font=\fontfamily{lmss}\selectfont,
	fonttitle=\fontfamily{lmss}\selectfont\bfseries,
	separator sign none,
	segmentation style={solid, mytheoremfr},
}
{th}




% Crée un environnement "Example" numéroté en fonction du document
\tcbuselibrary{theorems,skins,hooks}
\newtcbtheorem{Example}{Exemple.}
{
	enhanced,
	breakable,
	colback=lightBlue,
	frame hidden,
	boxrule = 0sp,
	borderline west = {2pt}{0pt}{myb},
	sharp corners,
	detach title,
	before upper = \tcbtitle\par\smallskip,
	coltitle = myb,
	description font=\fontfamily{lmss}\selectfont,
	fonttitle=\fontfamily{lmss}\selectfont\bfseries,
	separator sign none,
	segmentation style={solid, mytheoremfr},
}
{th}



% Crée un environnement "EExample" numéroté en fonction du document
\tcbuselibrary{theorems,skins,hooks}
\newtcbtheorem{EExample}{Exemple}
{
	enhanced,
	breakable,
	colback=white,
	frame hidden,
	boxrule = 0sp,
	borderline west = {2pt}{0pt}{myb},
	sharp corners,
	detach title,
	before upper = \tcbtitle\par\smallskip,
	coltitle = myb,
	description font=\mdseries\fontfamily{lmss}\selectfont,
	fonttitle=\fontfamily{lmss}\selectfont\bfseries,
	separator sign none,
	segmentation style={solid, mytheoremfr},
}
{th}



% Crée un environnement "Lemme" numéroté en fonction du document
\tcbuselibrary{theorems,skins,hooks}
\newtcbtheorem{Lemme}{Lemme}
{
	enhanced,
	breakable,
	colback=mylenmabg,
	frame hidden,
	boxrule = 0sp,
	borderline west = {2pt}{0pt}{mylenmafr},
	sharp corners,
	detach title,
	before upper = \tcbtitle\par\smallskip,
	coltitle = mylenmafr,
	description font=\mdseries\fontfamily{lmss}\selectfont,
	fonttitle=\fontfamily{lmss}\selectfont\bfseries,
	separator sign none,
	segmentation style={solid, mytheoremfr},
}
{th}


\tcbuselibrary{theorems,skins,hooks}
\newtcbtheorem{PreuveL}{Preuve.}
{
	enhanced,
	breakable,
	colback=white,
	frame hidden,
	boxrule = 0sp,
	borderline west = {2pt}{0pt}{mylenmafr},
	sharp corners,
	detach title,
	before upper = \tcbtitle\par\smallskip,
	coltitle = mylenmafr,
	description font=\fontfamily{lmss}\selectfont,
	fonttitle=\fontfamily{lmss}\selectfont\bfseries,
	separator sign none,
	segmentation style={solid, mytheoremfr},
}
{th}


\newtcbtheorem{Remarque}{Remarque}
{
	enhanced,
	breakable,
	colback=white,
	frame hidden,
	boxrule = 0sp,
	borderline west = {2pt}{0pt}{myb},
	sharp corners,
	detach title,
	before upper = \tcbtitle\par\smallskip,
	coltitle = myb,
	description font=\mdseries\fontfamily{lmss}\selectfont,
	fonttitle=\fontfamily{lmss}\selectfont\bfseries,
	separator sign none,
	segmentation style={solid, mytheoremfr},
}
{th}


\newtcbtheorem{DefG}{Définition}
{
	enhanced,
	breakable,
	colback=mygbg,
	frame hidden,
	boxrule = 0sp,
	borderline west = {2pt}{0pt}{myg},
	sharp corners,
	detach title,
	before upper = \tcbtitle\par\smallskip,
	coltitle = myg,
	description font=\mdseries\fontfamily{lmss}\selectfont,
	fonttitle=\fontfamily{lmss}\selectfont\bfseries,
	separator sign none,
	segmentation style={solid, mytheoremfr},
}
{th}



% Crée une boîte ayant la même couleur que l'environnement theorem.
\tcbuselibrary{theorems,skins,hooks}
\newtcolorbox{Theoremcon}
{%
	enhanced
	,breakable
	,colback = mytheorembg
	,frame hidden
	,boxrule = 0sp
	,borderline west = {2pt}{0pt}{mytheoremfr}
	,sharp corners
	,description font = \mdseries
	,separator sign none
}

% Crée un environnement "Definition" numéroté en fonction de la section
\newtcbtheorem[number within=chapter]{Definition}{Définition}{enhanced,
	before skip=2mm,after skip=2mm, colback=red!5,colframe=red!80!black,boxrule=0.5mm,
	attach boxed title to top left={xshift=1cm,yshift*=1mm-\tcboxedtitleheight}, varwidth boxed title*=-3cm,
	boxed title style={frame code={
			\path[fill=tcbcolback!10!red]
			([yshift=-1mm,xshift=-1mm]frame.north west)
			arc[start angle=0,end angle=180,radius=1mm]
			([yshift=-1mm,xshift=1mm]frame.north east)
			arc[start angle=180,end angle=0,radius=1mm];
			\path[left color=tcbcolback!10!myred,right color=tcbcolback!10!myred,
			middle color=tcbcolback!60!myred]
			([xshift=-2mm]frame.north west) -- ([xshift=2mm]frame.north east)
			[rounded corners=1mm]-- ([xshift=1mm,yshift=-1mm]frame.north east)
			-- (frame.south east) -- (frame.south west)
			-- ([xshift=-1mm,yshift=-1mm]frame.north west)
			[sharp corners]-- cycle;
		},interior engine=empty,
	},
	fonttitle=\bfseries,
	title={#2},#1}{def}

% Crée un environnement "definition" numéroté en fonction du Chapitre
\newtcbtheorem[number within=section]{definition}{Définition}{enhanced,
	before skip=2mm,after skip=2mm, colback=red!5,colframe=red!80!black,boxrule=0.5mm,
	attach boxed title to top left={xshift=1cm,yshift*=1mm-\tcboxedtitleheight}, varwidth boxed title*=-3cm,
	boxed title style={frame code={
			\path[fill=tcbcolback]
			([yshift=-1mm,xshift=-1mm]frame.north west)
			arc[start angle=0,end angle=180,radius=1mm]
			([yshift=-1mm,xshift=1mm]frame.north east)
			arc[start angle=180,end angle=0,radius=1mm];
			\path[left color=tcbcolback!60!black,right color=tcbcolback!60!black,
			middle color=tcbcolback!80!black]
			([xshift=-2mm]frame.north west) -- ([xshift=2mm]frame.north east)
			[rounded corners=1mm]-- ([xshift=1mm,yshift=-1mm]frame.north east)
			-- (frame.south east) -- (frame.south west)
			-- ([xshift=-1mm,yshift=-1mm]frame.north west)
			[sharp corners]-- cycle;
		},interior engine=empty,
	},
	fonttitle=\bfseries,
	title={#2},#1}{def}

\usetikzlibrary{arrows,calc,shadows.blur}
\tcbuselibrary{skins}
\newtcolorbox{note}[1][]{%
	enhanced jigsaw,
	colback=gray!20!white,%
	colframe=gray!80!black,
	size=small,
	boxrule=1pt,
	title=\textbf{Note : },
	halign title=flush center,
	coltitle=black,
	breakable,
	drop shadow=black!50!white,
	attach boxed title to top left={xshift=1cm,yshift=-\tcboxedtitleheight/2,yshifttext=-\tcboxedtitleheight/2},
	minipage boxed title=1.5cm,
	boxed title style={%
		colback=white,
		size=fbox,
		boxrule=1pt,
		boxsep=2pt,
		underlay={%
			\coordinate (dotA) at ($(interior.west) + (-0.5pt,0)$);
			\coordinate (dotB) at ($(interior.east) + (0.5pt,0)$);
			\begin{scope}
				\clip (interior.north west) rectangle ([xshift=3ex]interior.east);
				\filldraw [white, blur shadow={shadow opacity=60, shadow yshift=-.75ex}, rounded corners=2pt] (interior.north west) rectangle (interior.south east);
			\end{scope}
			\begin{scope}[gray!80!black]
				\fill (dotA) circle (2pt);
				\fill (dotB) circle (2pt);
			\end{scope}
		},
	},
	#1,
}


% Crée un environnement "qstion" 
\newtcbtheorem{qstion}{Question}{enhanced,
	breakable,
	colback=white,
	colframe=mygr,
	attach boxed title to top left={yshift*=-\tcboxedtitleheight},
	fonttitle=\bfseries,
	title={#2},
	boxed title size=title,
	boxed title style={%
		sharp corners,
		rounded corners=northwest,
		colback=tcbcolframe,
		boxrule=0pt,
	},
}{def}


% Pour créer un environnement "Liste" 

\tcbuselibrary{theorems,skins,hooks}
\newtcbtheorem[number within=section]{Liste}{Liste}
{%
	enhanced
	,breakable
	,colback = myp!10
	,frame hidden
	,boxrule = 0sp
	,borderline west = {2pt}{0pt}{myp!85!black}
	,sharp corners
	,detach title
	,before upper = \tcbtitle\par\smallskip
	,coltitle = myp!85!black
	,fonttitle = \bfseries\sffamily
	,description font = \mdseries
	,separator sign none
	,segmentation style={solid, myp!85!black}
}
{th}


\tcbuselibrary{theorems,skins,hooks}
\newtcbtheorem{Syntaxe}{Syntaxe.}
{%
	enhanced
	,breakable
	,colback = myp!10
	,frame hidden
	,boxrule = 0sp
	,borderline west = {2pt}{0pt}{myp!85!black}
	,sharp corners
	,detach title
	,before upper = \tcbtitle\par\smallskip
	,coltitle = myp!85!black
	,fonttitle = \bfseries\fontfamily{lmss}\selectfont 
	,description font = \mdseries\fontfamily{lmss}\selectfont 
	,separator sign none
	,segmentation style={solid, myp!85!black}
}
{th}



% Crée un environnement "Concept" numéroté en fonction du document
\tcbuselibrary{theorems,skins,hooks}
\newtcbtheorem{Concept}{Concept.}
{
	enhanced,
	breakable,
	colback=mylenmabg,
	frame hidden,
	boxrule = 0sp,
	borderline west = {2pt}{0pt}{mylenmafr},
	sharp corners,
	detach title,
	before upper = \tcbtitle\par\smallskip,
	coltitle = mylenmafr,
	description font=\mdseries\fontfamily{lmss}\selectfont,
	fonttitle=\fontfamily{lmss}\selectfont\bfseries,
	separator sign none,
	segmentation style={solid, mytheoremfr},
}
{th}


% Crée un environnement "codeEx" numéroté en fonction du document
\tcbuselibrary{theorems,skins,hooks}
\newtcbtheorem{codeEx}{Exemple}
{
	enhanced,
	breakable,
	colback=white,
	frame hidden,
	boxrule = 0sp,
	borderline west = {2pt}{0pt}{gruvbox-bg},
	sharp corners,
	detach title,
	before upper = \tcbtitle\par\smallskip,
	coltitle = gruvbox-bg,
	description font=\md:wqseries\fontfamily{lmss}\selectfont,
	fonttitle=\fontfamily{lmss}\selectfont\bfseries,
	separator sign none,
	segmentation style={solid, mytheoremfr},
}
{th}


% Crée un environnement "codeEx" numéroté en fonction du document
\tcbuselibrary{theorems,skins,hooks}
\newtcbtheorem{codeRem}{Remarque.}
{
	enhanced,
	breakable,
	colback=white,
	frame hidden,
	boxrule = 0sp,
	borderline west = {2pt}{0pt}{gruvbox-bg},
	sharp corners,
	detach title,
	before upper = \tcbtitle\par\smallskip,
	coltitle = gruvbox-bg,
	description font=\mdseries\fontfamily{lmss}\selectfont,
	fonttitle=\fontfamily{lmss}\selectfont\bfseries,
	separator sign none,
	segmentation style={solid, mytheoremfr},
}
{th}


\tcbuselibrary{theorems,skins,hooks}
\newtcbtheorem{Identite}{Identité.}
{
	enhanced,
	breakable,
	colback=white,
  before upper=\tcbtitle\par\Hugeskip,
	frame hidden,
	boxrule = 0sp,
	borderline west = {2pt}{0pt}{gruvbox-bg},
	sharp corners,
	detach title,
	before upper = \tcbtitle\par\smallskip,
	coltitle = gruvbox-bg,
	description font=\mdseries\fontfamily{lmss}\selectfont,
	fonttitle=\fontfamily{lmss}\selectfont\bfseries,
	fontlower=\fontfamily{cmr}\selectfont,
  separator sign none,
	segmentation style={solid, mytheoremfr},
}
{th}

\tcbuselibrary{theorems,skins,hooks}
\newtcbtheorem{Exercice}{Exercice}
{
	enhanced,
	breakable,
	colback=white,
  before upper=\tcbtitle\par\Hugeskip,
	frame hidden,
	boxrule = 0sp,
	borderline west = {2pt}{0pt}{gruvbox-green},
	sharp corners,
	detach title,
	before upper = \tcbtitle\par\smallskip,
	coltitle = gruvbox-green,
	description font=\mdseries\fontfamily{lmss}\selectfont,
	fonttitle=\fontfamily{lmss}\selectfont\bfseries,
	fontlower=\fontfamily{cmr}\selectfont,
  separator sign none,
	segmentation style={solid, mytheoremfr},
}
{th}

% Crée un environnement "Réponse" numéroté en fonction du document
\tcbuselibrary{theorems,skins,hooks}
\newtcbtheorem{Reponse}{Reponse}
{
	enhanced,
	breakable,
	colback=white,
	frame hidden,
	boxrule = 0sp,
	borderline west = {2pt}{0pt}{mytheoremfr},
	sharp corners,
	detach title,
	before upper = \tcbtitle\par\smallskip,
	coltitle = mytheoremfr,
	description font=\fontfamily{lmss}\selectfont,
	fonttitle=\fontfamily{lmss}\selectfont\bfseries,
	separator sign none,
	segmentation style={solid, mytheoremfr},
}
{th}

\newtcbtheorem{Definitionx}{Définition}
{
enhanced,
breakable,
colback=red!5,
  before upper=\tcbtitle\par\Hugeskip,
frame hidden,
boxrule = 0sp,
borderline west = {2pt}{0pt}{red!80!black},
sharp corners,
detach title,
before upper = \tcbtitle\par\smallskip,
coltitle = red!80!black,
description font=\mdseries\fontfamily{lmss}\selectfont,
fonttitle=\fontfamily{lmss}\selectfont\bfseries,
fontlower=\fontfamily{cmr}\selectfont,
  separator sign none,
segmentation style={solid, mytheoremfr},
}
{th}

\tcbuselibrary{theorems,skins,hooks}
\newtcbtheorem[number within=chapter]{prop}{Proposition}
{%
	enhanced,
	breakable,
	colback = mypropbg,
	frame hidden,
	boxrule = 0sp,
	borderline west = {2pt}{0pt}{mypropfr},
	sharp corners,
	detach title,
	before upper = \tcbtitle\par\smallskip,
	coltitle = mypropfr,
	fonttitle = \bfseries\sffamily,
	description font = \mdseries,
	separator sign none,
	segmentation style={solid, mypropfr},
}
{th}


\tcbuselibrary{theorems,skins,hooks}
\newtcbtheorem[number within=section]{Prop}{Proposition}
{%
	enhanced,
	breakable,
	colback = mypropbg,
	frame hidden,
	boxrule = 0sp,
	borderline west = {2pt}{0pt}{mypropfr},
	sharp corners,
	detach title,
	before upper = \tcbtitle\par\smallskip,
	coltitle = mypropfr,
	fonttitle = \bfseries\sffamily,
	description font = \mdseries,
	separator sign none,
	segmentation style={solid, mypropfr},
}
{th}


%================================
% Corollery
%================================
\tcbuselibrary{theorems,skins,hooks}
\newtcbtheorem[number within=section]{Corollary}{Corollary}
{%
	enhanced
	,breakable
	,colback = myp!10
	,frame hidden
	,boxrule = 0sp
	,borderline west = {2pt}{0pt}{myp!85!black}
	,sharp corners
	,detach title
	,before upper = \tcbtitle\par\smallskip
	,coltitle = myp!85!black
	,fonttitle = \bfseries\sffamily
	,description font = \mdseries
	,separator sign none
	,segmentation style={solid, myp!85!black}
}
{th}
\tcbuselibrary{theorems,skins,hooks}
\newtcbtheorem[number within=chapter]{corollary}{Corollaire}
{%
	enhanced
	,breakable
	,colback = myp!10
	,frame hidden
	,boxrule = 0sp
	,borderline west = {2pt}{0pt}{myp!85!black}
	,sharp corners
	,detach title
	,before upper = \tcbtitle\par\smallskip
	,coltitle = myp!85!black
	,fonttitle = \bfseries\sffamily
	,description font = \mdseries
	,separator sign none
	,segmentation style={solid, myp!85!black}
}
{th}



\usepackage[scr]{rsfso}


\title{\Huge{Calcul 1}\\{MATH1400}\\{\textbf{Introduction}}}
\author{\huge{Franz Girardin}}
\date{\today}
\lstset{inputencoding=utf8/latin1}

            %%%%%%%%%%%%%%%%%  Sect.                          %%%%%%%%%%%%%%%%%%%%%%%%%%%%%%%%%%%%%%%%%%%%%%%%%%%%%%%%%
\usepackage{helvet}
\titleformat{\chapter}
  {\fontfamily{phv}\bfseries\huge} % format
  {}                % label
  {0pt}             % sep
  {\color{myb}\huge}           % before-code



\titleformat{\section}
  {\normalfont\scshape}{\thesection}{1em}{}


% Customizing the spacing for the chapter titles
\titlespacing*{\chapter}{0pt}{0pt}{20pt}

\usepackage[utopia]{mathdesign}
% Allow hfill in math environment
\newcommand{\specialcell}[1]{\ifmeasuring@#1\else\omit$\displaystyle#1$\ignorespaces\fi}

% Allow you to do the non implication (implication barred)
\newcommand{\notimplies}{%
  \mathrel{{\ooalign{\hidewidth$\not\phantom{=}$\hidewidth\cr$\implies$}}}}



\DeclareRobustCommand{\looongrightarrow}{%
  \DOTSB\relbar\joinrel\relbar\joinrel\relbar\joinrel\rightarrow
}

\begin{document}
\maketitle

\pagebreak

\pagebreak
\begin{multicols*}{3}

    \paragraph{Fonction exponentielle}
    \begin{itemize}
        \item[$\rhd$]  \textbf{Domaine} : $\mathbb{R}$  
        \item[$\rhd$]  \textbf{Continuité} : Continue sur \textbf{dom} 
        \item[$\rhd$]  \textbf{Croissance}  $0 < base < 1$ : 
            Stric. $\downarrow$

        \item[$\rhd$]  \textbf{Croissance}  $base > 1$ : Stric. $\uparrow$
        \item[$\rhd$]  \textbf{Ordonnée O.} : $e^0 = 1$   
        \item[$\rhd$]  \textbf{Signe} : $\forall x \in \mathbb{R}, e^x > 0$  
        \item[$\rhd$]   $e^x : x\looongrightarrow\infty+$ :  
            $\lim\limits_{x\to+\infty}e^x  = \infty$  
            \item[$\rhd$]   $e^x : x\looongrightarrow\infty-$ :  
                $\lim\limits_{x\to-\infty}e^x  = 0$
    \end{itemize}


\paragraph{Propriétés exponentielles}  
\begin{align*}
      e^{x+y} = e^{x}e^{y} \quad |& \quad e^{xy} = (e^{x})^{y} \\
      (e^x)^{\prime} = e^x \quad | & \quad (a^x)^{\prime} = a^x\log_ea
\end{align*}


    \paragraph{Fonction exponentielle}
    \begin{itemize}
        \item[$\rhd$]  \textbf{Domaine} : $]0, \infty [$  
        \item[$\rhd$]  \textbf{Continuité} : Continue sur \textbf{dom} 
        \item[$\rhd$]  \textbf{Croissance}  $0 < base < 1$ : 
            Stric. $\downarrow$

        \item[$\rhd$]  \textbf{Croissance}  $base > 1$ : Stric. $\uparrow$
        \item[$\rhd$]  \textbf{Abscisse O.} : $\log_e(1) = 0$ 
        \item[$\rhd$]  \textbf{Signe} : $\forall x > AO, \log_ax > 0; \; 
            \forall x, 0 < x < AO, \log_ax < 0$  
        \item[$\rhd$]   $e^x : x\looongrightarrow\infty+$ :  
            $\lim\limits_{x\to+\infty}\log_ax  = \infty$  
            \item[$\rhd$]   $e^x : x\looongrightarrow\infty-$ :  
                $\lim\limits_{x\to-\infty}\log_ax  = -\infty$
    \end{itemize}



    \begin{align*}
      \log(x+y) = \log x + \log y \; \Big| & \; \log x^y = y\log x \\
      \log_a x = \dfrac{\log_a x}{\log_b x} \; \Big| & \; 
                                     (\log x)^{\prime} = \dfrac{1}{x}  
    \end{align*}

    \paragraph{Optimisation}

    \begin{itemize}
        \item[$\rhd$]  \textbf{Maximum} : point $x \in \textbf{ dom } : 
            \forall y \in f, y \neq x, f(x) \geq f(y)$     
        \item[$\rhd$]  \textbf{Minimum} : point $x \in \textbf{ dom } : 
            \forall y \in f, y \neq x, f(x) \leq f(y)$    
        \item[$\rhd$]  \textbf{Point d'inflexion}: $\uparrow - \downarrow$ 
            ou $\downarrow - \uparrow$
        \item[$\rhd$]  \textbf{Potentiel max ou min} :   
        $f^{\prime} \left( x \right) = 0$ ou 
        $f^{\prime} \left(x\right) \nexists$
    \end{itemize}

    \paragraph{Test de la dérivé première}
    Soit $f\left(x\right)$, on peut considérer $f^{\prime}(x)$ 
    pour déduire des \textbf{informations} propres à $f$.   

    \begin{table}[H]
      \caption {Test de la dérivé première pour une fonction hypothétique}

      \begin{center}
        \renewcommand{\arraystretch}{1.5}
        \fontfamily{flr}\selectfont
        \tiny
        \begin{tabular}{|l|l|l|l|l|l|l|l|l}
        \arrayrulecolor{blue}\hline
        \rowcolor{lightBlue}
        \textcolor{myb}{} & \textcolor{myb}{ $-\infty$ } & & -2 & & 1 & & 10
        \\
        \hline
        \hline
        \arrayrulecolor{black} 
        $f^{\prime} $ &  & + &  & + & $\nexists$  & - & 
        \\ 
        \hline 
        $f$ & $-\infty$  & $\nearrow$ & inflex. & 
        $\nearrow$ & max & $\searrow$ & 0 
        \\ 
        \hline
      \end{tabular}
    \end{center}
    \end{table}

    

    \begin{table}[h]
      \caption {Test de la dérivé première pour une fonction hypothétique}

      \begin{center}
        \renewcommand{\arraystretch}{1.5}
        \fontfamily{flr}\selectfont
        \footnotesize
        \begin{tabular}{|l|l|l|l|l|l|l|l|l}
        \arrayrulecolor{blue}\hline
        \rowcolor{lightBlue}
        \textcolor{myb}{} & \textcolor{myb}{ $-\infty$ } & & -2 & & 1 & & 10
        \\
        \hline
        \hline
        \arrayrulecolor{black} 
        $f^{\prime} $ &  & + &  & + & $\nexists$  & - & 
        \\ 
        \hline 
        $f$ & $-\infty$  & $\nearrow$ & inflex. & $\nearrow$ & max & $\searrow$ & 0 
        \\ 
        \hline
      \end{tabular}
    \end{center}
    \end{table}
\begin{EExample}{Interpréter un tableau de test de dérivé première}{}
  \textbf{1. Comportement à la frontière} 
  Appliquer une limite aux deux frontières de la fonction, dans ce cas-ci $x \rightarrow -\infty$ et 
  $x \rightarrow 10$. On a :
  \begin{align*}
    \lim\limits_{x\to\infty^{+}}f(x) = \infty  \textbf{ et }\lim\limits_{x\to 10}f(x) = 0        
  \end{align*}
  2. Calculer $f^{\prime}$. Trouver $x$ tels que :
  \begin{enumerate}
    \varitem{blue!40}{\textbf{1}} $f^{\prime}\left(x\right) = 0$
  \varitem{blue!40}{\textbf{2}} $f^{\prime}\left(x\right)$ n'existe pas  
  \end{enumerate}
  \textbf{Dans le contexte de l'exemple}, on a trouvé la valeur $-2$, qui correspond au moment ou 
  $f^{\prime}\left(x\right) = 0$. Et la valeur $1$ correspond au moment ou la dérivé n'existe pas. \\\\ 
  \textbf{3. Trouver le signe $f^{\prime}$} sur chacun des intervales entre nos points d'intérêts pour déterminer 
  le comportement de la fonction. \\ 
  \textbf{Entre $-\infty$ et $-2$}, la dérivé est positive; la fonction est donc \textbf{croissante} 
  sur cet interval. \\
  \textbf{Entre $-2$ et $1$}, la dérivé est positive; la fonction est donc \textbf{croissante} 
  sur cet interval. \\
  \textbf{Entre $1$ et $10$}, la dérivé est négative; la fonction est donc \textbf{décroissante} 
  sur cet interval. \\\\
Noter que pour déterminer le signe de la dérivé, il suffit d'évaluer $f^{\prime}\left(x\right)$ à n'importe 
quel endroit dans l'interval (e.g.  $f^{\prime}\left(1\right)$  pour l'intervale de $-\infty$ à $-2$)  
\end{EExample}



\section{Test de la dérivé seconde}
\begin{Concept}{Test de la dérivé seconde}{}
\textbf{Si et seulement si} on obtient un point d'intérêt ou la dérivée première est nulle, on peut trouver 
les maximums et minimums locaux, grâce au test de la dérivé seconde    
\end{Concept}

\begin{Definitionx*}{Maximum et minimum local}{}
  Soit $f^{\prime}\left(x\right) = 0$ \textcolor{myb}{\textbf{et}} $f^{\prime\prime}\left(x\right) < 0$, on a  
  un \textbf{maximum local} en x. \\\\
    Soit $f^{\prime}\left(x\right) = 0$ \textcolor{myb}{\textbf{et}} $f^{\prime\prime}\left(x\right) > 0$, on a  
  un \textbf{minimum local} en x.
\end{Definitionx*}

\chapter{Fonctions sinus et cosinus}
\begin{table}[H]
  \caption {Propriétés des fonctions sinus et cosinus}

  \begin{center}
    \renewcommand{\arraystretch}{1.5}
    \fontfamily{flr}\selectfont
    \footnotesize
    \begin{tabular}{l|l}
    \arrayrulecolor{blue}\hline
    \rowcolor{lightBlue}
    \textcolor{myb}{Propriété} & \textcolor{myb}{Descritpion}
    \\
    \hline
    \arrayrulecolor{black}
Domaine
& 
$\mathbb{R}$
\\
\hline
Continuité
&
Continue sur leur domaine  
\\
\hline
Croissance 
&
Toutes deux $2\pi$ périodiques. 
\\
\hline
\end{tabular}
\end{center}
\end{table}

\begin{Identite}{Cosinus pair et sinus impart}{}
  $\cos\left(-x\right) = \cos\left(x\right)$ \textcolor{myb}{\textbf{et}} 
  $\sin\left(-x\right) = -\sin\left(x\right)$
\end{Identite}

\begin{Identite}{Règle de dérivation de la fonction cosinus}{}
  $\dfrac{d}{dx}\cos\left(x\right) = -\sin\left(x\right)$
\end{Identite}

\begin{Identite}{Règle de dérivation de la fonction sinus}{}
  $\dfrac{d}{dx}\sin\left(x\right) = -\cos\left(x\right)$
\end{Identite}

\begin{Identite}{}{}
    $\cos^2\left(x\right) + \sin^2\left(x\right) = 1 $ 
\end{Identite}

\begin{Identite}{}{}
    $\cos\left(a +b \right) = \cos a \cos b -\sin a \sin b$
    $\sin\left(a +b \right) = -\sin a \cos b + \cos a \sin b$
\end{Identite}

\chapter{Limites des fonctions}
\section{Limite}
\begin{Definitionx*}{}{}
  $ \lim\limits_{x\to a}f(x)$ \textcolor{myb}{Converge vers L} si $f(x)$ est aussi proche que l'on veut de 
  $L$ \textbf{lorsque}   $x \to a$

\end{Definitionx*}

\section{Limite à droite et à gauche}
\begin{Definitionx*}{}{}
    Soit $f D\rightarrow \mathbb{R}$ \\ 
    \textcolor{myb}{La limite à droite} est la limite lorsque $x$ s'approche de a, venant de la 
    \textbf{droite}   : \\ 
    \[ \lim\limits_{x\to a^{+}} f(x) \implies  x \in D \text{ et } x > a \] 

    \textcolor{myb}{La limite à droite} est la limite lorsque $x$ s'approche de a, venant de la \textbf{gauche} : \\ 
    \[ \lim\limits_{x\to a^{+}} f(x) \implies  x \in D \text{ et } x < a \]   
  \textbf{Lorsque les deux limites sont équivalente}, \textcolor{myb}{la limite existe } :   
  
  \begin{center}
    $\lim\limits_{x\to a^{-}}f(x) = L = \lim\limits_{x\to a^{+}}f(x) $\\ 
    $\big\Updownarrow$ \\
    $\lim\limits_{x\to a} = L$
  \end{center}

  \textbf{Si aucun} $x < a$ :  
  
  \begin{center}
    $\lim\limits_{x\to a^{+}}f(x) = L $ \\ 
    $\big\Updownarrow$ \\
    $\lim\limits_{x\to a}f(x) = L $
  \end{center}

  \textbf{Si aucun} $x > a$ :  
  
  \begin{center}
    $\lim\limits_{x\to a^{-}}f(x) = L $ \\ 
    $\big\Updownarrow$ \\
    $\lim\limits_{x\to a}f(x) = L $
  \end{center}

\end{Definitionx*}

\begin{Definitionx*}{Divergence d'une fonction}{}
  $\lim\limits_{x\to a}f(x) $ \textbf{diverge}   si la limite ne converge vers aucun $L \in \mathbb{R}$
  \\
  \textcolor{myb}{Cas particulier} si 
  {\begin{enumerate}
      \varitem{blue!40}{\textbf{1}} $\lim\limits_{x\to a^{-}}f(x) = L $
      \varitem{blue!40}{\textbf{2}} $\lim\limits_{x\to a^{+}}f(x) = M $ 
      \varitem{blue!40}{\textbf{3}} $ L \neq M$ 
  \end{enumerate}}
  \begin{center}
  \textbf{alors}, $\lim\limits_{x\to a}f(x) $ \textcolor{myb}{diverge}. 
  \end{center}
\end{Definitionx*}


\section{Propriétés des limites}
\begin{Concept}{Addition, soustraction et multiplication de limite}{}
  Supposon que $\lim\limits_{x\to a}f(x) = L $ et $\lim\limits_{x\to a}g(x) \; | \; L, M \in \mathbb{R} $
  \begin{enumerate}
    \varitem{blue!40}{\textbf{1}} $\lim\limits_{x\to a}\Bigl(f(x) + g(x) \Bigr) $ 
    =  $\lim\limits_{x\to a}f(x)$ +  $\lim\limits_{x\to a}g(x) = L + M$

    \varitem{blue!40}{\textbf{1}} $\lim\limits_{x\to a}\Bigl(f(x) - g(x) \Bigr) $ 
    =  $\lim\limits_{x\to a}f(x)$ -  $\lim\limits_{x\to a}g(x) = L - M$ 

    \varitem{blue!40}{\textbf{1}} $\lim\limits_{x\to a}\Bigl(f(x)g(x) \Bigr) $ 
    =  $\lim\limits_{x\to a}f(x)$ $\lim\limits_{x\to a}g(x) = LM$

    \varitem{blue!40}{\textbf{1}} $\lim\limits_{x\to a}\Bigl(c f(x)\Bigr) =  c\lim\limits_{x\to a}f(x)$ 

    \varitem{blue!40}{\textbf{1}} $\lim\limits_{x\to a} \dfrac{f(x)}{g(x)} $ 
    =  $\dfrac{\lim\limits_{x\to a}f(x)}{\lim\limits_{x\to a}g(x)} = \dfrac{L}{M} \; si \; M \neq 0$

\end{enumerate}
\end{Concept}


\begin{Concept}{Comportement asymptotique et c > 0}{}
  Soit $\dfrac{\lim\limits_{x\to a}f(x)}{\lim\limits_{x\to a}g(x)} \; | \; 
  \lim\limits_{x\to a}f(x) = c \neq 0, \pm \infty$ si c > 0, on a 

\end{Concept}
  \begin{table}[h]
    \begin{center}
      \renewcommand{\arraystretch}{1.5}
      \fontfamily{flr}\selectfont
      \footnotesize
      \begin{tabular}{|l|l|l|l|l|}
      \arrayrulecolor{blue}\hline
      \rowcolor{lightBlue}
      \textcolor{myb}{Forme} & \textcolor{myb}{$\dfrac{c}{0^{+}}$} & \textcolor{myb}{$\dfrac{c}{0^{-}}$}
                             & \textcolor{myb}{$\dfrac{c}{\infty}$} & \textcolor{myb}{$\dfrac{c}{-\infty}$}
      \\
      \hline
      \hline
      \arrayrulecolor{black}
      tend vers & $\infty$ & $-\infty$ & 0 & 0 
      \\
      \hline
      

  \end{tabular}
  \end{center}
  \end{table}


\begin{Concept}{Comportement asymptotique et c < 0}{}
  Soit $\dfrac{\lim\limits_{x\to a}f(x)}{\lim\limits_{x\to a}g(x)} \; | \; 
  \lim\limits_{x\to a}f(x) = c \neq 0, \pm \infty$ si c < 0, on a 

\end{Concept}
  \begin{table}[h]
    \begin{center}
      \renewcommand{\arraystretch}{1.5}
      \fontfamily{flr}\selectfont
      \footnotesize
      \begin{tabular}{|l|l|l|l|l|}
      \arrayrulecolor{blue}\hline
      \rowcolor{lightBlue}
      \textcolor{myb}{Forme} & \textcolor{myb}{$\dfrac{c}{0^{+}}$} & \textcolor{myb}{$\dfrac{c}{0^{-}}$}
                             & \textcolor{myb}{$\dfrac{c}{\infty}$} & \textcolor{myb}{$\dfrac{c}{-\infty}$}
      \\
      \hline
      \hline
      \arrayrulecolor{black}
      tend vers & $-\infty$ & $\infty$ & 0 & 0 
      \\
      \hline
      

  \end{tabular}
  \end{center}
  \end{table}
\section{Continuité}
\begin{Concept}{Fonction continue}{}
  $f\text{:} \; D \rightarrow \mathbb{R}$ est \textcolor{myb}{continue}   en $a \in D$ si $\lim\limits_{x\to a} f(x) = f(a)$. 
  Autrement dit, une fonction est continue sur sont domaine si pour chaque élément $a \in D$, la limite 
  lorsque $x \rightarrow a$ est égale à $f(a)$. Et donc, la limite à gauche et à droite est approche 
  la même valeur $f(a)$
\end{Concept}

\begin{Identite}{Conséquence de la continuité de deux fonctions}{}
    Si $f$ et $g$ sont continues en $a$, alors 
    \begin{enumerate}
      \varitem{blue!40}{\textbf{1.}} $f+g$ et $fg$ sont continues en $a$ 
      \varitem{blue!40}{\textbf{2.}} $\dfrac{f}{g}$ est $fg$ continues en $a$ si $g(a) \neq 0$
      \varitem{blue!40}{\textbf{3.}} $f \circ g$ et $fg$ sont continues en $a$ si $f \circ g$ 
      est définie près de $a$
    \end{enumerate}
\end{Identite}

\section{Dérivée}
\begin{Definitionx*}{La dérivée d'une fonction}{}
  Soit $f^{\prime}(a) = \lim\limits_{h\to 0}\frac{f(a + h) - f(a)}{h} $ Si cette limite existe, on dit que 
  on dit qu'il s'agit de \textcolor{myb}{la dérivée de la fonction}   $f$ au point $a$. Géométriquement, 
  la valeur vers laquelle converge $f^{\prime}(a)$ correspond à la pente de la droite tengente en $a$.
\end{Definitionx*}

\section{Formule}
\begin{Concept}{Règles de dérivations pour des fonctions courantes}{}
  $ \textbf{1. }   (c)^{\prime} = 1$, c une constante \;\;
  $ \textbf{2. }   (x^r)^{\prime} = rx^{r -1}, \; \forall r \in \mathbb{R}$ \\\\
  $ \textbf{3. }   (a^x)^{\prime} = a^xln(a)\;\;$ 
  $ \textbf{4. }   (e^x)^{\prime} = e^x$ \;\; 
  $ \textbf{5. }   (\ln(x))^{\prime} = \dfrac{1}{x} $ \;\; 
  $ \textbf{6. }   (\log_a(x))^{\prime} = \dfrac{1}{x\ln(a)}$ \\\\\\\\
  $ \textbf{7. }   (\sin x)^{\prime} = \cos x$ \;\; 
  $ \textbf{8. }   (\cos x)^{\prime} = - \sin x$ \;\;
  $ \textbf{9. }   (\tan x)^{\prime} = -\sec^2 x$ \\\\\
  $ \textbf{10. }   (\arctan x)^{\prime} = \dfrac{1}{x^2 +1}$ \;\;
  $ \textbf{11. }   (\text{arcsec}x)^{\prime} = \dfrac{1}{x\sqrt{x^2 -1}}$ \;\;
  $ \textbf{11. }   (\arcsin x)^{\prime} = \dfrac{1}{x\sqrt{1 - x^2}}$ \;\;
  $ \textbf{11. }   (\arcsin x)^{\prime} = -\dfrac{1}{x\sqrt{1 - x^2}}$ \;\;

\end{Concept}

\section{Propriétés d'addition et de multiplication}
\begin{Concept}{Propriétés de la dérivée}{}
  Soit $f, g \text{:} \; I \rightarrow \mathbb{R}$ deux fonctions dérivables \\\\
  $\textbf{1. } \Bigl(cf(x)\Bigr)^{\prime} = cf^{\prime}(x)$, ou c est une constante. \\\\
  $\textbf{2. } \Bigl(f(x) + g(x)\Bigr)^{\prime} = cf^{\prime}(x) +  g^{\prime}(x)$ \\\\
  $\textbf{3. } f^{\prime}(x) = 0$ si et seulement si $f$ est une constante. 
\end{Concept}



\section{Règles de différenciation}
\begin{Concept}{Règles de calcul}{}
  $\textbf{1. Produit :} \Bigl(f(x)g(x)\Bigr)^{\prime} = f^{\prime}(x)g(x) + f(x)g^{\prime}(x)$ \\\\
  $\textbf{1. Quotient :} \Bigl(\dfrac{f(x)}{g(x)}\Bigr)^{\prime} = \dfrac{f^{\prime}(x)g(x) - f(x)g^{\prime}(x)}{g(x)^2}$ \\\\
  $\textbf{1. Dérivation en chaîne :} \Bigl(f(g(x)) \Bigr)^{\prime} = f^{\prime}\Bigl(g(x)\Bigr)g^{\prime}(x)$
\end{Concept} 

\section{Les formes indéterminées}
Toute expression représenté par une des formes suivantes est dite \textcolor{myb}{indéterminée} :
\begin{align*}
  \textcolor{red}{\dfrac{0}{0}}, \;\; \textcolor{red}{\dfrac{\infty}{\infty}}, \;\; 
  \textcolor{black}{\infty - \infty} \;\; \textcolor{myyellow}{1^{\infty}}, \;\; 
  0 \times \infty, \;\; \textcolor{myyellow}\infty^{0}, \;\; \textcolor{myyellow}{0^0}   
\end{align*}
Ces formes indéterminées peuvent être simplifiées en utilisant différentes techniques :
\begin{enumerate}
  \varitem{black}{} \textbf{Manipulations algébriques} facorisation, multipliucation par le conjugué, simplification
  \varitem{red}{} Règle de l'Hôpital 
  \varitem{myyellow}{} Utilisation du logarithme
\end{enumerate}

\chapter{Intégration}
\section{Définition d'une intégrale}

\begin{Concept}{Intégrale et théorème fondamental du calcul}{}
  Soit $f \text{:} \; [a,b] \rightarrow \mathbb{R} $ une fonction continue, \textcolor{myb}{l'intégrale} 
  de $a$ à $b$ de $f$ est noté :  
  \[ \int_{a}^{b}f(x)dx \]
\end{Concept}


\section{Propriétés de l'intégrale}

\begin{Concept}{}{}
$\text{Soit } f\text{:} \; [a,b] \rightarrow \mathbb{R}$ \textbf{Alors}, \\\\
$\textbf{1. } \int_{a}^{a}f(x)dx = 0 \textbf{ et } \int_{b}^{a}f(x)dx = - \int_{a}^{b}f(x)dx$ \\\\
$\textbf{2. Si } a < c < b,  \text{alors} \int_{a}^{b}f(x)dx = \int_{a}^{c}f(x)dx + \int_{c}^{b}f(x)dx $ \\\\
$\textbf{3.} \int_{a}^{b}cf(x)dx = c\int_{a}^{b}f(x)dx$ \\\\
$\textbf{4.} \int_{a}^{b}\Bigl(f(x) + g(x)\Bigr)dx = \int_{a}^{b}f(x)dx + \int_{a}^{b}g(x)dx$
\end{Concept}

\begin{Definitionx*}{}{}
  $\text{Soit } f\text{:} \; [a,b] \rightarrow \mathbb{R}$ continue, \\\\ 
  \begin{center}
  $$\textbf{1. } \text{Si on pose } \textcolor{myb}{F}^{\prime}(x) = \int_{a}^{b}f(t)dt, \text{ alors } 
  F^{\prime}(x) = f(t)$$ 
  \big\Updownarrow
  $$\dfrac{d}{dx}\int_{a}^{b}f(t)dt = f(x)$$
  \end{center}

  \begin{center}
  $$\textbf{2. } \text{Soit } F \text{ telle que } F^{\prime}(x) = f(x), \text{ Alors} 
  \int_{a}^{b}f(x)dx = F(x)  \bigg|_{a}^{b} = F(b) - F(a) $$
  \big\Updownarrow
  $$\int_{a}^{b}\dfrac{d}{dt}f(t)dt  f(b) - f(a)$$ 
\end{center}
\end{Definitionx*}


\section{Trouver l'aire sous la courbe}


\chapter{Techniques de bases}
\section{Polynôme}
\begin{Definitionx*}{}{}
  Un polynôme a la forme $p(x) = a_0 + a_1x + \cdots + a_nx^n$ et la \textcolor{myb}{puissance} d'un polynôme 
  est l'exposant le plus elevé de l'expression.    
\end{Definitionx*}

\begin{note}{}{}
    Lorsque le degré du numérateur est plus grand ou égal au degré du dénominateur, on peut effectuer une division 
    polynomiale pour simplifier une expression : 
    \begin{align*}
      \dfrac{x^-1}{x^2 +1} = 1 - \dfrac{2}{x^2 +1}
    \end{align*}
\end{note}


\begin{Concept}{Complétion du carré}{}
  Soit un polynôme $p = ax^2 + bx + c$, on peut compléter le carré en considérant : 
  \begin{center}
  $h = \left(\dfrac{b}{2}\right)^2$ \\ 
  $p(x) = a\Bigl(x^2 + \dfrac{b}{a}x - \textcolor{blue}{\left(\dfrac{b}{2a}\right)^2} 
  - \textcolor{red}{\left(\dfrac{b}{2a}\right)^2}\Bigr)  + c  $ \\
  $p(x) = a\Bigl(x^2 + \dfrac{b}{a} x - \textcolor{blue}{\dfrac{b^2}{4a}}\Bigr) + \textcolor{red}{\dfrac{b^2}{4a}}  + c  $ \\ 
  $p(x) = a\Bigl(x - \textcolor{blue}{\dfrac{b}{2}}\Bigr)^2 +  
  \textcolor{red}{\dfrac{b^2}{4a}}  + c  $

  \end{center}
  
\end{Concept}

\end{multicols*}
\end{document}

